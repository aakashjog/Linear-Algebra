\documentclass[fleqn]{article}
\setcounter{secnumdepth}{4}
\usepackage{amsmath, amssymb}
\usepackage{mathtools}
\usepackage{datetime}
\usepackage{ulem}
\usepackage{enumerate}
\newcommand\numberthis{\addtocounter{equation}{1}\tag{\theequation}}
\newenvironment{amatrix}[1]{%
	\left(\begin{array}{@{}*{#1}{c}|c@{}}
}{%
	\end{array}\right)
}

\setlength{\mathindent}{0pt}

%opening
\title{Recitation 2}
\author{}
\date{\formatdate{5}{11}{2014}}

\begin{document}

\maketitle
\setlength{\mathindent}{0pt}

\tableofcontents

\newpage
\section{Row Echelon Form and Soving Systems of Linear Equations}

\subsection{Solve the following system of linear equations}

\begin{align*}
	x + 2y - 3z &= 4\\
	x + 3y + x &= 11\\
	2x + 5y - 4z &= 12
\end{align*}

\paragraph*{Solution:\\}

\begin{align*}
	A &= 
	\begin{pmatrix}
		1 & 2 & -3 \\
		1 & 3 & 1 \\
		2 & 5 & -4 \\
	\end{pmatrix}\\
	x &= 
	\begin{pmatrix}
		x\\
		y\\
		z\\
	\end{pmatrix}\\
	b &= 
	\begin{pmatrix}
		4\\
		11\\
		13\\
	\end{pmatrix}
\end{align*}

$Ax = b$ is the matrix form of the system.\\
The augmented matrix is $(A|b)$.
\begin{equation*}
	(A|b) = 
	\begin{amatrix}{3}
	1 & 2 & -3 & 4\\
	1 & 3 & 1 & 11\\
	2 & 5 & -4 & 13\\
	\end{amatrix}
\end{equation*}
We will try to bring the augmented matrix into reduced REF, i.e. of the form 
$\begin{amatrix}{3}
	1 & 0 & 0 & b_1\\
	0 & 1 & 0 & b_2\\
	0 & 0 & 1 & b_3\\
\end{amatrix}$
In order to transform the matrix to \emph{Reduced REF}. \\
We can do \emph{one} of the following operations at each time.
\begin{enumerate}
	\item $R_i \rightarrow c R_i$ \label{row }
	\item $R_i \leftrightarrow R_j$
	\item $R_i \rightarrow R_i + c R_j$
\end{enumerate}
These elementary operations preserve the set of elements.

\subsection{Gaussian Eliminiation}

\begin{enumerate}
	\item We will make sure that in the upper left corner we have an element different from $0$. If we don't, we will switch the $1^{\text{st}}$ row with another row. \\
	If all elements in column 1 are zeros, we will ignore this column and consider the next one.
	\item We will multiply the first row with a constant such that the first element will be $1$
	\item We will cancel all other elements in the first column, except the one in the first row, by elementary row operation $R_i \rightarrow R_i + c R_j$
	\item We will repeat the above steps, ignoring the last row and last column, until we get an upper-triangular matrix.
\end{enumerate}

\subsection{Find the solutions of }

\subsubsection{}

\begin{align*}
	x + 2y - 3z &= -1\\
	3x - y + 2z &= 7\\
	5x + 3y - 4z &= 2\\
\end{align*}

\paragraph*{Solution:\\}
$
	\begin{amatrix}{3}
		1 & 2 & -3 & -1\\
		3 & -1 & 2 & 7\\
		5 & 3 & -4 & 2\\
	\end{amatrix}
	\xrightarrow [R_3 \rightarrow R_3 - 5 R_1] {R_2 \rightarrow R_1 - 3 R_1}
	\begin{amatrix}{3}
		1 & 2 & -3 & -1\\
		0 & -7 & 11 & 10\\
		0 & -7 & 11 & 7\\
	\end{amatrix}
	\xrightarrow{R_2 \rightarrow -\frac{1}{7} R_2}
	\begin{amatrix}{3}
		1 & 2 & -3 & -1\\
		0 & 1 & -\dfrac{11}{7} & -\dfrac{10}{7}\\
		0 & -7 & 11 & 7\\
	\end{amatrix}
	\xrightarrow{R_3 \rightarrow R_3 + 7 R_2}
	\begin{amatrix}{3}
		1 & 2 & -3 & -1\\
		0 & 1 & -\dfrac{11}{7} & -\dfrac{10}{7}\\
		0 & 0 & 0 & -3\\
	\end{amatrix}
$
\begin{equation*}
	0 \neq -3 \Rightarrow \text{No solution}
\end{equation*}

\subsubsection{}

\begin{align*}
x + 2y - 3z &= 6\\
2x - y + 4z &= 2\\
4x + 3y - 2z &= 14
\end{align*}

\paragraph*{Solution:\\}

$
\begin{amatrix}{3}
	1 & 2 & -3 & 6 \\
	2 & -1 & 4 & 2\\
	4 & 3 & -2 & 14\\
\end{amatrix}
\xrightarrow[R_3 \rightarrow R_3 - 4 R_1]{R_2 \rightarrow R_2 - 2 R_1}
\begin{amatrix}{3}
	1 & 2 & -3 & 6\\
	0 & -5 & 10 & -10\\
	0 & -5 & 10 & -10\\
\end{amatrix}
\xrightarrow{R_2 \rightarrow -\frac{1}{5} R_2}
\begin{amatrix}{3}
	1 & 2 & -3 & 6\\
	0 & 1 & -2 & 2\\
	0 & -5 & 10 & -10\\
\end{amatrix}
\xrightarrow{R_3 \rightarrow R_3 + 5 R_2}
\begin{amatrix}{3}
	1 & 2 & -3 & 6\\
	0 & 1 & -2 & 2\\
	0 & 0 & 0 & 0 \\
\end{amatrix}
\xrightarrow{R_1 \rightarrow R_1 - 2 R_2}
\begin{amatrix}{3}
	1 & 0 & 1 & 2\\
	0 & 1 & -2 & 2\\
	0 & 0 & 0 & 0\\
\end{amatrix}
$
\begin{align*}
x + z &= 2\\
y - 2z &=2\\
\end{align*}
$z$ is a free variable and $x,y$ are dependant variables.
\begin{equation*}
\therefore x = 
\begin{pmatrix}
	2 - t\\
	2t + 2\\
	t\\
\end{pmatrix}
\end{equation*}
Therefore, the system of equations has infinite number of solutions.

\subsection{Check if the following matrices are invertible and find the inverse if it exists.}

\subsubsection{}

\begin{equation*}
	\begin{pmatrix*}
		1 & 0 & 2\\
		2 & -1 & 3\\
		4 & 1 & 8\\
	\end{pmatrix*}
\end{equation*}

\paragraph{Solution:\\}

$
\begin{pmatrix}
	1 & 0 & 2 & 1 & 0 & 0\\
	2 & -1 & 3 & 0 & 1 & 0\\
	4 & 1 & 8 & 0 & 0 & 1\\
\end{pmatrix}
\xrightarrow[R_3 \rightarrow R_3 - 4 R_1]{R_2 \rightarrow R_2 - 2 R_1}
\begin{pmatrix}
	1 & 0 & 2 & 1 & 0 & 0\\
	0 & 1 & 1 & 2 & 1 & 0\\
	0 & 1 & 0 & -4 & 0 & 1\\
\end{pmatrix}
\xrightarrow{R_3 \rightarrow R_3 - R_2}
\begin{pmatrix}
	1 & 0 & 2 & 1 & 0 & 0\\
	0 & 1 & 1 & 2 & -1 & 0\\
	0 & 0 & 0 & 6 & -1 & -1\\
\end{pmatrix}
\xrightarrow[R_2 \rightarrow R_2 - R_3]{R_1 \rightarrow R_1 - 2 R_3}
\begin{pmatrix}
	1 & 0 & 0 & -11 & 2 & 2\\
	0 & 1 & 0 & -4 & 0 & 1\\
	0 & 0 & 1 & 6 & -1 & -1\\
\end{pmatrix}
$

\subsection{Invertible Matrices}

A square matrix is invertible iff its reduced REF is $I$.\\
\begin{align*}
	(A|I) &\xrightarrow{\text{Gaussian Elimination}} (I|B)\\
	\therefore B \cdot A &= I \\
	\therefore B & = A^{-1}
\end{align*}

\end{document}
