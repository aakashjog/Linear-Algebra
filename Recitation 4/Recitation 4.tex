\documentclass[fleqn, a4paper, 12pt]{article}
\setcounter{secnumdepth}{4}
\usepackage{amsmath, amssymb, amsthm}
\usepackage{mathtools}
\usepackage{datetime}
\usepackage{ulem}
\usepackage{enumerate}

\newcommand\numberthis{\addtocounter{equation}{1}\tag{\theequation}}

\newenvironment{amatrix}[1]{%
	\left(\begin{array}{@{}*{#1}{c}|c@{}}
}{%
	\end{array}\right)
}

%\setlength{\mathindent}{0pt}

\theoremstyle{definition}
\newtheorem{example}{Example}
\newtheorem{definition}{Definition}

\theoremstyle{theorem}
\newtheorem{theorem}{Theorem}

\newenvironment{solution}
{\begin{proof}[Solution]\let\qed\relax}
	{\end{proof}}

%opening
\title{Recitation 4}
\author{}
\date{\formatdate{19}{11}{2014}}

\begin{document}

\maketitle
%\setlength{\mathindent}{0pt}

\tableofcontents

\newpage
\section{Linear Vector Spaces}

\subsection{Examples}

\begin{enumerate}
	\item $\mathbb{R}^n = \{(x_1, \dots, x_n) ; x_1, \dots, x_n \in \mathbb{R}\}$
	\item $\mathbb{C}^n = \{(x_1, \dots, x_n) ; x_1, \dots, x_n \in \mathbb{C}\}$
	\item $C([0, 1]) = \{f:[0,1] \rightarrow \mathbb{R} ; f \text{ is continuous}\}$\\
	where $(f+g)(x) = f(x) + g(x), (\alpha f)(x) = \alpha f(x) ; \alpha \in \mathbb{R}, f \in C([0, 1])$
	\item Matrices over $\mathbb{F}$ with matrix addition and scalar multiplication as per standard definitions.
	\item $\mathbb{R}_n[x] = \{p(x) = p_0 + p_1 x + \dots + p_n x^n ; p_0, \dots, p_n \in \mathbb{R}\}$\\
	where $(p+q)(x) = (p_0 + q_0) + \dots + (p_n + q_n) x^n, (\alpha p)(x) = \alpha (p(x)) = (\alpha p_0) + \dots + (\alpha p_n) x^n$
\end{enumerate}

\subsection{Exercises}

\begin{example}
	Is
	\begin{equation*}
		W_1 = \{(a, b, c) \in \mathbb{R}^3 ; a + b + c = 0\}
	\end{equation*}
	a subspace of $\mathbb{R}^3$?
\end{example}

\begin{solution}
	\begin{align*}
		0 + 0 + 0 = 0 &\Rightarrow (0, 0, 0) \in W_1
	\end{align*}
	\begin{align*}
		v, u &\in W_1\\
		\therefore v_1 + v_2 + v_3 = u_1 + u_2 + u_3 &= 0 \\
		\therefore v_1 + u_1 + v_2 + u_2 + v_3 + u_2 &= 0 \\
		\therefore v+u = (v_1 + u_1, v_2 + u_2, v_3 + u_3) &\in W_1
	\end{align*}
	\begin{align*}
		\alpha &\in \mathbb{R} \\
		v = (v_1, v_2, v_3) &\in W_1 \\
		\therefore v_1 + v_2 + v_3 &= 0 \\
		\therefore \alpha v &= (\alpha v_1, \alpha v_2, \alpha v_3) \\
		\alpha v_1 + \alpha v_2 + \alpha v_3 &= 0 \\
		\therefore \alpha v &\in \mathbb{R}
	\end{align*}
	\end{solution}

	\begin{example}
		Is
		\begin{equation*}
		W_2 = \{(a, b, c) \in \mathbb{R}^3 ; a \geq 0\}
		\end{equation*}
		a subspace of $\mathbb{R}^3$?
	\end{example}

	\begin{solution}
		$W_2$ is not a linear subspace of $\mathbb{R}^3$, as for $\alpha = -1, v = (1, 0, 0)$, $\alpha v = (-1, 0, 0) \notin W_2$
	\end{solution}
	
	\begin{example}
		Is
		\begin{equation*}
		W_3 = \{p(x) \in \mathbb{R}_3[x] ; p(0) = 1\}
		\end{equation*}
		a subspace of $\mathbb{R}^3$?
	\end{example}
	
	\begin{solution}
		\begin{align*}
			0(0) = 0 &\neq 1 \\
			\therefore 0 &\notin W_3
		\end{align*}
		Hence, $W_3$ is not a linear subspace of $R_3[x]$.
	\end{solution}
	
	\begin{example}
		Show that the solutions space of homogeneous linear system is a linear subspace.
	\end{example}

	\begin{solution}
		Let $A$ be the matrix representing the homogeneous matrix, with $n$ variables over $\mathbb{F}$. 
		\begin{equation*}
			N(A) = \{x ; x \in \mathbb{F}^n, Ax = 0\}
		\end{equation*}
		\begin{align*}
			x = 0 &\in N(A) \\
			\therefore A \cdot 0 &= 0	\\
			\therefore 0 &\in N(A)
		\end{align*}
		\begin{align*}
			x, y &\in N(A) \\
			A x = A y &= 0 \\
			\therefore A(x + y) &= A x + A y = 0 \\
			\therefore x + y &\in N(A)
		\end{align*}
			\begin{align*}
			\lambda &\in \mathbb{F} \\
			x &\in N(A) \\
			\therefore A x = 0\\
			\therefore A(\lambda x) &= \lambda (A x) \\
			&= \lambda \cdot 0 \\
			&= 0 \\
			\therefore \lambda x &\in N(A)
			\end{align*}
			Therefore, $N(A)$ is a linear subspace.
	\end{solution}

\subsection{Linear Combinations}

Let $V$ be a linear vector space over $\mathbb{F}$ and $K \subset V$ be a finite set of vectors from $V$.
\begin{equation*}
	k = \{v_1, \dots, v_n\}
\end{equation*}
Every expression of the form 
\begin{equation*}
	\sum_{i=1}^{n} \alpha_i v_i = 0 ; \alpha_1, \dots, \alpha_n \in \mathbb{F}
\end{equation*}
is called a linear combination of $K$.\\
$K$ is said to be linearly dependant if $\exists \alpha_1, \dots, \alpha_n$, not all zeros, s.t. 
\begin{equation*}
	\sum_{i=1}^{n} \alpha_i v_i = 0
\end{equation*}
otherwise, $K$ is said to be linearly dependant, i.e.
\begin{equation*}
	\alpha_1 = \dots = \alpha_n = 0
\end{equation*}
\end{document}
