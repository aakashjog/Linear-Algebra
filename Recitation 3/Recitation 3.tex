\documentclass[fleqn]{article}
\setcounter{secnumdepth}{4}
\usepackage{amsmath, amssymb}
\usepackage{mathtools}
\usepackage{datetime}
\usepackage{ulem}
\usepackage{enumerate}

\newcommand\numberthis{\addtocounter{equation}{1}\tag{\theequation}}

\newenvironment{amatrix}[1]{%
	\left(\begin{array}{@{}*{#1}{c}|c@{}}
}{%
	\end{array}\right)
}

%\setlength{\mathindent}{0pt}

%opening
\title{Recitation 3}
\author{}
\date{\formatdate{12}{11}{2014}}

\begin{document}

\maketitle
%\setlength{\mathindent}{0pt}

\tableofcontents

\newpage
\section{Matrix Determinant}

The determinant function maps each square matrix $A \in M_n (\mathbb{F})$ to a scalar $|A| = \det(A) \in \mathbb{F}$.\\ 
For $n = 1$, we define $\det((a)) = a$.\\
Let $A$ be a square matrix of dimensions $n \times n$. \\
The $(i, j)$ minor of $A$, $M_{ij}$, is the determinant of the $(n-1) \times (n-1)$ matrix obtained by removing the $i^{\text{th}}$ row and $j^{\text{th}}$ column of $A$.\\
The determinant can be developed using the $i^{\text{th}}$ row as
\begin{equation*}
	\det (A) = \sum_{k = 1}^{n} (-1)^{i+k} a_{ik} M_{ik}
\end{equation*}
The determinant can also be developed using the $j^{\text{th}}$ column as
\begin{equation*}
\det (A) = \sum_{k = 1}^{n} (-1)^{k+j} a_{kj} M_{kj}
\end{equation*}

\subsection{Properties}

\subsubsection{If $B$ is formed by multiplying one row (or column) of $A$ by a scalar $c$, then, $\det (B) = c \cdot \det (A)$}

\subsubsection{If $B$ is formed by replacing two rows (or columns) of $A$ by each other, then, $\det (B) = - \det (A)$}

\subsubsection{If $B$ is formed by adding one row (or column), multiplied by a scalar, to another row (or column), then, $\det (B) = \det (A)$}

\subsubsection{If $A$ has atleast one row or column having all elements equal to zero, then, $\det (A) = 0$.}

\subsubsection{The determinant of an upper-triangular or lower-triangular matrix is equal to the product of the elements on the principal diagonal.}

\subsubsection{$\det (A^T) = \det (A)$}

\subsubsection{$\det (A \cdot B) = \det (A) \cdot \det (B)$}

\subsubsection{Matrix $A$ is invertible iff $\det (A) \neq 0$. If so, $\det(A^{-1}) = \dfrac{1}{\det (A)}$}

\subsection{Examples}

\subsubsection{Let $A, B, P$ be invertible matrices, satisfying $B = P^{-1} A P$. Show that $|A^{-1} B| = 1$.}

Multiplying the given equation by $A^{-1}$ from the left, 
\begin{align*}
	A^{-1} B &= A^{-1} P^{-1} A P \\
	\therefore |A^{-1} B| &= |A^{-1} P^{-1} A P|\\
	&= |A^{-1}| |P^{-1}| |A| |P| \\
	&= \dfrac{|A| |P|}{|A| |P|} \\
	&= 1
\end{align*}

\subsection{Cramer's Rule}

Let $A \in M_n(\mathbb{F})$ and $b \in \mathbb{F}^n$. If $|A| \neq 0$, the system $A x = b$ has a unique solution, whose individual values are given by
\begin{equation*}
	x_j = \dfrac{|A_j|}{|A|}
\end{equation*}
where $A_j$ is the matrix formed by replacing the $j^{\text{th}}$ column of $A$ by the vector $b$.

\section{Adjoint Matrix}

A square matrix of dimensions $n \times n$ is said to be adjoint to $A \in M_n(\mathbb{F})$ if $(\text{adj}(A))_{ij} = (-1)^{i + j} M_{ji}$.
\begin{equation*}
	A \cdot \text{adj}(A) = |A| \cdot I
\end{equation*}
If $|A| \neq 0$
\begin{equation*}
	A^{-1} = \dfrac{\text{adj}(A)}{|A|}
\end{equation*}

\end{document}
