\documentclass[fleqn, a4paper, 12pt]{article}
\setcounter{secnumdepth}{4}
\usepackage{amsmath, amssymb, amsthm}
\usepackage{mathtools, xfrac}
\usepackage{datetime}
\usepackage{ulem}
\usepackage{enumerate}

\newcommand\numberthis{\addtocounter{equation}{1}\tag{\theequation}}

\newcommand{\rank}{\mathrm{rk\,}}

\newcommand{\R}{\mathrm{R}}

\newcommand{\C}{\mathrm{C}}

\newcommand{\im}{\mathrm{im}\,}

\newenvironment{amatrix}[1]{%
	\left(\begin{array}{@{}*{#1}{c}|c@{}}
}{%
	\end{array}\right)
}

%\setlength{\mathindent}{0pt}

\theoremstyle{definition}
\newtheorem{example}{Example}
\newtheorem{definition}{Definition}

\theoremstyle{theorem}
\newtheorem{theorem}{Theorem}

\newenvironment{solution}
{\begin{proof}[Solution]\let\qed\relax}
	{\end{proof}}

\DeclareMathOperator{\vspan}{\mathrm{span}} %declares span operator for matrices

%opening
\title{Recitation 9}
\author{}
\date{\formatdate{24}{12}{2014}}

\begin{document}

\maketitle
%\setlength{\mathindent}{0pt}

\tableofcontents

\newpage
\section{Kernel and Image}

If $\{u_1, \dots, u_n\}$ is a basis of $U$, then $\im (T) = \vspan\{T(u_1), \dots, T(u_n)\} $

\begin{example}
	Find a basis for $\im (T)$.
	\begin{align*}
		T &: \mathbb{R}^2 \to \mathbb{R}^2 & T(x, y) &= (x-y, y-x)
	\end{align*}
\end{example}

\begin{solution}
	\begin{align*}
		\im (T) &= \vspan \{T(e_1), T(e_2)\}\\
		&= \vspan\{T(1,0), T(0,1)\}\\
		&= \vspan\{(1, -1), (-1, 1)\}\\
		&= \vspan\{(1, -1)\}
	\end{align*}
\end{solution}

\begin{example}
	Find a basis for $\im (T)$.
	\begin{align*}
	T &: \mathbb{R}_2 [x] \to M_2 (\mathbb{R}) & T(ax^2 + bx + c) &= 
		\begin{pmatrix}
			b & c\\
			-c & a\\
		\end{pmatrix}
	\end{align*}
\end{example}

\begin{solution}
	\begin{align*}
		\im (T) &= \vspan \{T(e_1), T(e_2), T(e_3)\}\\
		&= \vspan\{T(1), T(x), T(x^2)\}\\
		&= \vspan
			\{
				\begin{pmatrix}
					0 & 1\\
					-1 & 0\\
				\end{pmatrix}
				,
				\begin{pmatrix}
					1 & 0\\
					0 & 0\\
				\end{pmatrix}
				,
				\begin{pmatrix}
					0 & 0\\
					0 & 1\\
				\end{pmatrix}
			\}
	\end{align*}
\end{solution}

\begin{example}
	Let
	\begin{equation*}
		T : \mathbb{R}^5 \to \mathbb{R}^5
	\end{equation*}
	Is it possible that $\dim (\im(T)) = \dim (\ker(T))$?
\end{example}

\begin{solution}
	\begin{align*}
		\dim (\im(T)) + \dim (\ker(T)) &= \dim(V)
	\end{align*}
	Therefore, if $\dim (\im(T)) = \dim (\ker(T))$, $\dim (\im(T)) = \dim (\ker(T)) = \dfrac{5}{2}$. Therefore such a case is not possible.
\end{solution}

\section{Linear Maps}

\begin{definition}[Isomorphism]
	A linear map $T$ is called an isomorphism, if it is one-to-one and onto.
\end{definition}

The representation matrix is invertible if and only if the map is an isomorphism.

\begin{example}
	\begin{align*}
		T &: \mathbb{R}^3 \to M_{2 \times 2} (\mathbb{R}) & T((x,y,x)) &= 
		\begin{pmatrix}
			y & z\\
			-z & x\\
		\end{pmatrix}
	\end{align*}
	Find the representation matrix of $T$ corresponding to the bases
	\begin{align*}
		B &= 
			\left\lbrace
				u_1 = 
					\begin{pmatrix}
						1\\
						1\\
						1\\
					\end{pmatrix}
				,
				u_2 = 
					\begin{pmatrix}
						1\\
						1\\
						0\\
					\end{pmatrix}
				,
				u_3 = 
					\begin{pmatrix}
						1\\
						1\\
						0\\
					\end{pmatrix}
			\right\rbrace\\
		C &= 
			\left\lbrace
				v_1 = 
					\begin{pmatrix}
						1 & 0\\
						0 & 1\\
					\end{pmatrix}
				,
				u_2 = 
					\begin{pmatrix}
						1 & 0\\
						0 & -1\\
					\end{pmatrix}
				,
				u_3 = 
					\begin{pmatrix}
						0 & 1\\
						1 & 0\\
					\end{pmatrix}
				,
				u_4 = 
					\begin{pmatrix}
						0 & 1\\
						-1 & 0\\
					\end{pmatrix}
			\right\rbrace
	\end{align*}
	Further, find $T(1, -2, 1)$.
\end{example}

\begin{solution}
	\begin{align*}
		[T]_{B,C} &= 
			\begin{pmatrix}
				[T(u_1)]_C & [T(u_2)]_C & [T(u_3)]_C\\
			\end{pmatrix}\\
		&= 
		\begin{pmatrix}
			\left[
				\begin{pmatrix}
					1 & 1\\
					-1 & 1\\
				\end{pmatrix}
			\right]_C 
			& 
			\left[
				\begin{pmatrix}
					1 & 0\\
					0 & 1\\
				\end{pmatrix}
			\right]_C 
			&
			\left[
				\begin{pmatrix}
					0 & 0\\
					0 & 1\\
				\end{pmatrix}
			\right]_C 
		\end{pmatrix}\\
		&= 
			\begin{pmatrix}
				1 & 1 & \sfrac{1}{2}\\
				0 & 0 & -\sfrac{1}{2}\\
				0 & 0 & 0\\
				1 & 0 & 0\\
			\end{pmatrix}
	\end{align*}
	\begin{align*}
		[T(1, -2, 1)] = [T]_{B, C} [(1, -2, 1)]_{B}
	\end{align*}
\end{solution}

\end{document}
