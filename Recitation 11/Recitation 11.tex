\documentclass[fleqn, a4paper, 12pt]{article}
\setcounter{secnumdepth}{4}
\usepackage{amsmath, amssymb, amsthm}
\usepackage{mathtools, xfrac}
\usepackage{datetime}
\usepackage{ulem}
\usepackage{enumerate}
\usepackage[table]{xcolor}

\newcommand\numberthis{\addtocounter{equation}{1}\tag{\theequation}}

\newcommand{\rank}{\mathrm{rk\,}}

\newcommand{\R}{\mathrm{R}}

\newcommand{\C}{\mathrm{C}}

\newcommand{\N}{\mathrm{N}}

\newcommand{\im}{\mathrm{im}\,}

\newenvironment{amatrix}[1]{%
	\left(\begin{array}{@{}*{#1}{c}|c@{}}
}{%
	\end{array}\right)
}

%\setlength{\mathindent}{0pt}

\theoremstyle{definition}
\newtheorem{example}{Example}
\newtheorem{definition}{Definition}

\theoremstyle{theorem}
\newtheorem{theorem}{Theorem}

\newenvironment{solution}
{\begin{proof}[Solution]\let\qed\relax}
	{\end{proof}}

\DeclareMathOperator{\vspan}{\mathrm{span}} %declares span operator for matrices

%opening
\title{Recitation 11}
\author{}
\date{\formatdate{4}{1}{2015}}

\begin{document}

\maketitle
%\setlength{\mathindent}{0pt}

\tableofcontents

\newpage
\section{Algebraic and Geometric Multiplicity}

\begin{example}
	\begin{align*}
		A &=
			\begin{pmatrix}
				-1 & 2 & 0\\
				0 & 1 & 2\\
				0 & 2 & 1\\
			\end{pmatrix}
	\end{align*}
	Find the eigenvalues and algebraic multiplicity and eigenspaces with geometric multiplicity.
\end{example}

\begin{solution}
	\begin{align*}
		\det(A - \lambda I) &= 
			\begin{vmatrix}
				-1 - \lambda & 2 & 0\\
				0 & 1 - \lambda & 2\\
				0 & 2 & 1 - \lambda\\
			\end{vmatrix}
		&= (-1 - \lambda)((1 - \lambda)^2 - 4)\\
		&= - (\lambda + 1)^2 (\lambda - 3)
	\end{align*}
	Therefore,\\
	\begin{tabular}{|c|c|c|c|}
		\hline
		Eigenvalue & Algebraic Multiplicity & Eigenspace & Geometric Multiplicity\\
		\hline
		-1 & 2 & $\vspan
			\left\lbrace
				\begin{pmatrix}
					1\\
					0\\
					0\\
				\end{pmatrix}
			\right\rbrace
			$
		& 1\\
		\hline
		3 & 1 & $\vspan
			\left\lbrace
				\begin{pmatrix}
				1\\
				2\\
				2\\
				\end{pmatrix}
			\right\rbrace
			$
		& 1\\ 
		\hline
	\end{tabular}
\end{solution}

\begin{example}
	Check if the following matrix is diagonalizable, and if it is, diagonalize it.
	\begin{equation*}
		A =
			\begin{pmatrix}
				4 & 0 & -3\\
				-3 & 1 & 3\\
				6 & 0 & -5\\
			\end{pmatrix}
	\end{equation*}
\end{example}

\begin{solution}
	\begin{align*}
		A &=
		\begin{pmatrix}
			4 & 0 & -3\\
			-3 & 1 & 3\\
			6 & 0 & -5\\
		\end{pmatrix}\\
		\therefore p_A(x) &= |A - \lambda I|\\
		&= 
			\begin{vmatrix}
				4 - \lambda & 0 & -3\\
				-3 & 1 - \lambda & 3\\
				6 & 0 & -5 - \lambda\\
			\end{vmatrix}\\
		&= (4 - \lambda) (1 - \lambda) (-5 - \lambda) + 3 (1 - \lambda) 6\\
		&= (1 - \lambda) ((4 - \lambda)(-5 - \lambda) + 18)\\
		&= -(\lambda - 1)^2 (\lambda + 2)\\
		\therefore \lambda &= 1, -2
	\end{align*}
	\begin{align*}
		V_1 &= \N 
			\begin{pmatrix}
				3 & 0 & -3\\
				-3 & 0 & 3\\
				6 & 0 & -6\\
			\end{pmatrix}\\
		&= \N
			\begin{pmatrix}
				1 & 0 & -1\\
				0 & 0 & 0\\
				0 & 0 & 0\\
			\end{pmatrix}\\
		&= \vspan
			\left\lbrace
				\begin{pmatrix}
					0\\
					1\\
					0\\
				\end{pmatrix}
				,
				\begin{pmatrix}
					1\\
					0\\
					1\\
				\end{pmatrix}
			\right\rbrace
	\end{align*}
	Therefore,\\
	\begin{tabular}{|c|c|c|c|}
		\hline
		Eigenvalue & Algebraic Multiplicity & Eigenspace & Geometric Multiplicity\\
		\hline
		1 & 2 & $\vspan
			\left\lbrace
				\begin{pmatrix}
					0\\
					1\\
					0\\
				\end{pmatrix}
				,
				\begin{pmatrix}
					1\\
					0\\
					1\\
				\end{pmatrix}
			\right\rbrace
			$
		& 2\\
		\hline
		-1 & 1 & $\vspan
			\left\lbrace
				\begin{pmatrix}
					-1\\
					1\\
					-2\\
				\end{pmatrix}
			\right\rbrace
			$
		& 1\\ 
		\hline
	\end{tabular}
	Therefore, the geometric multiplicity is 3. Hence, $A$ is diagonalizable.\\
	Therefore,
	\begin{align*}
		P &=
			\begin{pmatrix}
				0 & 1 & -1\\
				1 & 0 & 1\\
				0 & 1 & -2\\
		\end{pmatrix}\\
		D &= 
			\begin{pmatrix}
				1 & 0 & 0\\
				0 & 1 & 0\\
				0 & 0 & -2\\
			\end{pmatrix}
	\end{align*}
\end{solution}

\begin{example}
	Check if the following matrix is diagonalizable, and if it is, diagonalize it.
	\begin{equation*}
	A =
		\begin{pmatrix}
			4 & 0 & -3\\
			1 & 1 & 1\\
			6 & 0 & 5-\\
		\end{pmatrix}
	\end{equation*}
\end{example}

\begin{solution}
	\begin{align*}
		A &=
			\begin{pmatrix}
				4 & 0 & -3\\
				1 & 1 & 1\\
				6 & 0 & 5-\\
			\end{pmatrix}\\
			\therefore p_A(x) &= |A - \lambda I|\\
		&= 
		\begin{vmatrix}
			4 - \lambda & 0 & -3\\
			1 & 1 - \lambda & 1\\
			6 & 0 & -5 - \lambda\\
		\end{vmatrix}\\
		&= (4 - \lambda) (1 - \lambda) (-5 - \lambda) + 3 (1 - \lambda) 6\\
		&= (1 - \lambda) ((4 - \lambda)(-5 - \lambda) + 18)\\
		&= -(\lambda - 1)^2 (\lambda + 2)\\
	\end{align*}
		Therefore,\\
	\begin{tabular}{|c|c|c|c|}
		\hline
		Eigenvalue & Algebraic Multiplicity & Eigenspace & Geometric Multiplicity\\
		\hline
		1 & 2 & $\vspan
			\left\lbrace
				\begin{pmatrix}
					0\\
					1\\
					0\\
				\end{pmatrix}
			\right\rbrace
			$
		& 1\\
		\hline
		-1 & 1 & & 1\\ 
		\hline
	\end{tabular}
	Hence, by the explicit criterion for diagonalization, $B$ cannot be diagonalized.
\end{solution}

\begin{example}
	Let $V = M_{2 \times 2} (\mathbb{R})$ and $T : V \to V$.
	\begin{equation*}
		T 
			\begin{pmatrix}
				a & b\\
				c & d\\
			\end{pmatrix}
		=
			\begin{pmatrix}
				d & a - 2b\\
				c & a\\
			\end{pmatrix}
	\end{equation*}
	Find $[T]_E$, all eigenvalues, the eigenspace of the maximal eigenvalue. Determine whether $T$ is diagonalizable. 
\end{example}

\begin{solution}
	\begin{align*}
		[T]_E &= 
			\begin{pmatrix}
				[T(e_1)]_E & [T(e_2)]_E & [T(e_3)]_E & [T(e_4)]_E\\
			\end{pmatrix}\\
		&= 
			\begin{pmatrix}
				\left[
					\begin{pmatrix}
						0 & 1\\
						0 & 1\\
					\end{pmatrix}
				\right]_E
				&
				\left[
					\begin{pmatrix}
						0 & -2\\
						0 & 0\\
					\end{pmatrix}
				\right]_E
				&
				\left[
					\begin{pmatrix}
						0 & 0\\
						1 & 0\\
					\end{pmatrix}
				\right]_E
				&
				\left[
					\begin{pmatrix}
						1 & 1\\
						0 & 0\\
					\end{pmatrix}
				\right]_E
			\end{pmatrix}
	\end{align*}
\end{solution}

\end{document}
