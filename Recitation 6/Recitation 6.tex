\documentclass[fleqn, a4paper, 12pt]{article}
\setcounter{secnumdepth}{4}
\usepackage{amsmath, amssymb, amsthm}
\usepackage{mathtools}
\usepackage{datetime}
\usepackage{ulem}
\usepackage{enumerate}

\newcommand\numberthis{\addtocounter{equation}{1}\tag{\theequation}}

\newcommand{\rank}{\mathrm{rk\,}}

\newenvironment{amatrix}[1]{%
	\left(\begin{array}{@{}*{#1}{c}|c@{}}
}{%
	\end{array}\right)
}

%\setlength{\mathindent}{0pt}

\theoremstyle{definition}
\newtheorem{example}{Example}
\newtheorem{definition}{Definition}

\theoremstyle{theorem}
\newtheorem{theorem}{Theorem}

\newenvironment{solution}
{\begin{proof}[Solution]\let\qed\relax}
	{\end{proof}}

\DeclareMathOperator{\vspan}{\mathrm{span}} %declares span operator for matrices

%opening
\title{Recitation 6}
\author{}
\date{\formatdate{3}{12}{2014}}

\begin{document}

\maketitle
%\setlength{\mathindent}{0pt}

\tableofcontents

\newpage
\section{Spanning Sets}

\begin{example}
	Find a set $k \in \mathbb{R}^4$, s.t. 
	\begin{align*}
		\vspan(k) &= \{x \in \mathbb{R}^4 | A x = 0\} & ; \quad A = 
		\begin{pmatrix}
			1 & 1 & -2 & 1\\
			1 & -2 & 1 & 2\\
			0 & 3 & -3 & -1\\
		\end{pmatrix}
	\end{align*}
\end{example}

\begin{solution}
	\begin{align*}
		\begin{pmatrix}
			1 & 1 & -2 & 1\\
			1 & -2 & 1 & 2\\
			0 & 3 & -3 & -1\\
		\end{pmatrix}
		\to 
		\begin{pmatrix}
			1 & 1 & -2 & 1\\
			- & -3 & 3 & 1\\
			0 & 0 & 0 & 0\\
		\end{pmatrix}
	\end{align*}
	Therefore,
	\begin{align*}
	x + y - 2z + w &= 0\\
	-3y + 3z + w &= 0\\
	\intertext{Solving,}
	y &= z + \dfrac{w}{3}\\
	x &= z - \dfrac{4}{3} w
	\end{align*}
	Therefore,
	\begin{align*}
		\vspan(k) &= 
		\left\lbrace 
			\begin{pmatrix}
				x\\
				y\\
				z\\
				w\\
			\end{pmatrix} 
			\vert
			z, w \in \mathbb{R}, x = z - \dfrac{4}{3} w, y = z + \dfrac{w}{3}
		\right\rbrace\\
		&= 
		\left\lbrace 
			\begin{pmatrix}
				z - \dfrac{4}{3} w\\
				y = z + \dfrac{w}{3}\\
				z\\
				w\\
			\end{pmatrix} 
			\vert
			z, w \in \mathbb{R}, x = z - \dfrac{4}{3} w, y = z + \dfrac{w}{3}
		\right\rbrace\\
		&= \vspan 
		\left\lbrace 
			\begin{pmatrix}
				1\\
				1\\
				1\\
				0\\
			\end{pmatrix}
			,
			\begin{pmatrix}
				-4\\
				1\\
				0\\
				3\\
			\end{pmatrix}
		\right\rbrace 
	\end{align*}
\end{solution}

\begin{example}
	Given 
	\begin{equation*}
		k = \{x^2 + 1, 2x, x_2 - 1\} \subseteq \mathbb{R}_2 [x]
	\end{equation*}
	prove 
	\begin{equation*}
		\vspan (k) = \mathbb{R}_2 [x]
	\end{equation*}
\end{example}

\begin{solution}
	Proving
	\begin{equation*}
		\vspan (k) = \mathbb{R}_2 [x]
	\end{equation*}
	is equivalent to proving
	\begin{align*}
		\vspan (k) &\subseteq \mathbb{R}_2 [x]\\
		& \&\\
		\mathbb{R}_2 [x] &\subseteq \vspan (k)
	\end{align*}
	Let
	\begin{equation*}
		p(x) = ax^2 + bx + c
	\end{equation*}
	Let
	\begin{align*}
		ax^2 + bx + c &= \alpha (x^2 + 1) + \beta (2x) + \gamma (x^2 - 1)\\
		&= (\alpha + \gamma) x^2 + (2 \beta) x + (\alpha - \gamma)
	\end{align*}
	This system has a solution. Therefore there exists such a linear combination.\\
	Therefore, 
	\begin{equation*}
		\mathbb{R}_2 [x] \subseteq \vspan (k)
	\end{equation*}
	It is obvious that
	\begin{equation*}
		\vspan (k) \subseteq \mathbb{R}_2 [x]
	\end{equation*}
	Therefore,
	\begin{equation*}
		\vspan (k)  = \mathbb{R}_2 [x]
	\end{equation*}
\end{solution}

\section{Subspaces}

\begin{example}
	Given subspaces
	\begin{align*}
		W_1 &= \vspan \{(1, 1, 0, 1), (1, -1, 0, 1)\}\\
		W_2 &= \vspan \{(0, 1, 1, -1), (0,1, -1, 1)\}
	\end{align*}
	Calculate $W_1 \cap W_2$.
\end{example}

\begin{solution}
	Let $u \in W_1 \cap W_2$.\\
	Therefore, $u \in W_1$ and $u \in W_2$.\\
	Therefore,
	\begin{align*}
		u = \alpha v_1 + \beta v_2 &= \gamma v_3 + \delta v_4\\
		\therefore \alpha v_1 + \beta v_2 - \gamma v_3 - \delta v_4 &= 0
	\end{align*}
	Therefore, the matrix of $v_1, v_2, -v_3, -v_4$ is
	\begin{equation*}
		\begin{pmatrix}
			v_1 & v_2 & -v_3 & -v_4\\
		\end{pmatrix}
		=
		\begin{pmatrix}
			1 & 1 & 0 & 0\\
			1 & -1 & -1 & -1\\
			0 & 0 & -1 & 1\\
			1 & 1 & 1 & -1\\
		\end{pmatrix}
	\end{equation*}
	\begin{equation*}
		\begin{pmatrix}
			1 & 1 & 0 & 0\\
			1 & -1 & -1 & -1\\
			0 & 0 & -1 & 1\\
			1 & 1 & 1 & -1\\
		\end{pmatrix}
		\to
		\begin{pmatrix}
			1 & 1 & 0 & 0\\
			0 & 2 & 1 & 1\\
			0 & 0 & -1 & 1\\
			0 & 0 & 0 & 0\\
		\end{pmatrix}
	\end{equation*}
	Therefore, $\delta$ is free.\\
	Therefore,
	\begin{align*}
		\alpha &= \delta\\
		\beta &= -\delta\\
		\gamma &= -\delta
	\end{align*}
	Therefore,
	\begin{align*}
		u &= \alpha v_1 + \beta v_2\\
		&= \delta (v_1 - v_2)\\
		\therefore W_1 \cap W_2 &= \{\delta (v_1 - v_2) | \delta \in \mathbb{R}\}\\
		&= \vspan \{0, 1, 0, 0\}
	\end{align*}
\end{solution}

\section{Coordinates}

\begin{definition}
	Let $V$ be a finitely generated vector space and $B = \{v_1, \dots, v_n\}$ be a basis of $V$.\\
	Each $u \in V$ can be uniquely represented as
	\begin{equation*}
		u = \alpha_1 v_1 + \dots + \alpha_n v_n
	\end{equation*}
	The scalars $\alpha_1, \dots, \alpha_n$ are called the coordinates of $u$ according to the basis $B$.\\
	The vector 
	\begin{equation*}
		[u]_B = 
		\begin{pmatrix}
			\alpha_1\\
			\vdots\\
			\alpha_n\\
		\end{pmatrix}
	\end{equation*}
	is called the coordinate vector of $u$ according to the basis $B$.
\end{definition}

\begin{example}
	Find a basis and dimension to the subspace
	\begin{equation*}
		W = \vspan \{1 - x, x + x^2, 1 + x^2\} \subset \mathbb{R}_2 [x]
	\end{equation*}
\end{example}

\begin{solution}
	Let 
	\begin{align*}
		p_1 (x) &= 1 - x\\
		p_2 (x) &= x + x^2\\
		p_3 (x) &= 1 + x^2
	\end{align*}
	The standard basis of $\mathbb{R}_2 [x]$ is $E = \{1, x, x^2\}$.\\
	Therefore,
	\begin{align*}
		[p_1]_E
		&=
		\begin{pmatrix}
			1\\
			-1\\
			0\\
		\end{pmatrix}\\
		[p_2]_E
		&=
		\begin{pmatrix}
			0\\
			1\\
			1\\
		\end{pmatrix}\\
		[p_3]_E
		&=
		\begin{pmatrix}
			1\\
			0\\
			1\\
		\end{pmatrix}\\
	\end{align*}
	Let 
	\begin{equation*}
		\tilde{W} = \vspan 
		\left\lbrace 
			\begin{pmatrix}
				1\\
				-1\\
				0\\
			\end{pmatrix}
			,
			\begin{pmatrix}
				0\\
				1\\
				1\\
			\end{pmatrix}
			,
			\begin{pmatrix}
				1\\
				0\\
				1\\
			\end{pmatrix}
		\right\rbrace 
		= \mathrm{R} 
		\begin{pmatrix}
			1 & -1 & 0\\
			0 & 1 & 1\\
			1 & 0 & 1\\
		\end{pmatrix}
		= \vspan 
		\left\lbrace 
			\begin{pmatrix}
				1\\
				0\\
				1\\
			\end{pmatrix}
			,
			\begin{pmatrix}
				0\\
				1\\
				1\\
			\end{pmatrix}
		\right\rbrace 
	\end{equation*}
	Therefore,
	\begin{equation*}
		W = \vspan\{1 + x^2, x + x^2\}
	\end{equation*}
\end{solution}

\section{Transformation Matrices}

\begin{definition}
	Let $B = \{b_1, \dots, b_n\}$ and $C= \{c_1, \dots, c_n\}$ be two bases of $V$. The transformation matrix from the base $C$ to the base $B$ is
	\begin{equation*}
		[B]_C = 
		\begin{pmatrix}
			[b_1]_C & \dots & [b_n]_C\\
		\end{pmatrix}
	\end{equation*}
\end{definition}

\subsection{Properties}

\begin{align*}
	[V]_C &= [B]_C [V]_B\\
	[C]_B &= [B]^{-1}_C\\
	[D]_C &= [B]_C [D]_B
\end{align*}

\end{document}
