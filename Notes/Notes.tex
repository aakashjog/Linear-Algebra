%change section to theorem environments
%correct references
%correct overfull hboxes

\date{}
\documentclass[fleqn, a4paper, 12pt]{article}
\usepackage{amsmath, amssymb, amsthm, thmtools, amsfonts}
\usepackage{datetime}
\usepackage{hyperref}
\usepackage{ulem}
\usepackage{tikz}
\usepackage{enumerate, enumitem}
%\usepackage{background}
\setcounter{secnumdepth}{4}

\newcommand\numberthis{\addtocounter{equation}{1}\tag{\theequation}}

\DeclareMathOperator{\vspan}{\mathrm{span}} %declares span operator for matrices

\newenvironment{amatrix}[1]{%declares augmented matrix environment
	\left(\begin{array}{@{}*{#1}{c}|c@{}}
	}{%
\end{array}\right)
} 

\theoremstyle{definition}
\newtheorem{example}{Example} %defines example environment
\newtheorem{definition}{Definition} %defines definition environment

\theoremstyle{theorem}
\newtheorem{theorem}{Theorem} %defines theorem environment
\newtheorem{corollary}{Corollary}

\theoremstyle{remark}
\newtheorem{remark}{Remark}
\newtheorem{case}{Case}

\newcommand{\suchthat}{\mathrm{\,s.t.\,}}

\newcommand{\R}{\mathrm{R}}

\newcommand{\C}{\mathrm{C}}

\newcommand{\rr}{\mathrm{rr}}

\newcommand{\im}{\mathrm{im}\,}

\newenvironment{solution} %declares solution environment and removes qed at end
	{\begin{proof}[Solution]\let\qed\relax}
	{\end{proof}}

\makeatletter
\@addtoreset{section}{part} %resets section numbers in new part
\makeatother

\makeatletter
\@addtoreset{theorem}{part} %resets theorem numbers in new part
\makeatother

\makeatletter
\@addtoreset{corollary}{theorem} %resets corollary numbers after a theorem
\makeatother

\numberwithin{corollary}{theorem}

\numberwithin{equation}{theorem}

\makeatletter %changes spacing between rows of matrices to fit fractions
\newif\ifcenter@asb@\center@asb@false
\def\center@arstrutbox{%
	\setbox\@arstrutbox\hbox{$\vcenter{\box\@arstrutbox}$}%
}
\newcommand*{\CenteredArraystretch}[1]{%
	\ifcenter@asb@\else
	\pretocmd{\@mkpream}{\center@arstrutbox}{}{}%
	\center@asb@true
	\fi
	\renewcommand{\arraystretch}{#1}%
}
\makeatother

\title{Linear Algebra}
\author{Aakash Jog}

\begin{document}

\maketitle

\tableofcontents

\newpage
\part{General Information}

\section{Contact Information}

\textbf{Prof. Boris Kunyavskii}\\
kunyav@gmail.com\\

\section{Grades}

Final Exam: 80\%\\
Midterm Exam: 10\%\\
Homework: 10\%\\
Passing Criteria: 60\%

\newpage
\part{Fields}
\section{Definition}

\begin{definition}[Field]
	The set $\mathbb{F}$ is a field if there are operations +, $\centerdot$ satisfying the following properties:
	
	\begin{enumerate}[label=(A\arabic*)]
		\item	$\forall a, b \in \mathbb{F}; a + b = b + a $
		\item	$\forall a, b \in \mathbb{F};  (a+b)+c = a+(b+c)$
		\item	There is an element $0 \in \mathbb{F} $ s.t. $a+0 = 0+a = a$
		\item	$\forall a \in F, \exists \, b \in \mathbb{F}$ s.t. $a + b = 0$
	\end{enumerate}
	
	\begin{enumerate}[label=(M\arabic*)]
		\item	$\forall a, b \in \mathbb{F}, a \cdot b = b \cdot a$
		\item	$\forall a, b \in \mathbb{F}, (a \cdot b) \cdot c = a \cdot (b \cdot c)$
		\item	There is an element $1 \in \mathbb{F}$ s.t. $a \cdot 1 = 1 \cdot a = a (1 \neq 0)$
		\item	$\forall a \in \mathbb{F}, (a \neq 0), \exists \, b \in \mathbb{F}$ s.t. $a \cdot b = 1$
	\end{enumerate}
	
	\begin{enumerate}[label=(AM)]
		\item	$(a + b) \cdot c = (a \cdot c) + (b \cdot c)$
	\end{enumerate}
	
	If $\mathbb{F}$ is a field, one can define subtraction and division as follows.
	\begin{align*}
	a - b &\dot{=} a + (-b)\\
	\dfrac{a}{b} &\dot{=} a \cdot \dfrac{1}{b}
	\end{align*}
\end{definition}

\subsection{Examples of Fields}
\begin{enumerate}
	\item	$\mathbb{R}$
	\item	$\mathbb{C}$
	\item	$\mathbb{F}_p$
\end{enumerate}

\subsection{Examples of Non-fields (Rings)}
\begin{enumerate}
	\item	$\mathbb{Z}$, as M4 is not satisfied.
\end{enumerate}

If we define $\mathbb{F}_2 = {0,1}; 0 + 0 = 0; 0 + 1 = 1 + 0 = 1$; then, necessarily, $1 + 1 = 0$, otherwise, $1$ will have no additive inverse.\\

\section{Examples}

\begin{example}
	Let $p$ be a prime number.\\
	$\mathbb{F}_p$ is defined as follows.
	\begin{align*}
	\forall m \in \mathbb{Z}, m &= a \centerdot p + \overline{m}
	\end{align*}
	
	The operations $+$ and $\centerdot$ are defined as
	\begin{align*}
	\overline{a} + \overline{b} = \overline{(a + b)}\\
	\overline{a} \centerdot \overline{b} = \overline{(a \centerdot b)}
	\end{align*}
	\begin{enumerate}
		\item	$\mathbb{F}_p$ is a field.
		\item	If $\mathbb{F}$ is a set of $q$ elements, we can define on $\mathbb{F}$ a structure of a field iff $q = p^t$, where $p$ is prime, $t \geq 1$.
	\end{enumerate}

\end{example}

\begin{example}
	For a field of 4 elements \{0, 1 $\alpha$, $\beta$\}, the addition and multiplication tables are as follows.\\
	
	\begin{tabular}{|c|c|c|c|c|}
		
		\hline
		+ & 0 & 1 & $\alpha$ & $\beta$ \\
		\hline
		0 & 0 & 1 & $\alpha$ & $\beta$ \\
		\hline
		1 & 1 & 0 & $\beta$ & $\alpha$ \\
		\hline
		$\alpha$ & $\alpha$ & $\beta$ & 0 & 1\\
		\hline
		$\beta$ & $\beta$ & $\alpha$ & 0 & 1 \\
		\hline
		
	\end{tabular}

\end{example}
\newpage
\part{Matrices}

\section{Definition}
\begin{definition}[Matrix]
	Let $\mathbb{F}$ be a field, $m, n \geq 1$.\\
	Then, $A(m \times n)$ is a table consisting of $m$ rows and $n$ columns, filled by elements of $\mathbb{F}$.\\
	
	\begin{equation*}
	A =
	\begin{pmatrix}
		a_{11} & a_{12} & \dots & a_{1n}\\	
		a_{21} & a_{22} & \dots & a_{2n}\\
		\vdots & \vdots & & \vdots\\
		a_{m1} & a_{m2} & \dots & a_{mn}\\
	\end{pmatrix}
\end{equation*}

\end{definition}
\section{Addition of Matrices}

\begin{definition}[Addition of matrices]\label{addition of matrices}
	Let $A$, $B$ be $m \times n$ matrices over $\mathbb{F}$.\\
	Then, $C = A + B$ is defined as follows.
	
	\begin{equation*}
	c_{ij} = a_{ij} + b{ij}\\
	\end{equation*}
\end{definition}

\subsubsection{Properties}

\begin{enumerate}
	\item	$A + B = B + A, \forall A, B$ s.t. the sum is defined\\
	\item	$(A + B) + C = A + (B + C), \forall A, B, C$ s.t. the sums are defined\\
	\item	There is a matrix $\mathbb{O}$, s.t. $A + \mathbb{O} = \mathbb{O} + A = A$
	\item	For any $A, \exists B$ s.t. $B = -A$
\end{enumerate}

\section{Multiplication of a matrix by a scalar}

\begin{definition}[Multiplication of a matrix by a scalar]\label{multiplication of a matrix by a scalar}
	Let $A$ be a $m \times n$ matrix over $\mathbb{F}$. Let $\alpha \in \mathbb{F}$ be a scalar.
	Then, $C = \alpha A$ is defined as follows.
	\begin{equation*}
	c_{ij} = \alpha a_{ij}
	\end{equation*}
\end{definition}

\section{Multiplication of matrices}

\begin{definition}[Multiplication of matrices]
	Let $A$ be a $m \times n$ matrix over $\mathbb{F}$.\\
	Let $B$ be a $n \times p$ matrix over $\mathbb{F}$.\\
	Then, $C = AB$ is defined as follows.
	\begin{equation*}
		c_{ik} = \sum_{j = 1}^{n} a_{ij} b_{jk}
	\end{equation*}
\end{definition}

\begin{example}
	For matrices $A,B$, of same size, is $AB = BA$?
	\begin{solution}
		$
		A = 
		\begin{pmatrix}
		0 & 1\\
		0 & 0\\
		\end{pmatrix}
		,
		B = 
		\begin{pmatrix}
		1 & 0\\
		0 & 0\\
		\end{pmatrix}
		\\
		\therefore AB = 
		\begin{pmatrix}
		0 & 0\\
		0 & 0\\
		\end{pmatrix}
		,
		BA = 
		\begin{pmatrix}
		0 & 1\\
		0 & 0\\
		\end{pmatrix}
		\\
		\therefore AB \neq BA\\
		$
	\end{solution}
\end{example}

\begin{remark}
	$A \neq \mathbb{O}, B \neq \mathbb{O}$, but $AB = \mathbb{O}$.
\end{remark}

\section{Zero Divisor}

\begin{definition}[Zero divisor]
	We say that a square matrix $A \neq \mathbb{O}$ is a \emph{zero divisor} if either there is a square matrix $B$ s.t. $AB = \mathbb{O}$, or there is a square matrix C, s.t. $CA = \mathbb{O}$.
\end{definition}

\begin{remark}
	$\mathbb{O}B = C\mathbb{O} = \mathbb{O}$.
\end{remark}

\begin{remark}
	$AC = BC \nRightarrow A = B$. In general, we cannot cancel matrices on either side of an equation.
	\begin{equation*}
		A = 
		\begin{pmatrix}
			0 & 1\\
			0 & 0\\
		\end{pmatrix}
		,
		B = 
		\begin{pmatrix}
			1 & 0\\
			0 & 0\\
		\end{pmatrix}
		,
		C = \mathbb{O}
	\end{equation*}
	\begin{equation*}
		\therefore AB = CB = \mathbb{O} \,\&\, B \neq \mathbb{O}
	\end{equation*}
	But, we cannot cancel $B$, as $A \neq C$.
\end{remark}

\newpage
\section{Theorem ('Good properties of matrix multiplication')}

\begin{theorem}
	\begin{align}
	(AB)C &= A(BC) \\
	A(B + C) &= AB + AC\\
	(A + B)C &= AC + BC\\
	(\alpha A) &= \alpha (AB)
	\end{align}
\end{theorem}

\begin{proof}
	Denote $AB = D, BC = G, (AB)C = F, A(BC) = H$\\
	We need to prove $F = H$\\
	Let the dimensions of the matrices be as follows.\\
	$A_{m \times n}, B_{n \times p}, C_{p \times q}$\\
	$\therefore F_{m \times q}, H_{m \times q}$\\
	\vspace{1 cm}
	\begin{align*}
		d_{ik} &= \sum_{j} a_{ij} b_{jk} \\
		\therefore g_{jl}  &= \sum_{k} b_{jk} b{kl} \\
		f_{il} &= \sum_{k} d_{ik} c{kl} = \sum_{k} (\sum_{j} a_{ij} b{jk}) c_{kl} = \sum_{k} \sum_{j} a_{ij} b_{jk} c_{kl}\\
		h_{il} &= \sum_{j} a_{ij} g{jl} = \sum_{j} a_{ij} (\sum_{k} b{jk} c_{kl}) = \sum_{k} \sum_{j} a_{ij} b_{jk} c_{kl}\\
		f_{il} &= h_{il}\\
		F &= H
	\end{align*}
\end{proof}

\newpage
\section{Square Matrices}

Let $A$ be a square matrix of size $n \times n, n \geq 1$
\subsection{Diagonal Matrices}

\begin{definition}[Diagonal matrix]
	We say that A is a \emph{diagonal matrix} if $a_{ij} = 0$, whenever $i \neq j$.
\end{definition}

\begin{theorem}
	Let $A$ and $B$ be diagonal $n \times n$ matrices.\\
	\begin{equation*}
		a_{rr} = \alpha_r, b_{rr} = \beta_r
	\end{equation*}
	Then, $AB = BA = C, C $ is a diagonal matrix with $c_{rr} = a_{rr} b_{rr}$.
\end{theorem}
\subsubsection{Proof}

$
a_{ij} = 
\begin{cases}
0, i \neq j\\
\alpha_i, i = j\\
\end{cases}\\
$
$
b_{ij} = 
\begin{cases}
0, i \neq j\\
\beta_i, i = j\\
\end{cases}\\
$
$
c_{ik} = \sum_{j = 1}^{n} a_{ij} b{jk} = a_{ii} b_{ik} = \alpha_i b_{ik} = 
\begin{cases}
0, i \neq k\\
\alpha_i \beta_i, i = k\\
\end{cases}\\
$
Similarly for $BA$.

\subsection{Upper-triangular Matrices}

We say that A is an \emph{upper-triangular matrix} if $a_{ij} = 0$, whenever $i > j$.

\subsection{Lower-triangular Matrices}

We say that A is a \emph{lower-triangular matrix} if $a_{ij} = 0$, whenever $i < j$.

\subsubsection*{Remark}
Diagonal matrices are upper-triangular and lower-triangular. Conversely, if a matrix is both upper-triangular and lower-triangular, it is a diagonal matrix.

\subsection{Theorem}

If  $A$ and $B$ are both upper-triangular, then $AB$ and $BA$ are upper-triangular too.\\

\subsubsection{Proof}

Denote $C = AB$.\\
\[\therefore	c_{ik} = \sum_{j = 1}^{n} a_{ij} b_{jk}\]\\
Suppose $i > k$, then, either $i>j$ or $j>k$. So, in each case, atleast one of $a_{ij}$ or $b_{jk}$ is 0.

\subsection{Identity Matrix}

Let $n \geq 1$. We call $I_n$ the $n \times n$ \emph{identity matrix}.\\
$
I_n = 
\begin{pmatrix}
1 & 0 & 0 & 0 \\
0 & 1 & 0 & 0 \\
0 & 0 & \ddots & 0 \\
0 & 0 & 0 & 1\\
\end{pmatrix}
$

\subsection{Theorem}

Let $I_n$ be the identity $n \times n$ matrix. Then, for any $n \times n$ matrix $B$, we have
\[I_n B = B I_n = B\]

\subsubsection{Proof}

$I_n = (e_{ij}); e_{ij} = 
\begin{cases}
0, i \neq j\\
1, i = j\\
\end{cases}
$\\
Denote $C = I_n B$. We have
\begin{align*}
c_{ik} &= \sum_{j =1}^{n} e_{ij} b_{jk} = e_{ii} b_{ik} = 1 \cdot b_{ik} = b_{ik}\\
\therefore C &= B \Rightarrow I_n B = B
\end{align*}
Similarly for $B I_n = B$.

\subsection{Inverse of Matrix}

Let $A$ be an $n \times n$ matrix. We say that $A$ is \emph{invertible} if there exist $B, C$, s.t.
$AB = I_n$ and $CA = I_n$\\

\subsubsection*{Remark}
$A = \mathbb{O}$ is not invertible because $\mathbb{O} B = C \mathbb{O} = \mathbb{O} \neq I_n$\\

\subsubsection*{Remark}
There are \emph{non-zero} matrices which are not invertible.\\
Let 
$
A = 
\begin{pmatrix}
0 & 1 \\
0 & 0 \\
\end{pmatrix}
$\\
If possible, let there be $C$ s.t. $CA = I_2$.\\
Let
$
B = 
\begin{pmatrix}
1 & 0 \\
0 & 0 \\
\end{pmatrix}
$\\
We have $CA = I$. \\
$
\therefore	(CA)B = IB \\
\therefore	C(AB) = B \\
\therefore	C \mathbb{O} = B \\
\therefore	\mathbb{O} = B \\
$
But, $B \neq 0$. Therefore, $C$ does not exist.\\

\subsubsection{If $AB = I_n$ and $CA = I_n$, then $B = C$}

\begin{align*}
C &= CI \\
&= C(AB) \\
&= (CA)B \\
&= IB \\
&= B \\
\end{align*}

\subsubsection{Inverse of a Matrix}

If $A$ is invertible, i.e. if there exists $B$, s.t. $AB = BA = I$, then, $B$ is called the \emph{inverse} of $A$, and is denoted by $A^{-1}$.

\subsubsection{If $AB = I$, then $BA = I$.}

\subsubsection{If $A$ is invertible, then $A$ \emph{cannot} be a zero divisor.}

If possible, let $A$ be a zero divisor.\\
Therefore, either $AB = \mathbb{O}$, for some $B \neq \mathbb{O}$; or $CA = \mathbb{O}$, for some $C \neq \mathbb{O}$\\

\paragraph*{Case I: $AB = \mathbb{O}$}

\begin{align*}
	AB = \mathbb{O}\\
	\therefore A^{-1}(AB) = A^{-1}\mathbb{O}\\
	\therefore (A^{-1}A)B = \mathbb{O}\\
	\therefore IB = \mathbb{O}\\
	\therefore B = \mathbb{O}
\end{align*}

This contradicts the assumption $B \neq \mathbb{O}$

\paragraph*{Case II: $CA = \mathbb{O}$}

\begin{align*}
CA = \mathbb{O}\\
\therefore (CA)A^{-1} = \mathbb{O}A^{-1}\\
\therefore C(A^{-1}A) = \mathbb{O}\\
\therefore CI = \mathbb{O}\\
\therefore C = \mathbb{O}
\end{align*}

This contradicts the assumption $C \neq \mathbb{O}$

\subsubsection{If $A$ and $B$ are invertible, then $A+B$ may or may not be invertible.}

If $A = B$, then $A + B = 2A$ is invertible.\\
If $A = -B$, then $A + B = \mathbb{O}$ is not invertible.

\subsubsection{If $A$ and $B$ are invertible, then $AB$ must be invertible.}

\begin{align*}
	(AB) (B^{-1} A^{-1}) &= A(B B^{-1}) A^{-1}\\
	&= A I A^{-1}\\
	&= A A^{-1}\\
	&= I\\
	\text{Similarly, }& (B^{-1} A^{-1})(AB) = I\\
	\therefore (AB)^{-1} &= B^{-1} A^{-1}
\end{align*}

\section{Transpose of a Matrix}

Let $A$ be a $m \times n$ matrix, $A = (a_{ij})_{1 \leq i \leq m; 1\leq j \leq n}$\\

$B = A^{t}$ is defined as follows.
\begin{equation*}
	b_{ji} = a_{ij}
\end{equation*}

\subsection{Properties of $A^{t}$}

\begin{enumerate}
	\item	$(A + B)^t = A^t + B^t$
	\item	$(\alpha A)^t = \alpha A^t$
	\item	$(AB)^t = B^t A^t$
	\item	If $A$ is invertible, then, $A^t$ must be invertible, and $(A^t)^{-1} = (A^{-1})^t$
\end{enumerate}

\section{Adjoint Matrix}

\begin{equation*}
	A^* \dot{=} \overline{A}^t
\end{equation*}
For example,
\begin{align*}
	A &= 
	\begin{pmatrix}
	1 & 1+i & 2-1 \\
	i & -5i & 3 \\
	\end{pmatrix}\\
	B &=
	\begin{pmatrix}
	1 & -i \\
	1-i & 5i \\
	2+i & 3 \\
	\end{pmatrix}
\end{align*}

\subsubsection{Properties of Adjoint Matrices}

\begin{enumerate}
	\item $(A + B)^* = A^* + B^* $
	\item $(\alpha A)^* = \overline{\alpha} A^*$
	\item $(AB)^* = B^* A^*$
	\item If $A$ is invertible, then $A^*$ is invertible, and $(A^*)^{-1} = (A^{-1})^*$
\end{enumerate}

\section{Row Operations on Matrices}

\subsection{Elementary Row Operations}

Let $A$ be a $m \times n$ matrix with rows $a_1, \dots a_m$. We define 3 types of \emph{elementary row operations}.

\begin{enumerate}[label=\Roman*]
	\item $a_i \leftrightarrow a_j$ (Switch of the $i^{\text{th}}$ and $j^{\text{th}}$ rows.) \label{elementary row operation 1}
	\item $a_i \rightarrow \alpha a_i (\alpha \neq 0)$ (Multiplication of a row by a non-zero scalar.) \label{elementary row operation 2}
	\item $a_i \rightarrow a_i + \alpha a_j (j \neq i)$ (Addition of a row multiplied by a scalar, and another row.) \label{elementary row operation 3}
\end{enumerate}
$E_\text{I}, E_\text{II}, E_\text{III}$ are matrices obtained from the identity matrix by applying elementary row operations \ref{elementary row operation 1}, \ref{elementary row operation 2}, \ref{elementary row operation 3}, respectively. These matrices are called elementary matrices.

\subsection{Theorems}

Let 
$
e_i = 
\begin{pmatrix}
0 & \dots & 0 & 1 & 0 & \dots & 0\\
\end{pmatrix}
$
be a $1 \times m$ matrix.\\
Let $A$ be any ${m \times n}$ matrix.\\
Then, $e_i A = $ the $i^{\text{th}}$ row of A.

\subsubsection{$E_\text{I} A = $ the matrix obtained from $A$ by an elementary row operation \ref{elementary row operation 1}}
\paragraph*{Proof\\}
Let $A$ be any $m \times n$ matrix.\\
\begin{align*}
	\therefore E_\text{I} A = 
	\begin{pmatrix}
		e_1 A\\
		\vdots\\
		e_j A\\
		\vdots\\
		e_i A\\
		\vdots\\
		e_m A\\
	\end{pmatrix}
\end{align*}

\subsubsection{$E_\text{II} A = $ the matrix obtained from $A$ by an elementary row operation \ref{elementary row operation 2}}

\paragraph*{Proof\\}
Let $A$ be any $m \times n$ matrix.\\
\begin{align*}
\therefore E_\text{I} A = 
\begin{pmatrix}
e_1 A\\
\vdots\\
\alpha e_i A\\
\vdots\\
e_m A\\
\end{pmatrix}
\end{align*}

\subsubsection{$E_\text{III} A = $ the matrix obtained from $A$ by an elementary row operation \ref{elementary row operation 3}}

\paragraph*{Proof\\}
Let $A$ be any $m \times n$ matrix.\\
\begin{align*}
\therefore E_\text{I} A &= 
\begin{pmatrix}
e_1 A\\
\vdots\\
a_{i1} + \alpha a_{j1} \dots + a_{in}+ \alpha a_{jn}\\
\vdots\\
e_j A\\
\vdots\\
e_m A\\
\end{pmatrix}\\
&=
\begin{pmatrix}
1^{\text{st}} \text{ row of }A\\
\vdots\\
i^{\text{th}} \text{ row of }A + \alpha (j^{\text{th}}) \text{ row of }A\\
\vdots\\
j^{\text{th}} \text{ row of }A\\
\vdots\\
m^{\text{th}} \text{ row of }A\\
\end{pmatrix}
\end{align*}

\subsubsection{All elementary matrices are invertible, moreover, the inverses of $E_{\text{I}}, E_{\text{II}}, E_{\text{III}}$ are also elementary matrices of the same type.}

\begin{align*}
	E_{\text{I}}^{-1} &=  E_{\text{I}}\\
	\Leftrightarrow E_{\text{I}}^2 &= I_m
\end{align*}
\begin{align*}
	E_{\text{I}}^2 &= E_{\text{I}} E_{\text{I}}\\
	&= 
	\begin{pmatrix}
		e_1 E_{\text{I}}\\
		\vdots\\
		e_j E_{\text{I}}\\
		\vdots\\
		e_i E_{\text{I}}\\
		\vdots\\
		e_m E_{\text{I}}\\
	\end{pmatrix}\\
	&= 
	\begin{pmatrix}
		1^{\text{st}} \text{ row of }A\\
		\vdots\\
		j^{\text{th}} \text{ row of }A\\
		\vdots\\
		i^{\text{th}} \text{ row of }A\\
		\vdots\\
		m^{\text{th}} \text{ row of }A\\
	\end{pmatrix}\\
	&=
	\begin{pmatrix}
		e_1\\
		\vdots\\
		e_j\\
		\vdots\\
		e_i\\
		\vdots\\
		e_m\\
	\end{pmatrix}
	= I_m
\end{align*}
Similarly for $E_{\text{II}}$, to get the inverse, $\alpha$ is replaced by $\dfrac{1}{\alpha}$\\
\begin{align*}
	E_{\text{II}} &=
	\begin{pmatrix}
		1 & 0 & 0 & \dots & 0\\
		0 & 1 & 0 & \dots & 0\\
		0 & 0 & \alpha & \dots & 0\\
		\vdots & \vdots & \vdots & \ddots & \vdots \\
		0 & 0 & 0 & \dots & 1\\
	\end{pmatrix}\\
	\therefore 
	E_{\text{II}}^{-1} &=
		\begin{pmatrix}
		1 & 0 & 0 & \dots & 0\\
		0 & 1 & 0 & \dots & 0\\
		0 & 0 & \dfrac{1}{\alpha} & \dots & 0\\
		\vdots & \vdots & \vdots & \ddots & \vdots \\
		0 & 0 & 0 & \dots & 1\\
	\end{pmatrix}
\end{align*}
Similarly for $E_{\text{III}}$, to get the inverse, $\alpha$ is replaced by $-\alpha$\\

\begin{align*}
	E_{\text{III}} &=
	\begin{pmatrix}
		1 & 0 & 0 & \dots & 0\\
		0 & 1 & \alpha & \dots & 0\\
0 & 0 & 1 & \dots & 0\\
\vdots & \vdots & \vdots & \ddots & \vdots \\
0 & 0 & 0 & \dots & 1\\
\end{pmatrix}\\
\therefore 
E_{\text{III}}^{-1} &=
\begin{pmatrix}
1 & 0 & 0 & \dots & 0\\
0 & 1 & -\alpha & \dots & 0\\
0 & 0 & 1 & \dots & 0\\
\vdots & \vdots & \vdots & \ddots & \vdots \\
0 & 0 & 0 & \dots & 1\\
\end{pmatrix}
\end{align*}

\subsection{Row-equivalent of a Matrix}

A matrix $A'$ is a \emph{row-equivalent} of $A$, if $A'$ is obtained for $A$, by a finite sequence of elementary row operations.

\section{Row Echelon Form of a Matrix}

\subsection{Definition}
Let $A$ be an $m \times n$ matrix.\\
Denote the $i^{\text{th}}$ row of $A$ by $a_i$.\\
The \emph{leading entry} of a non-zero row $a_i$ is its \emph{first non-zero entry}.\\
Denote the column where the leading entry occurs by $l_i$.
\begin{align*}
	a_{ij} = 0 &\text{ if } j < l(i)\\
	a_{ij} \neq 0 &\text{ if } j = l(i)\\
\end{align*}
We say that $A$ is in row echelon form(REF) if the following conditions hold.
\begin{enumerate}
	\item The non-zero rows are at the top of $A$. ($r = $ the number of non-zero rows)
	\item The leading entries go right as we go down, i.e. $l(1) < l_2 < \dots < l(r)$
	\item All leading entries equal $1$, i.e. if $j = l(i)$, then, $a_{ij} = 1$
	\item Any column which contains a leading entry must have all other entries equal to $0$, i.e. if $j = l(i)$, then, $a_{kj} = 0; \forall k \neq i$
\end{enumerate}

\subsection{Notation}

The REF of $A$ will be denoted by $A_R$. 

\section{Row Rank of a Matrix}

The number of non-zero rows in $A_R$ is called the row rank of $A$. It is denoted by $r$.
\begin{equation*}
	r \leq n
\end{equation*}

\section{Gauss Theorem}

Any $m \times n$ matrix $A$ can be brought to REF by a sequence of elementary row operations.

\subsection{Elimination Algorithm}

\begin{enumerate}[label=Step \arabic*]
	\item Find the first non-zero column $C_p$ of $A$. \label{First step}
	\item Denote by $a_{ip}$ the first non-zero entry of $C_p$.
	\item Switch the $1^{\text{st}}$ and $i^{\text{th}}$ rows.
	\item Multiply the $1^{\text{st}}$ row by $\dfrac{1}{a_{ip}}$.
	\item Using row operations of type III, make all other entries of the $p^{\text{th}}$ column zeros. \label{Last step}
	\item Ignoring the top row and $C_p$, repeat steps \ref{First step} to \ref{Last step}.
\end{enumerate}

\subsubsection{Example}
$
\begin{pmatrix}
	0 & 0 & 0 & -1 \\
	0 & -1 & 4 & 7 \\
	0 & -1 & 7 & 6 \\
\end{pmatrix}
\xrightarrow{R_1 \rightarrow R_2}
\begin{pmatrix}
	0 & -1 & 4 & 7 \\
	0 & 0 & 0 & -1 \\
	0 & -1 & 7 & 6 \\
\end{pmatrix}
\xrightarrow{R_1 \rightarrow -R_1}
\begin{pmatrix}
	0 & 1 & -4 & -7 \\
	0 & 0 & 0 & -1 \\
	0 & -1 & 7 & 6 \\
\end{pmatrix}
\xrightarrow{R_3 \rightarrow R_3 + R_1}
\begin{pmatrix}
	0 & 1 & -4 & -7 \\
	0 & 0 & 0 & -1 \\
	0 & 0 & 3 & -1 \\
\end{pmatrix}
\xrightarrow{R_2 \leftrightarrow R_3}
\begin{pmatrix}
	0 & 1 & -4 & -7 \\
	0 & 0 & 3 & -1 \\
	0 & 0 & 0 & -1 \\
\end{pmatrix}
\xrightarrow{R_2 \rightarrow \frac{R_2}{3}}
\CenteredArraystretch{2}
\begin{pmatrix}
	0 & 1 & -4 & -7 \\
	0 & 0 & 1 & -\dfrac{1}{3} \\
	0 & 0 & 0 & -1 \\
\end{pmatrix}
\xrightarrow{R_1 \rightarrow R_1 + 4 R_2}
\CenteredArraystretch{2}
\begin{pmatrix}
	0 & 1 & 0 & -\dfrac{25}{3} \\
	0 & 0 & 1 & -\dfrac{1}{3} \\
	0 & 0 & 0 & -1 \\
\end{pmatrix}
\xrightarrow{R_3 \rightarrow -R_3}
\CenteredArraystretch{2}
\begin{pmatrix}
	0 & 1 & 0 & -\dfrac{25}{3} \\
	0 & 0 & 1 & -\dfrac{1}{3} \\
	0 & 0 & 0 & 1 \\
\end{pmatrix}
\xrightarrow{R_1 \rightarrow R_1 + \frac{25}{3} R_3}
\CenteredArraystretch{2}
\begin{pmatrix}
	0 & 1 & 0 & 0 \\
	0 & 0 & 1 & -\dfrac{1}{3} \\
	0 & 0 & 0 & 1 \\
\end{pmatrix}
\xrightarrow{R_2 \rightarrow R_2 + \frac{1}{3} R_3}
\CenteredArraystretch{1}
\begin{pmatrix}
	0 & 1 & 0 & 0 \\
	0 & 0 & 1 & 0 \\
	0 & 0 & 0 & 1 \\
\end{pmatrix}
$

\subsection{Row Spaces of Matrices}

\begin{definition}[Row space of a matrix]
	Let $A$ be a $m \times n$ matrix over $\mathbb{F}$.
	$\R(A)$ is defined as 
	\begin{equation*}
	\R(A) = \vspan{v_1, \dots, v_m}
	\end{equation*}
	where $v_1, \dots, v_m$ are rows of $A$.\\
	$\R(A)$ a subspace of the vector space of all rows of length $n$, is called the row space of $A$.
\end{definition}

\begin{definition}[Row rank of a matrix]
	$\dim \R(A)$ is called the row-rank of $A$, and is denoted by $\rr(A)$.
\end{definition}

\begin{theorem}	
	Let $P$ be a $l \times m$ matrix. Then
	\begin{enumerate}
		\item $\R(P A) \subseteq \R(A)$
		\item If $P$ is an invertible $m \times m$ matrix, then $\R(P A) = \R(A)$
	\end{enumerate}
\end{theorem}

\begin{corollary}
	\begin{equation*}
	A' \stackrel{\R}{\sim} A \implies \R(A') = \R(A)
	\end{equation*}
\end{corollary}

\begin{theorem}
	If $A$ is in REF, and if $r$ is the number of non-zero rows in $A$, then
	\begin{equation*}
	\rr(A) = r
	\end{equation*}
\end{theorem}

\begin{corollary}
	The following are equivalent
	\begin{enumerate}
		\item $A \stackrel{R}{\sim} A'$
		\item There is an invertible matrix $P, \suchthat A' = P A$
		\item $\R(A) = \R(A')$
		\item $A$ and $A'$ have the same REF
	\end{enumerate}
\end{corollary}

\subsection{Column Equivalence}

\begin{definition}
	If $A$ is a $m \times n$ matrix, we can define elementary column operations, column equivalence ($A \stackrel{C}{\sim}$) and column echelon form (CEF), the column space of $A$ ($\C(A)$), and the column rank of $A$ ($\mathrm{cr}(A)$).
\end{definition}

\begin{theorem}
	\begin{equation*}
	\mathrm{cr}(A) = \rr(A) =  r
	\end{equation*}
\end{theorem}

\begin{proof}
	Let $r = \rr(A) = \dim \R(A)$.\\
	Choose $r$ rows of $A$ which form a basis of $\R(A)$, WLG, say ${v_1, \dots, v_r}$.\\
	Let 
	\begin{equation*}
	X_{r \times n} = 
	\begin{pmatrix}
	v_1\\
	\vdots\\
	v_r\\
	\end{pmatrix}
	\end{equation*}
	
	\begin{equation*}
	\vspan(X) = \R(A)
	\end{equation*}
	Hence, any row of $A$ can be expressed as a linear combination of $v_1, \dots, v_r$
	\begin{equation*}
	v_i = \sum_{j = 1}^{r} y_{ij} v_j
	\end{equation*}
	Let 
	\begin{equation*}
	Y_{m \times r} = (y_{ij})
	\end{equation*}
	Therefore,
	\begin{equation*}
	A = Y X
	\end{equation*}
	Considering each column of $A$ as a linear combination of columns of $Y$,
	\begin{gather*}
	\C(A) \subseteq \C(Y)\\
	\therefore \mathrm{cr}(A) \leq \mathrm{cr}(Y) \leq r = \rr(A)\\
	\therefore \mathrm{cr}(A) \leq \rr(A)
	\intertext{Similarly,}
	\rr(A) \leq \mathrm{cr}(A)
	\therefore \mathrm{cr}(A) = \rr(A)
	\end{gather*}
\end{proof}

\begin{corollary}
	The following are equivalent
	\begin{enumerate}
		\item $A \stackrel{C}{\sim} A'$
		\item There is an invertible matrix $Q, \suchthat A' = Q A$
		\item $\C(A) = \C(A')$
		\item $A$ and $A'$ have the same CEF
	\end{enumerate}
\end{corollary}

\newpage
\part{Linear Systems}

\section{Definition}

\begin{align*}
	a_{11} x_{1} + a_{12} x_{2} + \dots + a_{1n} x_{n} &= b_{1} \\
	a_{21} x_{1} + a_{22} x_{2} + \dots + a_{2n} x_{n} &= b_{2} \\
	\vdots\\
	a_{m1} x_{1} + a_{m2} x_{2} + \dots + a_{mn} x_{n} &= b_{m} \\
\end{align*}
Here, all $x_i$ are taken to be unknowns, and all $a_{ij}, b_i$ are given. \\
A \emph{solution} to such a system is a collection $d_1, \dots, d_n$, s.t. after replacing $x_i$ by $d_i$, we get equalities.\\
We assume that all $a_{ij}, b_i$ belond to $\mathbb{F}$, and we are looking for solutions $d_i \in \mathbb{F}$.\\
Given such a system, we define $A_{m \times n} = (a_{ij}), b_{m \times 1} = 
\begin{pmatrix}
	b_1\\
	\vdots\\
	b_m\\
\end{pmatrix} 
,
x_{n \times 1} = 
\begin{pmatrix}
	x_1\\
	\vdots\\
	x_n\\
\end{pmatrix}
$\\
Then, we can write the system as
\begin{equation*}
	A x = b
\end{equation*}
A solution to this system is 
$
d_n = 
\begin{pmatrix}
	d_1\\
	\vdots\\
	d_n\\
\end{pmatrix}
$
, s.t. $A d = b$

Let $D$ be the set of all 
$
d = 
\begin{pmatrix}
	d_1\\
	\vdots\\
	d_n\\
\end{pmatrix}\\
$
$D$ may be empty, infinite, or a singleton set.

\section{Equivalent Systems}

Two systems $A x = b$ and $A' x = b'$ are called \emph{equivalent}, if every solution of the first system is also a solution of the second system, and vice versa.

\section{Solution of a System of Equations}

We want to bring a given system
\begin{equation*}
	A x = b
\end{equation*}
to the form 
\begin{equation*}
	A_R x = b_R
\end{equation*}
using elementary row operations.\\
We denote the augmented or extended matrix of the system as follows.
\begin{equation*}
	\overline{A}_{m \times (n+1)} = (A_{m \times n} | b_{m \times 1})
\end{equation*}
Then apply Gaussian elimination method to $\overline{A}$, in order to get the matrix 
\begin{equation*}
	(A_R | b_R)
\end{equation*}
As $A_R$ is obtained from $A$ using elementary row operations,
\begin{equation*}
	A_R = E_n \dots E_2 E_1 A
\end{equation*}
where every $E_i$ is an elementary matrix.\\
Let $P = E_n \dots E_2 E_1$. $P$ is invertible, as it is a product of elementary matrices.
\begin{align*}
	A_R &= P A \\
	\therefore A_R d &= P A d\\
	&= P b \\
	&= b_R
\end{align*}
Conversely, let $d$ be a solution to 
\begin{align*}
	A_R d &= b_R \\
	\therefore P A d &= b_R \\
	\therefore P^{-1} (P A d) &= P^{-1} b_R \\
	\therefore A d &= b
\end{align*}
If we have a system $A x = b$, we may and will assume that $A$ is in REF, i.e. $A = A_R, b = b_R$.\\
Let $l(1), \dots l(r)$ denote the numbers of the columns containing leading entries.
Let $b = 
\begin{pmatrix}
	b_1\\
	\vdots\\
	b_r\\
	b_{r+1}\\
	\vdots\\
	b_m\\
\end{pmatrix}
$\\
Therefore, 
\begin{align*}
	1 \cdot x_{l(1)} + \dots &= b_1\\
	1 \cdot x_{l(2)} + \dots &= b_2\\
	& \vdots\\
	1 \cdot x_{l(r)} &= b_r\\
	0 &= b_{r+1}\\
	& \vdots\\
	0 &= b_m\\
\end{align*}

\section{Homogeneous Systems}

\subsection{Definition}

A system of the form 
\begin{equation*}
	A x = \mathbb{O}
\end{equation*}
is called a homogeneous system.

\subsubsection*{Remark}

Any homogeneous system is consistent and has a trivial solution $x = \mathbb{O}$

\subsection {Solutions of Homogeneous Systems}

If $r = $ number of non-zero rows, let $t = n - r = $ number of free variables. If $t > 0$ , denote the numbers of the columns that \emph{do not} contain leading entries by $z(1), \dots, z(t)$

\subsubsection{Example}

\begin{equation*}
	A = 
	\begin{pmatrix}
		0 & 1 & 2 & 0 & 0 & -3 \\
		0 & 0 & 0 & 1 & 0 & -1 \\
		0 & 0 & 0 & 0 & 1 & 7 \\
		0 & 0 & 0 & 0 & 0 & 0 \\
	\end{pmatrix}
\end{equation*}
Therefore,
\begin{align*}
	m &= 4 \\
	n &= 6 \\
	r &= 3 \\
	t &= 3 \\
	l(1) &= 2 \\
	l(2) &= 4 \\
	l(3) &= 5 \\
	z(1) &= 1 \\
	z(2) &= 3 \\
	z(3) &= 6
\end{align*}
Therefore, 
\begin{align*}
	x_2 + 2x_3 - 3x_6 &= 0\\
	x_4 - x_6 &= 0 \\
	x_5 + 7x_6 &= 0
\end{align*}
Therefore, 
\begin{align*}
	x_2 &= -2x_3 + 3x_6 \\
	x_4 &= x _6 \\
	x_5 &= -7 x_6
\end{align*}
\begin{equation*}
	\begin{pmatrix}
		x_2 \\
		x_4 \\
		x_5 \\
	\end{pmatrix}
	=
	C_{3 \times 3} 
	\begin{pmatrix}
		x_1 \\
		x_3 \\
		x_6 \\
	\end{pmatrix}
\end{equation*}
where $C_{3 \times 3} = 
\begin{pmatrix}
	0 & -2 & 3 \\
	0 & 0 & 1 \\
	0 & 0 & -7 \\
\end{pmatrix}
$\\
The free variables $x_1, x_3, x_6$ can be considered as parameters, $x_1 = \gamma_1 , x_2 = \gamma_2 , x_3 = \gamma_3$.\\
Therefore, 
\begin{align*}
x_2 &= -2\gamma_3 + 3\gamma_6 \\
x_4 &=  \gamma_6 \\
x_5 &= -7 \gamma_6
\end{align*}

\subsubsection{General Solution}

\paragraph{Case I: $t = 0$\\}
If $t = 0$, there are no free variables, and the system has a unique trivial solution.

\paragraph{Case II: $t > 0$}

\begin{equation*}
	\begin{pmatrix}
		x_{l(1)} \\
		x_{l(2)} \\
		\vdots \\
		x_{l(r)} \\
	\end{pmatrix}
	=
	C_{r \times t}
	\begin{pmatrix}
		x_{z(1)} \\
		x_{z(2)} \\
		\vdots \\
		x_{z(t)} \\
	\end{pmatrix}
\end{equation*}
$C$ is filled by coefficients of the equations obtained after shifting the terms containing all $z_i$ to the RHS.

\subsection{Properties}

\subsubsection{For a homogeneous system $A x = 0$, if $c$ and $d$ are solutions, then $c + d$ is also a solution.}

\begin{align*}
	A c &= \mathbb{O} \\
	A d &= \mathbb{O} \\
	\therefore A(c + d) &= A c + A d \\
	&= \mathbb{O} + \mathbb{O} \\
	&= \mathbb{O}
\end{align*}

\subsubsection{For a homogeneous system $A x = 0$, if $c$ is a solution and $\alpha \in \mathbb{F}$, then, $\alpha c$ is a solution too.}

\begin{align*}
	A c &= \mathbb{O} \\
	\therefore A(\alpha c) &= \alpha (A c) \\
	&= \alpha \mathbb{O} \\
	&= \mathbb{O} 
\end{align*}

\subsection{Fundamental Solutions}

We define $t$ fundamental solutions or basic solutions, $v_1, \dots v_t$.\\
We define $t$ columns, each of length $n$ as follows.\\
For the $i^{\text{th}}$ column $v_i$, we set
\begin{align*}
	x_{z(1)} &= 0 \\
	x_{z(i)} &= 1 \\
	\vdots \\
	x_{z(t)} &= 0 \\
\end{align*}
and for $x_{l(1)} , \dots , x_{l(r)}$,
\begin{equation*}
	\begin{pmatrix}
		x_{l(1)} \\
		\vdots \\
		x_{l(r)} \\
	\end{pmatrix}
	=
	C
	\begin{pmatrix}
	x_{z(1)} \\
	\vdots \\
	x_{z(t)} \\
	\end{pmatrix}
	= i^{\text{th}}\text{column of } C
\end{equation*}

\subsubsection{Theorem: Any solution $d$ of the system $A x = \mathbb{O}$ can be obtained from the basic solutions $v_1, \dots, v_t$ as a linear combination of the basic solutions, $d = \alpha_1 v_1 + \dots \alpha_t v_t$}

One can choose another collection $v'_1, \dots, v'_t$ s.t. any solution of $A x = \mathbb{O}$ can be obtained as a linear combination of $v'_1, \dots, v'_t$. In such a case, we get another form of the general solution.

\subsection{}

\begin{align*}
	r \leq \min{m, n} \\
\end{align*}

If $r = n$, i.e. $t = 0$, the system has a unique solution.\\
If $r < n$, i.e. $t > 0$, the system has more than one solutions. Its general solution can be expressed as in terms of $t$ parameters, where each free variable serves as a parameter, whose value can be any element of $\mathbb{F}$. \\
If $m < n$, then $r < n$. Therefore, the system has more than one solution. 

\section{Non-Homogeneous Systems}

\subsection{Definition}

Consider a system $A x = b ; b \neq \mathbb{O}$. The extended matrix is defined as
\begin{equation*}
	\tilde{A} = (A|b) = 
	\begin{pmatrix}
		a_{11} & \dots & a_{1n} & b_1 \\
		\vdots & & \vdots & \vdots \\
		a_{m1} & \dots & a_{mn} & b_m \\
	\end{pmatrix}
\end{equation*}

\subsection{Solutions of Non-Homogeneous Systems}

Let $\tilde{r}$ be the number of non-zero rows in the REF of $\tilde{A}$, i.e. $\tilde{A_R}$.

\subsubsection{Case I: $\tilde{r} = r$}

\begin{equation*}
	b'_{r+1} = \dots = b'_{m} = 0
\end{equation*}

\paragraph{Case a: $r = n$, i.e. $t=0$\\}

Therefore,
\begin{align*}
	x_1 &= b'_1 \\
	\dots \\
	x_r &= b'_r \\
\end{align*}
Hence, the system has a unique solution. 

\paragraph{Case b: $r < n$, i.e. $t > 0$\\}

Therefore,
\begin{align*}
	x_l(1) &= b'_1 + c_{11} x_{z(1)} + \dots + c_{1t} x_{z(t)} \\
	&\vdots \\
	x_l(r) &= b'_1 + c_{r1} x_{z(1)} + \dots + c_{rt} x_{z(t)}
\end{align*}

\subsubsection{Case II: $\tilde{r} > r$}

In this case, the $(r+1)^{\text{th}}$ row represents an equation of the form $0 = 1$. Therefore, the system is inconsistent.

\subsection{General Solution}

The general solution of $A x = b$ can be expressed by adding the general solution of $A x = b$ and any particular solution of $A x = b$. \\ \\
If $c$ is a solution of $A x = \mathbb{O}$, and $d$ is a solution of $A x = b$, then $c + d$ is a solution of $A x = b$.\\ 
Conversely, if $d$ and $d'$ are solutions of $A x = b$, then, $c = d' - d$ is a solution of $A x = \mathbb{O}$.

\newpage
\part{Vector Spaces}

\section{Definition}

Let $\mathbb{F}$ be a field. A vector space $V$, over $\mathbb{F}$, is a set on which there are two operations, denoted by $+$ and $\cdot$ , where
\begin{itemize}
	\item[] $+$ is the addition of elements of $V$
	\item[] $\cdot$ is the multiplication of an element of $V$ by an element of $\mathbb{F}$
\end{itemize}
s.t. the sum of elements of $V$ lies in $V$, and the product of an element of $V$ by an element of $\mathbb{F}$ lies in $V$, and the following properties hold.
\begin{enumerate}[label=(A\arabic*)]
	\item $x + y = y + x ; \forall x, y \in V$ \label{Vector Spaces: A1}
	\item $(x + y) + x = x + (y + z) ; \forall x, y, z \in V$ \label{Vector Spaces: A2}
	\item $\exists \mathbb{O} \in V$, s.t. $\mathbb{O} + x = x + \mathbb{O} = x ; \forall x \in V$ \label{Vector Spaces: A3}
	\item $\forall x \in V, \exists y \in V$, s.t. $x + y = \mathbb{O}$. ($y$ is denoted as $-x$.) \label{Vector Spaces: A4}
\end{enumerate}
\begin{enumerate}[label=(M\arabic*)]
	\item $\alpha (x + y) = \alpha x + \alpha y ; \forall \alpha \in \mathbb{F}, \forall x, y \in V$ \label{Vector Spaces: M1}
	\item $(\alpha + \beta) x = \alpha x + \beta y ; \forall \alpha, \beta \in \mathbb{F}, \forall x \in V$ \label{Vector Spaces: M2}
	\item $(\alpha \beta) x = \alpha (\beta x) = \beta (\alpha x) ; \forall \alpha, \beta \in \mathbb{F}, \forall x \in V$ \label{Vector Spaces: M3}
	\item $1 \cdot x = x ; \forall x \in V$ \label{Vector Spaces: M4}
\end{enumerate}
Elements of $V$ are called \emph{vectors}, and elements of $\mathbb{F}$ are called \emph{scalars}.

\subsection{Examples}

\subsubsection{Geometric Vectors in Plane}

\subsubsection{Arithmetic Vector Space}

Let $\mathbb{F}$ be a field, and $n \geq 1 \in \mathbb{Z}$.\\
Let $V = \mathbb{F}^n$ be a set of ordered n-tuples.\\
We define 
\begin{align*}
	(\alpha_1, \dots, \alpha_n) + (\beta_1, \dots, \beta_n) &= (\alpha_1 + \beta_2, \dots, \alpha_n + \beta_n) \\
	\alpha (\alpha_1, \dots, \alpha_n) &= (\alpha \alpha_1, \dots, \alpha \alpha_n)
\end{align*}

\subsubsection{}

Let $\mathbb{F}$ be a field, and $m, n \geq 1 \in \mathbb{Z}$.\\
Let $V = \mathbb{F}^{mn}$ be the set of all $(m \times n)$ matrices over $\mathbb{F}$, i.e. a set of ordered mn-tuples. For $X, Y \in V$, we use the usual definitions of $X + Y$ and $\alpha X$ from algebra of matrices.

\section{Properties}

\begin{enumerate}
	\item $\alpha \mathbb{O} = \mathbb{O} ; \forall \alpha \in F$ \label{Vector Spaces: Multiplication by 0}
	\item $\alpha (-x) = -(\alpha x)$
	\item $x - y \doteq x + (-y)$ 
	\item $0 x = \mathbb{O} ; \forall x \in V$
	\item $(-1) x = -x ; \forall x \in V$
	\item $(\alpha - \beta) = \alpha x - \beta x ; \forall \alpha, \beta \in F, \forall x \in V$
\end{enumerate}

\subsubsection{Proof of \ref{Vector Spaces: Multiplication by 0}}

\begin{align*}
	\alpha \mathbb{O} &= \alpha (\mathbb{O} + \mathbb{O} \\
	&= \alpha \mathbb{O} + \alpha \mathbb{O} \\
\end{align*}
For $\alpha \mathbb{O} \exists y$ s.t. $\alpha \mathbb{O} + y = \mathbb{O}$. \\
Therefore,
\begin{align*}
	\alpha \mathbb{O} + y &= (\alpha \mathbb{O} + \alpha \mathbb{O}) + y \\
	\therefore \mathbb{O} &= \alpha \mathbb{O} + (\mathbb{O} + y) \\
	&= \alpha \mathbb{O} + \mathbb{O} \\
	&= \alpha \mathbb{O}
\end{align*}

\section{Subspaces}

Let $V$ be a vector space over $\mathbb{F}$. Let $U \subseteq V$. $U$ is called a subspace of $V$ if the following properties hold.
\begin{enumerate}[label=Axiom \arabic*]
	\item $\mathbb{O} \in U$ \label{Axiom 1}
	\item If $x, y \in U$, then, $(x + y) \in U$ \label{Axiom 2}
	\item If $x \in U, \alpha \in \mathbb{F}$, then, $\alpha x \in U$ \label{Axiom 3}
\end{enumerate}

\subsection{Examples}

\begin{example}
	Let $V$ be the set of all geometric vectors in plane.\\
	If $U_1$ is the set of all vectors along the $x$-axis, $U_2$ is the singleton set of a specific vector along the $x$-axis, and $U_3$ is the set of all vectors along the $x$-axis and a specific vector not along the $x$-axis. Which of $U_1, U_2, U_3$ are subspaces of $V$?
\end{example}

\begin{solution}
	$U_1$ is a subspace of $V$ as it satisfies all three axioms. \\
	$U_2$ is not a subspace of $V$ as it does not satisfy any of the three axioms. \\
	$U_3$ is not a subspace of $V$ as it does not satisfy \ref{Axiom 3}
\end{solution}

\begin{example}
	\begin{align*}
		\mathbb{F} &= \mathbb{R} \\
		V &= \mathbb{C} = \{\alpha + \beta i ; \alpha, \beta \in \mathbb{R}\} \\
		\intertext{where $+$ is addition in $\mathbb{C}$ and $\cdot$ is multiplication by real scalars.}
		U_1 &= \{\alpha + 0 i\} \\
		U_2 &= \{0 + \beta i\}
	\end{align*}
	Which of $U_1, U_2, U_3$ are subspaces of $V$?
\end{example}
\begin{solution}
	Both $U_1$ and $U_2$ are subspaces of $V$, as they satisfy all three axioms.
\end{solution}

\begin{example}
	Let $V = \mathbb{F}$, where $+$ is addition in $\mathbb{F}$, and $\cdot$ is multiplication in $\mathbb{F}$.\\
	\begin{align*}
		U_1 &= \{\alpha + 0 i\} \\
		U_2 &= \{0 + \beta i\}
	\end{align*}
	Which of $U_1, U_2$ are subspaces of $V$?
\end{example}

\begin{solution}
	Neither $U_1$ nor $U_2$ are subspaces of $V$.
\end{solution}

\begin{example}
	Let $V = \{f : [0, 1] \rightarrow \mathbb{R}\}$, where $+$ and $\cdot$ is defined as follows.
	\begin{align*}
		(f + g)(x) &= f(x) + g(x) \\
		(\alpha f)(x) &= \alpha f(x) \\
	\end{align*}
	$\mathbb{O}$ is the function with graph $x = 0$.
	\begin{align*}
		U = \{\text{all continuous functions} [0,1] \rightarrow \mathbb{R}\}
	\end{align*}
	Is $U$ is subspace?
\end{example}

\begin{solution}
	$\mathbb{O} \in \mathbb{R}$. Therefore, \ref{Axiom 1} is satisfied. Similarly, \ref{Axiom 2} and \ref{Axiom 3} are also satisfied.
\end{solution}

\subsection{Operations on Subspaces}

Let $V/F$ be a vector space, and $U_1, U_2$ be subspaces of $V$.

\begin{align*}
	U_1 \cap U_2 &= \{x \in V : x \in U_1 \text{ and } x \in U_2\}\\
	U_1 \cup U_2 &= \{x \in V : x \in U_1 \text{ or } x \in U_2\}\\
	U_1 + U_2 &= \{x \in V : x = x_1 + x_2, x_1 \in U_1, x_2 \in U_2\}\\
\end{align*}

\begin{example}
	Let $V$ be a set of geometric vectors in 3D space. \\
	Let $U_1$ be the $xy$-plane, and $U_2$ be the $yz$-plane.
	If $U_1 \cap U_2$ a subspace of $V$?
\end{example}

\begin{solution}
	\begin{align*}
		\mathbb{O} \in U_1, \mathbb{O} \in U_2 &\Rightarrow \mathbb{O} \in U_1 \cap U_2 \\
		x, y \in U_1 \cap U_2 &\Rightarrow x, y \in U_1, x, y \in U_2 \\
		&\Rightarrow x+y \in U_1, x+y \in U_2 \\
		&= x+y \in U_1 \cap U_2
	\end{align*}
	Similarly, if $x \in U_1 \cap U_2, \alpha in \mathbb{F}$, then, $\alpha x \in U_1 \cap U_2$.
	Therefore, $U_1 \cap U_2$ is a subspace of $V$.
\end{solution}

\section{Spans}

\begin{definition}[Span]
	Let $V/\mathbb{F}$ be a vector space. Let $S \subset V$ be non-empty.\\
	\begin{equation*}
	\vspan(S) = \{x\in V : x = \alpha_1 v_1 + \dots + \alpha_m v_m, \alpha_1, 
	\dots, \alpha_m \in \mathbb{F}, v_1, \dots, v_m \in S\}
	\end{equation*}
	$\vspan(S)$ is the collection of all linear combinations of finite number of vectors of $S$ with coefficients from $\mathbb{F}$
\end{definition}

\begin{theorem}
	$\vspan(S)$ is a subspace of $V$ 
\end{theorem}

\begin{proof}
	\begin{align*}
		\mathbb{O} = 0 v &\Rightarrow \mathbb{O} \in \vspan(S)
	\end{align*}
	\begin{align*}
		x, y \in \vspan(S) &\Rightarrow x = \alpha_1 v_1 + \dots + \alpha_m v_m, \beta_1 w_1 + \dots + \beta_m w_m\\
		&\Rightarrow x+y = \alpha_1 v_1 + \dots + \alpha_m v_m + \beta_1 w_1 + \dots + \beta_m w_m \in \vspan(S)
	\end{align*}
	\begin{align*}
		x \in \vspan(S), \alpha \in \mathbb{F} &\Rightarrow \alpha_1 v_1 + \dots + \alpha_m v_m \\
		&\Rightarrow \alpha x = \alpha(\alpha_1 v_1 + \dots + \alpha_m v_m) \\
		&\Rightarrow \alpha x = \alpha \alpha_1 v_1 + \dots + \alpha \alpha_m v_m \in \vspan(S)
	\end{align*}
\end{proof}

\begin{definition}[Spanning sets and dimensionality]
	Let $V/\mathbb{F}$ be a vector space. A set $S \subseteq V$ is said to be a spanning set, if $\vspan(S) = V$.\\
	If $V$ has atleast one finite spanning set, $V$ is said to be finite-dimensional. Otherwise, $V$ is said to be infinite-dimensional.
	\begin{remark}
		$V$ may have many finite spanning sets, of different sizes
	\end{remark}
\end{definition}

\begin{definition}[Basis of a vector space]
	Let $V/\mathbb{F}$ be a vector space. We say that $B = \{v_1, \dots, v_n\} \subset V$ is a basis of $V$ if every vector $v \in V$ can be expressed in a unique way 
	\begin{equation*}
		v = \alpha_1 v_1 + \dots + \alpha_n v_n \quad; \alpha_1, \dots, \alpha_n \in \mathbb{F}
	\end{equation*}
	that is, as a linear combination of elements of $B$.
\end{definition}

\begin{definition}[Isomorphic spaces]
	Let $V/\mathbb{F}$ and $W/\mathbb{F}$ be vector spaces. We say that $V$ is isomorphic to $W$ if there is a map $\varphi : V \to W$, s.t.
	\begin{enumerate}
		\item $\varphi$ is one-to-one and onto
		\item $\varphi(v_1 + v_2) = \varphi(v_1) + \varphi(v_2) ; \forall v_1, v_2 \in V$
		\item $\varphi(\alpha v) = \alpha \varphi(v) ; \forall v \in V, \forall \alpha \in \mathbb{F}$
	\end{enumerate}
\end{definition}

\begin{theorem}
	If a vector space $V/\mathbb{F}$ has a basis $B = \{v_1, \dots, v_n\}$ consisting of $n$ elements, then it is isomorphic to the space 
	\begin{equation*}
		W = \mathbb{F}^n = 
		\left\lbrace
			\begin{pmatrix}
				\alpha_1 \\
				\vdots \\
				\alpha_n \\
			\end{pmatrix}
		\right\rbrace
	\end{equation*}
\end{theorem}

\begin{proof}
	Let $B' = \{e_1, \dots, e_n\}$, where 
	\begin{equation*}
	e_1 = 
	\begin{pmatrix}
		1\\
		\vdots\\
		0\\
	\end{pmatrix}
	, \dots, 
	e_n = 
	\begin{pmatrix}
		0\\
		\vdots\\
		1\\
	\end{pmatrix}
	\end{equation*}
	$B'$ is a basis of $Q$, as any $w = 
	\begin{pmatrix}
		\alpha_1\\
		\vdots\\
		\alpha_n \\
	\end{pmatrix} \in W$ can be expressed in a unique way 
	\begin{equation*}
		w = \alpha_1 e_1 + \dots + \alpha_n e_n 
	\end{equation*}
	Let $\varphi : V \to W$, 
	\begin{align*}
		\varphi(v_1) &= e_1\\
		\vdots\\
		\varphi(v_n) &= e_n\\
	\end{align*}
	For any $v = \alpha_1 v_1 + \dots + \alpha_n v_n \in V$, 
	\begin{equation*}
		\varphi(v) = 
		\begin{pmatrix}
			\alpha_1\\
			\vdots\\
			\alpha_n\\
		\end{pmatrix}
	\end{equation*}
	Therefore,
	\begin{align*}
		\varphi(\alpha_1 v_1 + \dots + \alpha_n v_n) &= \alpha_1 e_1 + \alpha_n e_n \\
		&= \alpha_1 \varphi(v_1) + \dots + \alpha_n \varphi(v_n)
	\end{align*}
	If $v \neq v'$, 
	\begin{align*}
		v &= \alpha_1 v_1 + \dots + \alpha_n v_n  \\
		v' &= \alpha'_1 v_1 + \dots + \alpha'_n v_n 
	\end{align*}
	Hence $\varphi$ is one-to-one.\\
	For any $w = 
	\begin{pmatrix}
		\alpha_1\\
		\vdots\\
		\alpha_n\\
	\end{pmatrix}
	\in W$.\\
	Let $v = \alpha_1 v_1 + \dots + \alpha_n v_n$.\\
	Therefore, 
	\begin{equation*}
		\varphi(v) = 
		\begin{pmatrix}
			\alpha_1\\
			\vdots\\
			\alpha_n\\
		\end{pmatrix}
		= w
	\end{equation*}
	Therefore, $\varphi$ is onto.
\end{proof}

\section{Linear Dependence}

\begin{definition}[Linearly dependent subsets]
	Let $V/\mathbb{F}$ be a vector space. Let $S \subseteq V$ be a finite subset. $S$ is said to be linearly dependent if there exist scalars $\alpha_1, \dots, \alpha_n \in \mathbb{F}$, not all equal to zero, s.t. 
	\begin{equation*}
		\alpha_1 v_1 + \dots + \alpha_n v_n = \mathbb{O}
	\end{equation*}
	Otherwise, $S$ is said to be linearly independent if all $\alpha_1 = \dots = \alpha_n = 0$.
\end{definition}

\begin{example}
	Is $S = \{v_1, \dots, v_l, v, \alpha v\}$ linearly dependent?
\end{example}

\begin{solution}
	\begin{equation*}
		(0) v_1 + \dots + (0) v_l + (-\alpha) v + (1) \alpha v = \mathbb{O}
	\end{equation*}
	Therefore, as not all coefficients are zero, $S$ is linearly dependent.
\end{solution}

\begin{example}
	Is $S = \{v_1, \dots, v_l, \mathbb{O}\}$ linearly dependent?
\end{example}

\begin{solution}
	\begin{equation*}
	(0) v_1 + \dots + (0) v_l + (1) \mathbb{O} = \mathbb{O}
	\end{equation*}
	Therefore, as not all coefficients are zero, $S$ is linearly dependent.
\end{solution}

\begin{theorem}
	Any basis $B = \{v_1, \dots, v_n\}$ of a vector space $V$ is linearly independent.
\end{theorem}

\begin{proof}
	Let 
	\begin{equation*}
		\alpha_1 v_1 + \dots + \alpha_n v_n = \mathbb{O}
	\end{equation*}
	Also,
	\begin{align}
		(0) v_1 + \dots +(0) v_n = \mathbb{O}
	\end{align}
	Therefore, there are two representations of $v = \mathbb{O}$ as linear combinations of elements of $B$. By the definition of basis, they must coincide.\\
	Therefore,
	\begin{align*}
		\alpha_1 &= 0\\
		\vdots\\
		\alpha_n &= 0
	\end{align*}
	Hence, $B$ is linearly independent.
\end{proof}

\subsection{Properties of Linearly Dependent and Independent Sets}

\begin{theorem}
	If $S \subseteq S'$ and $S$ is linearly dependent, then $S'$ is also linearly dependent.
\end{theorem}

\begin{theorem}
	If $S \subseteq S'$ and $S'$ is linearly independent, then $S$ is also linearly independent.
\end{theorem}

\begin{theorem}
	Let $S = \{v_1, \dots, v_n\}$. $S$ is linearly dependent iff one of the $v_i$s is a linear combination of the others.
\end{theorem}

\begin{proof}[Proof of statement]
	Suppose
	\begin{gather*}
		v_n = \alpha_1 v_1 + \dots + \alpha_{n-1} v_{n-1}\\
		\therefore \alpha_1 v_1 + \dots + \alpha_{n-1} v_{n-1} + (-1) v_n = \mathbb{O}
	\end{gather*}
	Therefore, $S$ is linearly dependent.
\end{proof}

\begin{proof}[Proof of converse]
	Suppose
	\begin{equation*}
		\alpha_1 v_1 + \dots + \alpha_{n-1} v_{n-1} + \alpha_n v_n = \mathbb{O}
	\end{equation*}
	not all of $\alpha_i$s are 0. WLG, let $\alpha_n \neq 0$
	\begin{equation*}
		\therefore v_n = -\dfrac{\alpha_1}{\alpha_m} v_1 - \dots - \dfrac{\alpha_{n-1}}{\alpha_m} v_{m-1}
	\end{equation*}
\end{proof}

\begin{theorem}
	Let $S = \{v_1, \dots, v_m\}$. Let $w \in V$. Suppose $w$ is a linear combination of $v_i$s
	\begin{equation*}
		w = \alpha_1 v_1 + \dots + \alpha_n v_n
	\end{equation*}
	Then, such an expression is unique iff $S$ is linearly dependent.
\end{theorem}

\begin{proof}[Proof of statement]
	Let
	\begin{equation*}
		w = \alpha_1 v_1 + \dots + \alpha_n v_n
	\end{equation*}
	be unique.\\
	If possible, let
	\begin{equation*}
	\beta_1 v_1 + \dots + \beta_n v_n = \mathbb{O}
	\end{equation*}
	not all $\beta_i$s are zero.\\
	Then, 
	\begin{equation*}
		(\alpha_1 + \beta_1) v_1 + \dots + (\alpha_n \beta_n) v_n = w
	\end{equation*}
	This is another expression for $w$, and contradicts the assumption.
\end{proof}

\begin{proof}[Proof of converse]
	If possible, let $S$ be linearly independent.
	Assume 
	\begin{equation*}
	w = \alpha'_1 v_1 + \dots + \alpha'_n v_n
	\end{equation*}
	Therefore,
	\begin{equation*}
		(\alpha_1 - \alpha'_1) v_1 + \dots + (\alpha_n - \alpha'_n) v_n = \mathbb{O}
	\end{equation*}
	Therefore, $S$ is linearly dependent, which contradicts the assumption.
\end{proof}

\begin{theorem}[Main Lemma on Linear Independence]\label{main lemma on linear independence}
	Suppose $V$ is spanned by $n$ vectors.\\
	Let $S = \{v_1, \dots, v_m\} \subset V$. Suppose $m > n$.\\
	Then, $S$ is linearly dependent.
\end{theorem}

\begin{proof}
	Let $E = \{w_1, \dots, w_n\}$ be a spanning set for $V$, $V = \vspan(E)$.\\
	Therefore, all elements of $S$ can be represented as linear combinations of elements of $E$.
	\begin{align*}
		v_1 &= \beta_{11} w_1 + \dots + \beta_{1n} w_n \\
		&\vdots\\
		v_m &= \beta_{m1} w_1 + \dots + \beta{mn} w_n
	\end{align*}
	Let
	\begin{align*}
		\alpha_1 v_1 + \dots + \alpha_m v_m &= \mathbb{O} \\
		\therefore \alpha_1 (\beta_{11} w_1 + \dots + \beta_{1n} w_n) + \dots + \alpha_m (\beta_{m1} w_1 + \dots + \beta_{mn} w_n) &= \mathbb{O}\\
		\therefore (\alpha_1 \beta_{11} + \dots + \alpha_m \beta_{m1}) w_1 + \dots +  (\alpha_1 \beta_{1n} + \dots + \alpha_m \beta_{mn}) &= \mathbb{O}
	\end{align*}
	Therefore
	\begin{align*}
		\alpha_1 \beta_{11} + \dots + \alpha_m \beta_{m1} &= 0\\
		&\vdots \\
		\alpha_1 \beta_{1n} + \dots + \alpha_m \beta_{mn} &= 0
	\end{align*}
	These equations form a homogeneous linear system with respect to $\alpha_1, \dots, \alpha_m$.\\
	As $m > n$, the system has a non-zero solution. Therefore not all $\alpha_i$s are zero.\\
	Hence $S$ is linearly dependent.
\end{proof}

\begin{definition}[Alternative definition of a basis]
	$B = \{v_1, \dots, v_n\}$ is said to be a basis of $V$ if $B$ is a spanning set and $B$ is linearly independent.
\end{definition}

\begin{theorem}
	If $B$ and $B'$ are bases of $V$, then they contain the same number of elements.
\end{theorem}

\begin{proof}
	If possible, let $B$ contain $n$ elements $\{v_1, \dots, v_n\}$, and $B'$ contain $m$ elements $\{w_1, \dots, w_m\}$, $m > n$.\\
	Therefore, $B$ is a spanning set and $B'$ contains more elements than $n$, hence by \nameref{main lemma on linear independence}, $B'$ is linearly dependent. Also, $B'$ is a basis, so it is linearly independent.\\
	This is a contradiction.	
\end{proof}

\begin{definition}[Dimension of a vector space]
	Let $V/\mathbb{F}$ be a finite-dimensional vector space. The number of elements in any basis $B$ of $V$ is called the dimension of $V$.
	\begin{equation*}
		n = \dim V
	\end{equation*}
\end{definition}

\begin{remark}
	If $V$ and $W$ are vector spaces over $\mathbb{F}$, s.t. 
	\begin{equation*}
	\dim V = \dim W
	\end{equation*}
	then, $V$ is isomorphic to $W$
\end{remark}

\begin{theorem}
	If $S = \{v_1, \dots, v_m\}$ is a spanning set of $V$, and if $S$ is not a basis of $V$, a basis $B$ of $V$ can be obtained by removing some elements from $S$.
\end{theorem}

\begin{proof}
	If $S$ is linearly independent, then it is a basis.\\
	Otherwise, if $S$ is linearly dependent, it has an element, WLG, say $v_m$, which is a linear combination of the others.
	\begin{equation*}
		v_m = \alpha_1 v_1 + \dots + \alpha_{m-1} v_{m-1}
	\end{equation*} 
	Let 
	\begin{equation*}
		S' = S - \{v_m\}
	\end{equation*}
	$S'$ is a spanning set.\\
	Therefore, $\forall v \in V$
	\begin{align*}
		v &= \beta_1 v_1 + \dots + \beta_{m-1} v_{m-1} + \beta_m v_m \\
		&= \beta_1 v_1 + \dots + \beta_{m-1} + \beta_m (\alpha_1 v_1 + \dots + \alpha_{m-1} v_{m-1})\\
		&= \gamma_1 v_1 + \dots + \gamma_{m-1} v_{m-1}
	\end{align*}
	If $S'$ is linearly independent, then it is a basis, else the same process above can be repeated till we get a basis.\\
	Therefore, a basis is a smallest spanning set.
\end{proof}

\begin{theorem}
	If $B_0 = \{v_1, \dots, v_n\}$ is a linearly independent set, and if $B_0$ is a basis of $V$, a basis of $V$ can be obtained by adding elements to $B_0$.
\end{theorem}

\begin{theorem}
	Let $V$ be a vector space, s.t. $\dim V = n$.\\
	If $B$ satisfies 2 out of the 3 following conditions, then it is a basis.
	\begin{enumerate}
		\item $B$ has $n$ elements.
		\item $B$ is a spanning set.
		\item $B$ is linearly dependent.
	\end{enumerate}
\end{theorem}

\begin{theorem}[Dimension Theorem]
	\begin{equation*}
		\dim (U + W) = \dim U + \dim W - \dim (U \cap W)
	\end{equation*}
\end{theorem}

\begin{theorem}
	\begin{equation*}
		U + W = \vspan (U \cup W)\\
	\end{equation*}
	If 
	\begin{align*}
		U &= \vspan (B)\\
		W &= \vspan (B')
	\end{align*}
	then, 
	\begin{equation*}
		U + W = \vspan (B \cup B')
	\end{equation*}
\end{theorem}

\begin{proof}
	Let $v \in U + W$.\\
	Then,
	\begin{align*}
		v &= u + w \quad; u \in U, w \in W\\
		u &\in U \cup W\\
		w &\in U \cup W\\
		\therefore v &\in \vspan(U \cup W)\\
	\end{align*}
	Let
	\begin{align*}
		v \in \vspan (U \cup W)
		\therefore v&= \alpha_1 v_1 + \dots + \alpha_k v_k \quad;v_i \in U \cup W
	\end{align*}
	Let
	\begin{align*}
		v_1, \dots, v_l &\in U\\
		v_{l+1}, \dots, v_k &\in W
	\end{align*}
	Therefore,
	\begin{align*}
		v &= (\alpha_1 v_1 + \dots + \alpha_l v_l)+ (\alpha_{l+1} v_{l+1} + \dots + \alpha k v_k)\\
		\therefore v &\in U + W
	\end{align*}
\end{proof}

\subsection{Changing a Basis}

Let $B = \{v_1, \dots, v_n\}$ be a basis of $V$, s.t. $\dim V = n$. Let $B' = \{v'_1, \dots, v'_n\}$.\\
As $B$ is a spanning set, all of $v'_1, \dots, v'_n$ can be expressed as a linear combination of $v_1, \dots, v_n$.
\begin{align*}
	v'_1 &= \gamma_{11} v_1 + \dots + \gamma_{n1} v_n\\
	&\vdots\\
	v'_n &= \gamma_{1n} v_1 + \dots + \gamma_{nn} v_n
\end{align*}

\begin{definition}[Transition matrix]
	The matrix
	\begin{equation*}
		C = 
		\begin{pmatrix}
			\gamma_{11} & \dots &\gamma _{1n}\\
			\vdots & &\vdots\\
			\gamma_{n1} & \dots &\gamma_{nn}\\
		\end{pmatrix}
	\end{equation*}
	is called the transition matrix from $B$ to $B'$.
\end{definition}

If $B$ and $B'$ are considered as row vectors of length $n$ filled by vectors,
\begin{align*}
v'_1 &= \gamma_{11} v_1 + \dots + \gamma_{n1} v_n\\
&\vdots\\
v'_n &= \gamma_{1n} v_1 + \dots + \gamma_{nn} v_n
\end{align*}
can be written as 
\begin{equation*}
B'_{1 \times n} = B_{1 \times n} C_{n \times n}
\end{equation*}


\begin{theorem}
	$B'$ is a basis of $V$ iff $C$ is invertible.
\end{theorem}

\begin{proof}[Proof of statement]
	Let $B' = B C$ be a basis.\\
	$B'$ is a basis, and hence is a spanning set. Therefore, any vector from $B$ can be expressed as a linear combination of elements of $B'$.\\
	Therefore,
	\begin{align*}
		B &= B' Q\\
		&= B C Q\\
		\intertext{Also,}
		B &= B I\\
		\intertext{Therefore,}
		I &= C Q
	\end{align*}
	Similarly,
	\begin{align*}
		B' &= B C\\
		&= B' Q C\\
		\intertext{Also,}
		B' &= B' I\\
		\intertext{Therefore,}
		I &= Q C
	\end{align*}
	Therefore,
	\begin{equation*}
		C Q = Q C = I
	\end{equation*}
	Hence $C$ is invertible.
\end{proof}

\begin{proof}[Proof of converse]
	Let $B' = B C$ and $C$ be invertible. Therefore, $B'$ is a basis iff $B'$ is a spanning set.\\
	Let $z \in V$. As $B$ is a spanning set, 
	\begin{equation*}
		z = \alpha_1 v_1 + \dots + \alpha_n v_n
	\end{equation*}
	Therefore,
	\begin{align*}
		z &= B g\\
		\intertext{where}
		g &= 
		\begin{pmatrix}
			\alpha_1\\
			\vdots\\
			\alpha_n\\
		\end{pmatrix}\\
		\therefore z &= Bg\\
		&= B(Ig)\\
		&= B(C C^{-1}) g\\
		&= (B C) (C^{-1} g)\\
		\intertext{Let $C^{-1} g = f$}
		\therefore z &= B' f
	\end{align*}
	Therefore, $z$ can be expressed as a linear combination of vectors from $B'$.
\end{proof}

\begin{remark}
	Let $B$ be a basis of $V$. If 
	\begin{equation*}
		B P = B Q
	\end{equation*}
	where $P$ and $Q$ are $n \times n$ matrices, then
	\begin{equation*}
		P = Q
	\end{equation*}
\end{remark}

\begin{example}
	Let $B = \{e_1, e_2\}$ and $B' = \{e'_1, e'_2\}$, where 
	\begin{align*}
		e'_1 &= e_1 + e_2\\
		e'_2 &= -e_1 + e_2
	\end{align*}
\end{example}

\begin{solution}
	\begin{align*}
		e'_1 &= e_1 + e_2\\
		e'_2 &= -e_1 + e_2\\
		\therefore C &= 
		\begin{pmatrix}
			1 & -1\\
			1 & 1\\
		\end{pmatrix}
	\end{align*}
	\begin{align*}
		e_1 &= \dfrac{1}{2} e'_1 - \dfrac{1}{2} e'_2\\
		e_2 &= \dfrac{1}{2} e'_1 + \dfrac{1}{2} e'_2\\
		\therefore C^{-1} &=
		\CenteredArraystretch{2}
		\begin{pmatrix}
			\dfrac{1}{2} & \dfrac{1}{2}\\
			-\dfrac{1}{2} & \dfrac{1}{2}
		\end{pmatrix}
	\end{align*}
\end{solution}

\subsection{Representation of Vectors in a Basis}

Let $V$ be a vector space of dimension $n$. Let $B = \{v_1, \dots, v_n\}$ be a basis of $V$.\\
Let $z \in V$.\\
$z$ can be written as a unique linear combination of elements of $B$.
\begin{equation*}
	z = \alpha_1 v_1 + \dots + \alpha_n v_n
\end{equation*}
The representation of $z$ w.r.t $B$ can be represented as
\begin{equation*}
	[z]_B = 
	\begin{pmatrix}
		\alpha_1\\
		\vdots\\
		\alpha_n\\
	\end{pmatrix}
\end{equation*}

\subsubsection{Properties of Representations}

\begin{enumerate}
	\item $[z_1 + z_2]_B = [z_1]_B + [z_2]_B$
	\item $[\alpha z]_B = \alpha [z]_B$
	\item $[z_1]_B = [z_2]_B \iff z_1 = z_2$
	\item $\forall 
	\begin{pmatrix}
		\alpha_1\\
		\vdots\\
		\alpha_n\\
	\end{pmatrix}
	\in \mathbb{F}^n, \exists z \in V, \suchthat [z]_B = 
	\begin{pmatrix}
		\alpha_1\\
		\vdots\\
		\alpha_n\\
	\end{pmatrix}$
\end{enumerate}

\section{Determinants}

\subsection{Definition}

\begin{definition}[Determinants]
	Given an $n \times n$ matrix $A$, $n \geq 1$, $\det (A)$ is defined as follows.
	\begin{align*}
		n &= 1 & \det (a) &= a\\
		n &= 2 & \det 
		\begin{pmatrix}
		a_{11} & a_{12}\\
		a_{21} & a_{22}\\
		\end{pmatrix}
		&= a_{11} a_{22} - a_{12} a_{21}\\
		\vdots\\
		n &= n
	\end{align*}
	The determinant of a $n \times n$ matrix is the summation of $n!$ summands. Each summand is the product of $n$ elements, each from a different row and column.
\end{definition}
\begin{tabular}{|c|c|c|c|}
	\hline
	Summand & Permutation & Number of Elementary Permutations\footnotemark & Parity\\
	\hline
	$a_{11} a_{22} a_{33}$ & 
	$
	\begin{pmatrix}
		1 & 2 & 3\\
		1 & 2 & 3\\
	\end{pmatrix}
	$
	& 0 & even\\
	\hline
	$a_{12} a_{23} a_{31}$ & 
	$
	\begin{pmatrix}
		1 & 2 & 3\\
		2 & 3 & 1\\
	\end{pmatrix}
	$
	& 2 ($(1,2,3) \to (2,1,3) \to (2,3,1)$) & even\\
	\hline
	$a_{13} a_{21} a_{32}$ & 
	$
	\begin{pmatrix}
		1 & 2 & 3\\
		3 & 1 & 2\\
	\end{pmatrix}
	$
	& 2 ($(1,2,3) \to (1,3,2) \to (3,1,2)$) & even\\
	\hline
	$a_{13} a_{22} a_{31}$ & 
	$
	\begin{pmatrix}
		1 & 2 & 3\\
		3 & 2 & 1\\
	\end{pmatrix}
	$
	& 1 ($(1,2,3) \to (3,2,1)$ & odd\\
	\hline
	$a_{12} a_{21} a_{33}$ & 
	$
	\begin{pmatrix}
		1 & 2 & 3\\
		2 & 1 & 3\\
	\end{pmatrix}
	$
	& 1 ($(1,2,3) \to (2,1,3)$ & odd\\
	\hline
	$a_{11} a_{23} a_{32}$ & 
	$
	\begin{pmatrix}
		1 & 2 & 3\\
		1 & 3 & 2\\
	\end{pmatrix}
	$
	& 1 ($(1,2,3) \to (1,3,2)$ & odd\\
	\hline
\end{tabular}
\footnotetext{Any permutation can be represented as a result of a series of elementary permutations, i.e. permutations of 2 elements only. The parity of a particular permutation depends of the parity of the number of elementary functions required for it.}

\subsection{Properties}


\begin{theorem}
	If $A$, $A'$ are matrices s.t. all rows except the $i^{\text{th}}$ row are identical, and $A''$ is obtained by addition of $i^{\text{th}}$ row of $A$ and $i^{\text{th}}$ row of $A'$, then
	\begin{equation*}\label{Property 1}
		\det (A'') = \det (A) + \det (A')
	\end{equation*}
\end{theorem}
	
\begin{theorem}
	If $A'$ is obtained from $A$ by switching two rows, then
	\begin{equation*}\label{Property 2}
		\det (A') = - \det (A)
	\end{equation*}
\end{theorem}
	
\begin{theorem}
	If $A'$ is obtained from $A$ by multiplication of a row by a scalar $\alpha$, then
	\begin{equation*}\label{Property 3}
		\det (A') = \alpha \det (A)
	\end{equation*}
\end{theorem}
	
\begin{theorem}
	If $A'$ is obtained from $A$ by adding to the $i^{\text{th}}$ row the $j^{\text{th}}$ row multiplied by a scalar $\alpha$, then
	\begin{equation*}\label{Property 4}
		\det (A') = \det (A)
	\end{equation*}
\end{theorem}

\begin{corollary}[Corollary of Property 2]
	If $A$ has two identical rows, then $\det (A) = 0$.
\end{corollary}

\begin{theorem}
	The determinant of upper triangular and lower triangular matrices is the product of the elements on the principal diagonal.
\end{theorem}

\begin{theorem}
	\begin{equation*}
		\det (A^t) = \det (A)
	\end{equation*}
\end{theorem}

\begin{corollary}
	In all above theorems, the properties which are applicable to rows, are also applicable to columns.
\end{corollary}

\begin{theorem}
	If $A$, $B$, $C$ are some matrices, and $\mathbb{O}$ is the zero matrix,
	\begin{align*}
		\begin{pmatrix}
			A_{m \times m} & B\\
			\mathbb{O} & C_{n \times n}\\
		\end{pmatrix}
		=
		\det (A) \cdot \det (C)
	\end{align*}
\end{theorem}

\begin{theorem}
	\begin{equation*}
		\det (A B) = \det (A) \det (B)
	\end{equation*}
\end{theorem}

\begin{corollary}
	If $A$ is invertible, then
	\begin{equation*}
		\det (A) \neq 0
	\end{equation*}
\end{corollary}

\begin{proof}
	$A$ is invertible.\\
	Therefore, $\exists P$, s.t.
	\begin{align*}
		P A &= I\\
		\therefore \det (PA) &= \det (I)\\
		\therefore \det (P) \det (A) &= 1\\
		\therefore \det (A) \neq 0
	\end{align*}
\end{proof}

\begin{theorem}
	If
	\begin{equation*}
		\det (A) \neq 0
	\end{equation*}
	then $A$ is invertible.
\end{theorem}

\begin{proof}
	If possible let $A$ be non invertible.\\
	Let the REF of $A$ be $A_R$.\\
	As $A$ is non invertible, $A_R$ has a zero row. Therefore,
	\begin{equation*}
		\det (A_R) = 0
	\end{equation*}
	But
	\begin{equation*}
		\det (A) = 0
	\end{equation*}
	This is not possible as elementary row operations cannot change a non-zero determinant to zero.\\
	Therefore, $A$ is invertible.
\end{proof}

\begin{theorem}
	\begin{align*}
		\det (A) \neq 0
	\end{align*}
	iff the rows of $A$ are linearly independent iff the columns of $A$ are linearly independent.
\end{theorem}

\begin{proof}
	If possible, let the rows of $A$ be linearly dependent.\\
	Therefore, either all of them are zeros, or one row is the linear combination of the others.
	
	\begin{case}[All rows are zeros]
		\begin{equation*}
			\therefore \det (A) = 0
		\end{equation*}
	\end{case}
	
	\begin{case}[One row is a linear combination of the others]
		Let 
		\begin{align*}
			v_n &= \alpha_1 v_1 + \dots + \alpha_{n - 1} v_{n - 1}\\
			\therefore A &= 
			\begin{pmatrix}
				v_1\\
				\vdots\\
				v_{n - 1}\\
				v_n\\
			\end{pmatrix}
			\intertext{$v_n \to v_n - \alpha_1 v_1 + \dots + \alpha_{n - 1} v_{n - 1}$}
			\therefore A' &= 
			\begin{pmatrix}
				v_1\\
				\vdots\\
				v_{n - 1}\\
				\mathbb{O}\\
			\end{pmatrix}\\
			\therefore \det (A') &= 0\\
			\therefore \det (A) &= 0
		\end{align*}
	\end{case}
	This contradicts $\det (A) \neq 0$. Therefore, the rows of $A$ must be linearly independent.
	
	\vspace{1 in}
	
	If $v_1, \dots, v_n$ are linearly independent, 
	\begin{align*}
		\dim \R(A) &= n\\
		\therefore r&= n
	\end{align*}
	Therefore, there are no zero rows in REF of $A$. Hence $A$ is invertible.
	\begin{align*}
		\therefore \det (A) \neq 0
	\end{align*}
\end{proof}

\subsection{Practical Methods for Computing Determinants}

\subsection{Expansion along a row/ column}

Let $A$ be a $m \times n$ matrix, and let $A_{ij}$ be the matrix obtained by removing the $i^{\text{th}}$ row and $j^{\text{th}}$ column from $A$.

\begin{align*}
	\det (A) &= \sum_{j = 1}^{n} (-1)^{i + j} a_{ij} \det (A_{ij})
\end{align*}

\subsection{Determinant Rank}

\begin{definition}
	Let $A$ be any $m \times n$ matrix. Consider all square sub-matrices of $A$ and compute their determinants. If there is an $r \times r$ sub-matrix of $A$ s.t. its determinant is non-zero, but the determinants of all $(r + 1) \times (r + 1)$ sub-matrices of $A$ are zero, then, $r$ is called the determinant rank of $A$.
\end{definition}

\begin{theorem}
	The determinant rank of $A$ is equal to the rank of $A$.
\end{theorem}

\section{Linear Maps}

\subsection{Definition}

\begin{definition}
	Let $V$ and $W$ be vector spaces over the same field $\mathbb{F}$. 
	\begin{equation*}
		\varphi : V \to W
	\end{equation*}
	is said to be a linear map if
	\begin{enumerate}
		\item $\varphi(v_1 + v_2) = \varphi(v_1) + \varphi(v_2) ; \forall v_1, v_2 \in V$
		\item $\varphi(\alpha v) = \alpha \varphi(v) ; \forall v \in V, \forall \alpha \in \mathbb{F}$
	\end{enumerate}
\end{definition}

\subsection{Properties}

\begin{enumerate}
	\item $\varphi (\mathbb{O}) = \mathbb{O}$
	\item $\varphi (-v) = - \varphi (v)$
\end{enumerate}

\subsection{Matrix of a Linear Map}

\begin{definition}
	Let $\varphi : V \to W$ be a linear map.\\
	Let
	\begin{align*}
		n &= \dim V\\
		m &= \dim W
	\end{align*}
	Let
	\begin{align*}
		B &= \{v_1, \dots, v_n\}\\
		B' &= \{w_1, \dots, w_m\}
	\end{align*}
	be bases of $V$ and $W$ respectively.\\
	Let 
	\begin{align*}
		\varphi (v_1) &= \alpha_{11} w_{1} + \dots + \alpha_{m1} w_{m}\\
		\vdots\\
		\varphi (v_n) &= \alpha_{1n} w_{1} + \dots + \alpha_{mn} w_{m}
	\end{align*}
	The matrix
	\begin{equation*}
		A = 
		\begin{pmatrix}
			\alpha_{11} & \dots & \alpha_{1n}\\
			\vdots & & \vdots\\
			\alpha_{m1} & \dots & \alpha_{mn}
		\end{pmatrix}
	\end{equation*}
	is called the matrix of $\varphi$ with respect to the bases $B$ and $B'$.\\
	It is denoted as
	\begin{equation*}
		A = [\varphi]_{B, B'}
	\end{equation*}
\end{definition}

\begin{theorem}
	Let 
	\begin{equation*}
		\varphi : V \to W
	\end{equation*}
	be a linear map.\\
	Let $B$ and $B'$ be bases of $V$ and $W$ respectively, and let 
	\begin{equation*}
		A = [\varphi]_{B, B'}
	\end{equation*}
	be the matrix of $\varphi$ with respect to $B$ and $B'$. Then, $\forall x \in V$,
	\begin{equation*}
		[\varphi(z)]_{B'} = A [z]_B
	\end{equation*}
\end{theorem}

\begin{proof}
	Let
	\begin{align*}
		B &= \{v_1, \dots, v_n\}\\
		B' &= \{w_1, \dots, w_m\}
	\end{align*}
	
	\begin{case}[$z \in B$]
		WLG, let $z =  v_i$. Then,
		\begin{equation*}
			[z]_B = 
			\begin{pmatrix}
				0\\
				\vdots\\
				1\\
				\vdots\\
				0\\
			\end{pmatrix}
		\end{equation*}
		i.e. all rows except the $i^{\text{th}}$ row are 0.\\
		Let this vector be $e_i$.\\
		Therefore, 
		\begin{equation*}
			A [z]_B = A e_i
		\end{equation*}
		is the $i^{\text{th}}$ column of $A$.\\
		
		\begin{align*}
			[\varphi (z)]_{B'} = [\varphi (v_i)]_{B'}
		\end{align*}
		is the $i^{\text{th}}$ row in the formulae of $\varphi(v_1), \dots, \varphi(v_n)$.\\
		Therefore, it is the $i^{\text{th}}$ column of $A$.
	\end{case}
	
	\begin{case}[$z \in V$ is an arbitrary vector]
		Let 
		\begin{equation*}
			z = \alpha_1 v_1 + \dots + \alpha_n v_n
		\end{equation*}
		Therefore,
		\begin{align*}
			[\varphi(z)]_{B'} &= [\varphi(\alpha_1 v_1 + \dots + \alpha_n v_n)]_{B'}\\
			&= [\alpha_1 \varphi(v_2) + \dots + \alpha_n \varphi(v_n)]_{B'}\\
			&= \alpha_1 [\varphi(v_1)]_{B'} + \dots + \alpha_n [\varphi(v_n)]_{B'}\\
			&= \alpha_1 \cdot (1^{\text{st}} \text{column of } A) + \dots + \alpha_n c_n \cdot (n^{\text{th}} \text{column of } A)\\
			&= A [z]_B
		\end{align*}
	\end{case}
\end{proof}

\subsection{Change of Bases}

\begin{theorem}
	Let $V$, $W$ be vector spaces over $\mathbb{F}$, $\dim (V) = n$, $\dim (W) = m$. Let $\varphi : V \to W$ be a linear map. Let $B$, $\tilde{B}$ be bases of $V$ and let $B'$ and $\tilde{B'}$ be bases of $W$. Let $A = [\varphi]_{B, B'}$ and $\tilde{A} = [\varphi]_{\tilde{B}, \tilde{B'}}$ be the matrices of $\varphi$ w.r.t. the pairs $B$, $B'$ and $\tilde{B}$, $\tilde{B'}$. Let $P$ denote the transition matrix from $B$ to $\tilde{B}$, and let $Q$ denote the transition matrix from $B'$ to $\tilde{B'}$. Then,
	\begin{equation*}
		\tilde{A}_{m \times n} = Q^{-1}_{m \times m} A_{m \times n} P_{n \times n}
	\end{equation*}
\end{theorem}

\begin{proof}
	$\forall z \in V$,
	\begin{align}
		[\varphi(z)]_{B'} &= A [z]_B\label{eq 1}\\
		[\varphi(z)]_{\tilde{B'}} &= A [z]_{\tilde{B}}\label{eq 2}
	\end{align}
	We have
	\begin{align}
		[z]_B &= P [z]_{\tilde{B}}\label{eq 3}\\
		[\varphi(z)]_{B'} &= Q [\varphi(z)]_{\tilde{B'}}\label{eq 4}
	\end{align}
	Therefore,
	\begin{align}
		\intertext{\eqref{eq 1} in \eqref{eq 4} $\implies$}
		A [z]_B &= Q [\varphi(z)]_{\tilde{B'}}\label{eq 5}\\
		\intertext{\eqref{eq 3} in \eqref{eq 5} $\implies$}
		A P [z]_{\tilde{B}} &= Q [\varphi(z)]_{\tilde{B'}}
	\end{align}
	Multiplying on the left by $Q^{-1}$,
	\begin{align*}
		Q^{-1} A P [z]_{\tilde{B}} &= [\varphi(z)]_{\tilde{B'}}\\
		\therefore [\varphi(z)]_{\tilde{B'}} &= Q^{-1} A P [z]_{\tilde{B}}
		\intertext{Comparing with \eqref{eq 2},}
		\tilde{A} &= Q^{-1} A P
	\end{align*}
\end{proof}

\subsection{Operations on Linear Maps}

\begin{definition}
	Let
	\begin{align*}
		\varphi : V \to W\\
		\varphi' : V \to W
	\end{align*}
	be linear maps.
	\begin{align*}
		\varphi + \varphi' &: V \to W\\
		\intertext{is defined as}
		(\varphi + \varphi')(v) &= \varphi(v) + \varphi'(v)\\
		\intertext{and}
		\alpha \varphi &: V \to W\\
		\intertext{is defined as}
		(\alpha \varphi)(v) &= \alpha \varphi(v)
	\end{align*}
\end{definition}

\begin{definition}
	Let
	\begin{align*}
		\varphi : V \to W\\
		\varphi' : W \to U
	\end{align*}
	be linear maps.
	\begin{align*}
	(\varphi' \circ \varphi) &: V \to U\\
	\intertext{is defined as}
	(\varphi' \circ \varphi)(v) &= \varphi'(\varphi(v))
	\end{align*}
\end{definition}

\begin{theorem}[Matrix of composed map]
	Let $\varphi : V \to W$, $\varphi' : W \to U$ be linear maps. Let $(\varphi \circ \varphi') : V \to U$ be the composed map. Let $\dim V = n$, $\dim W = m$, $\dim U = l$. Let $B$, $B'$, $B''$ be bases of $V$, $W$, $U$ respectively. Let $A = [\varphi]_{B, B'}$, $A' = [\varphi']_{B', B''}$ be the matrices of $\varphi$, $\varphi'$. Let $A'' = [\varphi' \circ \varphi]_{B, B''}$ be the matrix of the composed map. Then, 
	\begin{equation*}
		A'' = A' A
	\end{equation*}
\end{theorem}

\begin{proof}
	Let $z \in V$.
	\begin{align*}
		[(\varphi' \circ \varphi)(z)]_{B''} &= [\varphi' (\varphi (z))]_{B''}\\
		&= A' [\varphi (z)]_{B'}\\
		&= A' A [z]_B
	\end{align*}
	By definition,
	\begin{align*}
		[(\varphi' \circ \varphi) (z)]_{B''} &= A'' [z]_B
	\end{align*}
	Therefore,
	\begin{align*}
		A'' &= A' A
	\end{align*}
\end{proof}

\subsection{Kernel and Image}

\begin{definition}
	Let $\varphi : V \to W$ be a linear map.
	\begin{align*}
		\ker \varphi &\doteq \{v \in V : \varphi (v) = \mathbb{O}\}\\
		\im \varphi &\doteq \{\phi (v) : v \in V\}
	\end{align*}
\end{definition}

\begin{theorem}
	$\ker \varphi$ is a subspace of $V$ and $\im \varphi$ is a subspace of $W$.
\end{theorem}

\begin{proof}
	\begin{align*}
		\varphi (\mathbb{O}) &= \mathbb{O}\\
		\therefore \mathbb{O} &\in \ker \varphi
	\end{align*}
	If $v_1, v_2 \in \ker \varphi$, then
	\begin{align*}
		\varphi (v_1 + v_2) &= \varphi (v_1) + \varphi (v_2)\\
		&= \mathbb{O} + \mathbb{O}\\
		&= \mathbb{O}\\
		\therefore v_1 + v_2 \in \ker V
	\end{align*}
	If $v \in \ker \varphi$, $\alpha \in \mathbb{F}$, then
	\begin{align*}
		\varphi (\alpha v) &= \alpha \varphi (v)\\
		&= \alpha \mathbb{O} \\
		&= \mathbb{O}
		\therefore \alpha v \in \ker \varphi
	\end{align*}
	Therefore, $\ker \varphi$ is a subspace of $W$.\\
	\begin{align*}
		\varphi (\mathbb{O}) &= \mathbb{O}\\
		\therefore \mathbb{O} &\in \im \varphi
	\end{align*}
	If $w_1, w_2 \in \im \varphi$, then
	\begin{align*}
		w_1 &= \varphi (v_1)\\
		w_2 &= \varphi (v_2)\\
		\therefore w_1 + w_2 &= \varphi (v_1) + \varphi (v_2)\\
		&= \varphi (v_1 + v_2)\\
		\therefore w_1 + w_2 &\in \im \varphi
	\end{align*}
	If $w \in W$, $\alpha \in \mathbb{F}$, then
	\begin{align*}
		\alpha w &= \alpha \phi (v)\\
		&= \varphi (\alpha v)\\
		\therefore \alpha w &\in \im \varphi
	\end{align*}
	Therefore, $\im \varphi$ is a subspace of $W$.
\end{proof}

\subsubsection{Dimensions of Kernel and Image}

\begin{theorem}
	Let $\varphi : V \to W$ be a linear map. Then
	\begin{equation*}
		\dim (\ker (\varphi)) + \dim (\im (\varphi))
	\end{equation*}
\end{theorem}

\begin{proof}
	Let $\ker \varphi = U$, $U \subseteq V$.\\
	Let $B_0 = \{v_1, \dots, v_k\}$ be a basis of $U$.\\
	Completing $B_0$ to a basis $B$ of $V$, 
	\begin{equation*}
		B = \{v_1, \dots, v_k, v_{k+1}, \dots, v_n\}
	\end{equation*}
	Let
	\begin{align*}
		w_{k+1} &= \varphi(v_{k+1})\\
		&\vdots\\
		w_n &= \varphi(v_n)
	\end{align*}
	Therefore, we need to prove that $B'$ is a basis of $W' = \im (\varphi)$, by proving that $B'$ is a spanning set and that $B'$ is linearly independent.\\
	Take $w \in \im(\varphi)$, so that there is $v \in V$ s.t. $\varphi (v) = w$.\\
	Representing $v$ as a linear combination of elements of $B$,
	\begin{align*}
		v &= \alpha_1 v_1 + \dots + \alpha_k v_k + \alpha_{k+1} v_{k+1} + \dots + \alpha_n v_n\\
		\therefore w &= \varphi(v)\\
		&= \varphi(\alpha_1 v_1 + \dots + \alpha_k v_k + \alpha_{k+1} v_{k+1} + \dots + \alpha_n v_n)\\
		&= \alpha_1 \varphi(v_1) + \dots + \alpha_k \varphi(v_k) + \alpha_{k+1} \varphi(v_{k+1}) + \dots + \alpha_n \varphi(v_n)\\
		&= \alpha_{k+1} \varphi(v_{k+1}) + \dots + \alpha_n \varphi(v_n)\\
		&= \alpha_{k+1} w_{k+1} + \dots + \alpha_n w_n\\
		&\in \vspan(B')
	\end{align*}
	Therefore, $B'$ is a spanning set for $W'$.\\
	Let
	\begin{equation*}
		\beta_{k+1} w_{k+1} + \dots + \beta_n w_n = \mathbb{O}
	\end{equation*}
	Therefore, $B'$ is linearly independent iff 
	\begin{equation*}
		\beta_{k+1} = \dots = \beta_n = 0
	\end{equation*}
	As $\varphi$ is a linear map, 
	\begin{align*}
		\varphi (\beta_{k+1} v_{k+1} + \dots + \beta_n v_n) &= \mathbb{O}\\
		\therefore \beta_{k+1} v_{k+1} + \dots + \beta_n v_n &\in \ker \varphi
	\end{align*}
	Therefore, it can be expressed as a linear combination of vectors of $B_0$, which is a basis of $\ker \varphi$.\\
	Let 
	\begin{align*}
		\beta_{k+1} v_{k+1} + \dots + \beta_n v_n &= \alpha_{k+1} v_{k+1} + \dots + \alpha_n v_n\\
		\therefore \alpha_{k+1} v_{k+1} + \dots + \alpha_n v_n - \beta_{k+1} v_{k+1} - \dots - \beta_n v_n &= \mathbb{O}
	\end{align*}
	As $\{v_1, \dots, v_n\}$ is a basis of $V$, all coefficients must be 0\\
	Therefore,
	\begin{equation*}
		\beta_{k+1} v_{k+1} = \dots = \beta_n v_n = 0
	\end{equation*}
	Hence, as $B'$ is a spanning set of $\im \varphi$ and also linearly independent, $B'$ is a basis of $\im \varphi$.\\
	Therefore,
	\begin{align*}
		\dim (\im \varphi) &= \text{ size of $B'$ }\\
		&= n - k\\
		&= n - \dim (\ker \varphi)\\
		\therefore \dim (\im \varphi) + \dim (\ker \varphi) &= \dim V
	\end{align*}
\end{proof}

\begin{corollary}
	\begin{align*}
		\dim (\im \varphi) &= r
	\end{align*}
	where $r$ is the rank of $A$
\end{corollary}

\begin{corollary}
	Let $A_{m \times n}$ be a matrix of rank $r$. Let $\C(A)$ be the column space of $A$, and let $\dim \C(A)$ be the column rank of $A$. Then
	\begin{equation*}
		\dim \C(A) = r
	\end{equation*}
\end{corollary}

\begin{proof}
	Define
	\begin{equation*}
		\varphi : \mathbb{F}^n \to \mathbb{F}^m
	\end{equation*}
	s.t. $A = [\varphi]_{B, B'}$, where $B$ is the standard basis of $\mathbb{F}$.
	\begin{align*}
		B &= 
		\left\lbrace
			\begin{pmatrix}
				1\\
				\vdots\\
				0\\
			\end{pmatrix}
			,
			\dots
			,
			\begin{pmatrix}
				0\\
				\vdots\\
				1\\
			\end{pmatrix}
		\right\rbrace\\
		&= \{e_1, \dots, e_n\}
	\end{align*}
	$\forall v \in \mathbb{F}^n$, we have
	\begin{equation*}
		[\varphi(v)]_{B'} = A [v]_B
	\end{equation*}
	If $v = e_i$,
	\begin{equation*}
		[\varphi (e_i)] = A e_i
	\end{equation*}
	which is the $i^{\text{th}}$ column of $A$.
	So, the space spanned by $\{\varphi(e_1), \dots, \varphi(e_n)\}$ is equal to $\C(A)$. But it is also in $\im \varphi$.\\
	Therefore,
	\begin{align*}
		\im \varphi &= \C(A)\\
		\intertext{and}
		\dim (\im \varphi) &= \dim (\C(A))\\
		\therefore r &= \dim (\C(A))
	\end{align*}
\end{proof}

\begin{remark}
	Let $\varphi : V \to W$ be a linear map. Let $w \in \im (\varphi)$, so that there is $v \in V$ s.t. $\varphi (v) = w$. Then any $v'$ s.t. $\varphi(v') = w$ can be written down as $v' = v + v_0$ where $v_0 \in \ker \varphi$.
\end{remark}

\part{Linear Operators}

\section{Definition}

\begin{definition}
	A linear operator or transformation
	\begin{equation*}
		T : V \to V
	\end{equation*}
	is a linear map from a vector space $V$ to itself.
\end{definition}

\section{Similar Matrices}

Let $B$ and $\tilde{B}$ be bases of $V$. Let $A$ and $\tilde{A}$ be the representing matrices
\begin{align*}
	A &= [T]_B\\
	\tilde{A} &= [T]_{\tilde{B}}
\end{align*}
Both these are $n \times n$ matrices, where $n = \dim V$. Let $P$ denote the transition matrix from $B$ to $\tilde{B}$. Then,
\begin{equation*}
	\tilde{A} = P^{-1} A P
\end{equation*}

\begin{definition}
	Let $A$, $\tilde{A}$ be $n \times n$ matrices. $A$ is said to be similar to $\tilde{A}$, denoted as $A \sim \tilde{A}$, if there exists an invertible $n \times n$ matrix $P$, s.t. $\tilde{A} = P^{-1} A P$.
\end{definition}

\subsection{Properties of Similar Matrices}

\begin{enumerate}
	\item $A \sim A$ 
	\item If $A \sim \tilde{A}$, then $\tilde{A} \sim A$
	\item If $A \sim \tilde{A}$ and $\tilde{A} \sim \tilde{\tilde{A}}$, then $A \sim \tilde{\tilde{A}}$
	\item If $A \sim \tilde{A}$, then $\det (A) = \det (\tilde{A})$
	\item If $A \sim I$, then $A = I$
\end{enumerate}

\section{Diagonalization}

Given a square matrix $A_{n \times n}$, decide whether or not $A$ is similar to some diagonal matrix $D$. If it is, find $D$, and $P$ s.t. $P^{-1} A P = D$.\\
Alternatively,\\
Given an operator $T : V \to V$, decide whether or not there exists a basis $B$ of $V$, s.t. $[T]_B$ is a diagonal matrix D. If it exists, find $D$, and $B$, s.t. $[T]_B = D$.

\begin{definition}[Diagonalizability]
	If $A$ is similar to a diagonal matrix, $A$ is said to be diagonalizable. $P$, s.t. $P^{-1} A P = D$ is called a diagonalizing matrix for $A$. $D$ is called a diagonal form of $A$.
\end{definition}

\section{Eigenvalues and Eigenvectors}

\begin{definition}[Eigenvalue and eigenvector]
	Let $A$ be a $n \times n$ matrix over $\mathbb{F}$. $\lambda \in \mathbb{F}$ is said to be an eigenvalue of $A$, if $\exists v \in \mathbb{F}, v\neq 0$, such that
	\begin{equation*}
		A v = \lambda v
	\end{equation*}
	$v$ is called an eigenvector corresponding to $\lambda$.
\end{definition}

\begin{definition}[Alternate definition of eigenvalue and eigenvector]
	Let $T : V \to V$ be a linear operator, where $V$ is a vector space over $\mathbb{F}$. $\lambda \in \mathbb{F}$ is said to be an eigenvalue of $A$, if $\exists v \in V, v\neq 0$, such that
	\begin{equation*}
		T(v) = \lambda v
	\end{equation*}
	$v$ is called an eigenvector corresponding to $\lambda$.
\end{definition}

\begin{definition}[Spectrum]
	The collection of all eigenvalues of a matrix, or a linear operator is called the spectrum.
\end{definition}

\begin{theorem}
	Let $A$ be a $n \times n$ matrix. $\lambda \in \mathbb{F}$ is an eigenvalue of $A$ iff
	\begin{equation*}
		\det (\lambda I_n - A) = 0
	\end{equation*}
\end{theorem}

\begin{proof}
	$\lambda$ is an eigenvalue of $A$
	\begin{align*}
		&\iff \exists v \in \mathbb{F}^n, v \neq 0, \suchthat Av = \lambda v\\
		&\iff \exists v \in \mathbb{F}^n, v \neq 0, \suchthat (\lambda I - A) v = \mathbb{O}\\
		&\iff v = 
		\begin{pmatrix}
			x_1\\
			\vdots\\
			x_n\\
		\end{pmatrix}\\
		&\iff (\lambda I - A) 
		\begin{pmatrix}
			x_1\\
			\vdots\\
			x_n\\
		\end{pmatrix}
		= 0 \text{ has a non-zero solution }\\
		&\iff \text{there are free variables }\\
		&\iff \det (\lambda I - A) = 0
	\end{align*}
\end{proof}

\begin{theorem}[General criterion for diagonalization]
	Let $A$ be a $n \times n$ matrix. $A$ is diagonalizable if and only if there exists a basis $B = \{v_1, \dots, v_n\}$ of $\mathbb{F}^n$ consisting of eigenvectors of $A$. In such a case, the diagonal entries of $D$ are eigenvalues of $A$, and $B$ can be chosen as consisting of the columns of $P$, where $P^{-1} A P = D$.
\end{theorem}

\begin{corollary}
	If $A$ has no eigenvalues, then it is not diagonalizable.
\end{corollary}

\end{document}
