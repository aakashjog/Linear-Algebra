\documentclass[fleqn, a4paper, 12pt]{article}
\usepackage{amsmath, amssymb, amsthm, thmtools, amsfonts, commath}
\usepackage{datetime}
\usepackage{hyperref}
\usepackage{ulem}
\usepackage{tikz}
\usepackage{enumerate, enumitem}
\usepackage{cancel}
\usepackage{xfrac}
%\usepackage{background}
\setcounter{secnumdepth}{4}

\newcommand\numberthis{\addtocounter{equation}{1}\tag{\theequation}}

\DeclareMathOperator{\vspan}{\mathrm{span}} %declares span operator for matrices

\newenvironment{amatrix}[1]{%declares augmented matrix environment
	\left(\begin{array}{@{}*{#1}{c}|c@{}}
	}{%
\end{array}\right)
} 

\theoremstyle{definition}
\newtheorem{example}{Example} %defines example environment
\newtheorem{definition}{Definition} %defines definition environment

\theoremstyle{theorem}
\newtheorem{theorem}{Theorem} %defines theorem environment
\newtheorem{corollary}{Corollary}

\theoremstyle{remark}
\newtheorem{remark}{Remark}
\newtheorem{case}{Case}

\newcommand{\suchthat}{\mathrm{\,s.t.\,}}

\newcommand{\R}{\mathrm{R}}

\newcommand{\C}{\mathrm{C}}

\newcommand{\N}{\mathrm{N}}

\newcommand{\rr}{\mathrm{rr}}

\newcommand{\im}{\mathrm{im}\,}

\newcommand{\trace}{\mathrm{trace}}

\newcommand{\distance}{\mathrm{d}}

\newenvironment{solution} %declares solution environment and removes qed at end
	{\begin{proof}[Solution]\let\qed\relax}
	{\end{proof}}

\makeatletter
\@addtoreset{section}{part} %resets section numbers in new part
\makeatother

\makeatletter
\@addtoreset{theorem}{part} %resets theorem numbers in new part
\makeatother

\makeatletter
\@addtoreset{corollary}{theorem} %resets corollary numbers after a theorem
\makeatother

\numberwithin{corollary}{theorem}

\numberwithin{equation}{theorem}

\makeatletter %changes spacing between rows of matrices to fit fractions
\newif\ifcenter@asb@\center@asb@false
\def\center@arstrutbox{%
	\setbox\@arstrutbox\hbox{$\vcenter{\box\@arstrutbox}$}%
}
\newcommand*{\CenteredArraystretch}[1]{%
	\ifcenter@asb@\else
	\pretocmd{\@mkpream}{\center@arstrutbox}{}{}%
	\center@asb@true
	\fi
	\renewcommand{\arraystretch}{#1}%
}
\makeatother

\title{Review Session 2}
\author{Aakash Jog}
\date{\formatdate{25}{1}{2015}}

\begin{document}

\maketitle

\begin{example}
	Consider $V = M_{2 \times 2}(\mathbb{R})$ with inner product 
	\begin{equation*}
		\langle A, B \rangle = \trace 
			\left(
				B^T 
					\begin{pmatrix}
						1 & 2\\
						2 & 5\\
					\end{pmatrix}
				A
			\right)
	\end{equation*}
	Find an orthonormal basis of $V$ with respect to this inner product.
\end{example}

\begin{solution}
	\begin{align*}
		v_1 &= e_1\\
		v_2 &= e_2 - \dfrac{\langle e_2, v_1 \rangle}{{\| v_1 \|}^2} v_1\\
		&= e_2 - 
			\dfrac
			{\trace
				\left( 
					\begin{pmatrix}
						1 & 0\\
						0 & 0\\
					\end{pmatrix}
					\begin{pmatrix}
						1 & 2\\
						2 & 5\\
					\end{pmatrix}
					\begin{pmatrix}
						0 & 1\\
						0 & 0\
					\end{pmatrix}
				\right)
			}
			{\trace
				\left( 
				\begin{pmatrix}
				1 & 0\\
				0 & 0\\
				\end{pmatrix}
				\begin{pmatrix}
				1 & 2\\
				2 & 5\\
				\end{pmatrix}
				\begin{pmatrix}
				1 & 0\\
				0 & 0\
				\end{pmatrix}
				\right)
			}
			e_1\\
		&= e_2\\
		v_3 &= 
			\begin{pmatrix}
				-2 & 0\\
				1 & 0\\
			\end{pmatrix}\\
		v_4 &= 
			\begin{pmatrix}
				0 & -2\\
				0 & 1\\
			\end{pmatrix}
	\end{align*}
\end{solution}

\begin{example}
	Let $A \in M_{6 \times 6} (\mathbb{C})$. Given
	\begin{equation*}
		p_A(\lambda) = \lambda^2 (\lambda - 2) (\lambda + 1)^3
	\end{equation*}
	find $\det(A^2 + A)$.
\end{example}

\begin{solution}
	\begin{align*}
		\det(A + A^2) &= \det(A (A + I))\\
		&= \det(A) \det(A + I)
	\end{align*}
	As $0$ is an eigenvalue of $A$, the product of all eigenvalues is $0$. Therefore, $\det(A) = 0$.\\
	Therefore, $\det(A^2 + A) = 0$.
\end{solution}

\end{document}
