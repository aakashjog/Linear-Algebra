\documentclass[fleqn, a4paper, 12pt]{article}
\setcounter{secnumdepth}{4}
\usepackage{amsmath, amssymb, amsthm}
\usepackage{mathtools, xfrac}
\usepackage{datetime}
\usepackage{ulem}
\usepackage{enumerate}
\usepackage[table]{xcolor}

\newcommand\numberthis{\addtocounter{equation}{1}\tag{\theequation}}

\newcommand{\rank}{\mathrm{rk\,}}

\newcommand{\R}{\mathrm{R}}

\newcommand{\C}{\mathrm{C}}

\newcommand{\N}{\mathrm{N}}

\newcommand{\im}{\mathrm{im}\,}

\newenvironment{amatrix}[1]{%
	\left(\begin{array}{@{}*{#1}{c}|c@{}}
}{%
	\end{array}\right)
}

%\setlength{\mathindent}{0pt}

\theoremstyle{definition}
\newtheorem{example}{Example}
\newtheorem{definition}{Definition}

\theoremstyle{theorem}
\newtheorem{theorem}{Theorem}

\newenvironment{solution}
{\begin{proof}[Solution]\let\qed\relax}
	{\end{proof}}

\DeclareMathOperator{\vspan}{\mathrm{span}} %declares span operator for matrices

%opening
\title{Recitation 13}
\author{}
\date{\formatdate{14}{1}{2015}}

\begin{document}

\maketitle
%\setlength{\mathindent}{0pt}

\tableofcontents

\newpage
\section{Orthogonality}

\begin{example}
	Find scalars $a, b, c \in \mathbb{R}$, s.t. 
	\begin{align*}
		B &= 
			\left\lbrace
				v_1 = 
					\begin{pmatrix}
						1\\
						-2\\
						1\\
					\end{pmatrix}
				,
				v_2 = 
					\begin{pmatrix}
						1\\
						1\\
						1\\
					\end{pmatrix}
				,
				v_2 = 
					\begin{pmatrix}
						a\\
						b\\
						c\\
					\end{pmatrix}
			\right\rbrace
	\end{align*}
	is an orthogonal basis of $\mathbb{R}^3$ with scalar product.
\end{example}

\begin{solution}
	\begin{align*}
		\langle v_1, v_2 \rangle &= 0\\
		\langle v_2, v_3 \rangle &= 0\\
		\langle v_3, v_1 \rangle &= 0
	\end{align*}
	Therefore,
	\begin{align*}
			\left\langle
				\begin{pmatrix}
					1\\
					1\\
					1\\
				\end{pmatrix}
				,
				\begin{pmatrix}
					a\\
					b\\
					c\\
				\end{pmatrix}
			\right\rangle
		&= 0\\
			\left\langle
				\begin{pmatrix}
					1\\
					-2\\
					1\\
				\end{pmatrix}
				,
				\begin{pmatrix}
					a\\
					b\\
					c\\
				\end{pmatrix}
			\right\rangle
		&= 0
	\end{align*}
	Therefore,
	\begin{align*}
		a &= -c\\
		b &= 0
	\end{align*}\\
	Therefore,
	\begin{align*}
		v_3 &= 
			\begin{pmatrix}
				a\\
				0\\
				-a\\
			\end{pmatrix}
	\end{align*}
\end{solution}

\begin{example}
	Find an orthonormal basis of the inner product space $V = \vspan\{B\}$ over $\mathbb{C}$.
	\begin{equation*}
		B =
			\left\lbrace
				b_1 = 
					\begin{pmatrix}
						1\\
						i\\
					\end{pmatrix}
				,
				b_2 = 
					\begin{pmatrix}
						2\\
						0\\
					\end{pmatrix}
			\right\rbrace
	\end{equation*}
	\begin{equation*}
		\langle x, y \rangle = x_1 \overline{y_1} + x_2 \overline{y_2}
	\end{equation*}
\end{example}

\begin{solution}
	\begin{align*}
		\widetilde{v_1} &= 
			\begin{pmatrix}
				1\\
				i\\
			\end{pmatrix}\\
		\widetilde{v_2} &= b_2 - \dfrac{\langle b_2, b_1 \rangle}{\| b_1 \|^2} \widetilde{v_1}\\
		&= 
			\begin{pmatrix}
				2\\
				0\\
			\end{pmatrix}
			-
			\dfrac
				{
					\left\langle
						\begin{pmatrix}
							2\\
							0\\
						\end{pmatrix}
						,
						\begin{pmatrix}
							1\\
							i\\
						\end{pmatrix}
					\right\rangle
				}
				{
					\left\|
						\begin{pmatrix}
							1\\
							i\\
						\end{pmatrix}
					\right\|^2
				}
				\begin{pmatrix}
					1\\
					i\\
				\end{pmatrix}\\
		&= 
			\begin{pmatrix}
				2\\
				0\\
			\end{pmatrix}
			-
			\begin{pmatrix}
				1\\
				i\\
			\end{pmatrix}\\
		&= 
			\begin{pmatrix}
				1\\
				-i\\
			\end{pmatrix}\\
		\therefore \widetilde{B} &= 
			\left\lbrace
				\begin{pmatrix}
					1\\
					i\\
				\end{pmatrix}
				,
				\begin{pmatrix}
					1\\
					-i\\
				\end{pmatrix}
			\right\rbrace\\
		\therefore {B}^0 &= 
			\left\lbrace
				\dfrac{1}{\sqrt{2}}
					\begin{pmatrix}
						1\\
						i\\
					\end{pmatrix}
				,
				\dfrac{1}{\sqrt{2}}
					\begin{pmatrix}
						1\\
						-i\\
					\end{pmatrix}
			\right\rbrace
	\end{align*}
\end{solution}

\begin{example}
	Do there exist $a, b, c \in \mathbb{R}$, s.t. $U$ is orthogonal?
	\begin{equation*}
		U = 
			\begin{pmatrix}
				-\dfrac{1}{\sqrt{2}} & -\dfrac{1}{\sqrt{6}} & c\\
				0 & b & \dfrac{1}{\sqrt{3}}\\
				a & -\dfrac{1}{\sqrt{6}} & \dfrac{1}{\sqrt{3}}\\
			\end{pmatrix}
	\end{equation*}
\end{example}

\begin{solution}
	\begin{align*}
		U^t \cdot U &= I\\
		\therefore I &=
			\begin{pmatrix}
				\dfrac{1}{2} + a^2 & \dfrac{1}{\sqrt{12}} - \dfrac{a}{\sqrt{6}} & -\dfrac{c}{\sqrt{2}} + \dfrac{a}{\sqrt{3}}\\
				\dfrac{1}{\sqrt{12}} - \dfrac{a}{\sqrt{6}} & \dfrac{1}{6} + b^2 + \dfrac{1}{6} & -\dfrac{c}{\sqrt{6}} + \dfrac{b}{\sqrt{3}} - \dfrac{1}{\sqrt{18}}\\
				-\dfrac{c}{\sqrt{2}} + \dfrac{a}{\sqrt{3}} & -\dfrac{c}{\sqrt{6}} + \dfrac{b}{\sqrt{3}} - \dfrac{1}{\sqrt{18}} & c^2 + \dfrac{2}{3}\\
			\end{pmatrix}
	\end{align*}
	Therefore,
	\begin{align*}
		\dfrac{1}{2} + a^2 &= 1\\
		\dfrac{1}{3} + b^2 &= 1\\
		\dfrac{2}{3} + c^2 &= 1\\
		\dfrac{1}{\sqrt{12}} - \dfrac{a}{\sqrt{6}} &= 0\\
		-\dfrac{c}{\sqrt{2}} + \dfrac{a}{\sqrt{3}} &= 0\\
		-\dfrac{c}{\sqrt{6}} + \dfrac{b}{\sqrt{3}} - \dfrac{1}{\sqrt{18}} &= 0
	\end{align*}
	Solving,
	\begin{align*}
		a &= \dfrac{1}{\sqrt{2}}\\
		b &= \sqrt{\dfrac{2}{3}}\\
		c &= \dfrac{1}{\sqrt{3}}
	\end{align*}
\end{solution}

\begin{example}
	Find a unitary matrix $U \in M_{3 \times 3}(\mathbb{C})$ that has the vector 
	\begin{equation*}
		b_1 = 
			\begin{pmatrix}
				\dfrac{1 + i}{\sqrt{3}} &  -\dfrac{1}{\sqrt{3}} & 0
			\end{pmatrix}
	\end{equation*}
	as the first row.
\end{example}

\begin{solution}
	Completing to a basis of $\mathbb{C}^3$, let
	\begin{align*}
		b_2 &= (0,1,0)\\
		b_3 &= (0,0,1)
	\end{align*}
	Using Gram - Schmidt Process, the orthonormal basis is
	\begin{align*}
		B^0 &=
			\left\lbrace
				\begin{pmatrix}
					\sfrac{1 + i}{\sqrt{3}}\\
					-\sfrac{1}{\sqrt{3}}\\
					0\\
				\end{pmatrix}
				,
				\begin{pmatrix}
					\sfrac{1 + i}{3}\\
					\sfrac{2}{3}\\
					0\\
				\end{pmatrix}
				,
				\begin{pmatrix}
					0\\
					0\\
					1\\
				\end{pmatrix}
			\right\rbrace
	\end{align*}
\end{solution}

\end{document}