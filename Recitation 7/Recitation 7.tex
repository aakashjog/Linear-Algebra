\documentclass[fleqn, a4paper, 12pt]{article}
\setcounter{secnumdepth}{4}
\usepackage{amsmath, amssymb, amsthm}
\usepackage{mathtools, xfrac}
\usepackage{datetime}
\usepackage{ulem}
\usepackage{enumerate}

\newcommand\numberthis{\addtocounter{equation}{1}\tag{\theequation}}

\newcommand{\rank}{\mathrm{rk\,}}

\newenvironment{amatrix}[1]{%
	\left(\begin{array}{@{}*{#1}{c}|c@{}}
}{%
	\end{array}\right)
}

%\setlength{\mathindent}{0pt}

\theoremstyle{definition}
\newtheorem{example}{Example}
\newtheorem{definition}{Definition}

\theoremstyle{theorem}
\newtheorem{theorem}{Theorem}

\newenvironment{solution}
{\begin{proof}[Solution]\let\qed\relax}
	{\end{proof}}

\DeclareMathOperator{\vspan}{\mathrm{span}} %declares span operator for matrices

%opening
\title{Recitation 7}
\author{}
\date{\formatdate{10}{12}{2014}}

\begin{document}

\maketitle
%\setlength{\mathindent}{0pt}

\tableofcontents

\newpage
\section{Transformation matrices}

\begin{example}
	$B$ and $E$ are bases of $V$, which is the set of all $2 \times 2$ matrices.
	\begin{align*}
		B &=
		\left\lbrace
		\begin{pmatrix}
			1 & 0\\
			0 & 1\\
		\end{pmatrix}
		,
		\begin{pmatrix}
			0 & 1\\
			0 & 0\\
		\end{pmatrix}
		,
		\begin{pmatrix}
			1 & 0\\
			0 & -1\\
		\end{pmatrix}
		,
		\begin{pmatrix}
			0 & 0\\
			1 & 0\\
		\end{pmatrix}
		\right\rbrace\\
		E &=
		\left\lbrace
		\begin{pmatrix}
		1 & 0\\
		0 & 0\\
		\end{pmatrix}
		,
		\begin{pmatrix}
		0 & 1\\
		0 & 0\\
		\end{pmatrix}
		,
		\begin{pmatrix}
		0 & 0\\
		1 & 0\\
		\end{pmatrix}
		,
		\begin{pmatrix}
		0 & 0\\
		0 & 1\\
		\end{pmatrix}
		\right\rbrace
	\end{align*}
	Find $[E]_B$ which holds $[m]_B = [E]_B [m]_E$, $\forall m \in V$.
\end{example}

\begin{solution}
	For ease of calculation, $[E]_B$ can be written as $[B]_E^{-1}$.
	\begin{align*}
		[E]_B &= [B]_E^{-1}\\
		&= 
		\begin{pmatrix}
			\left[
				\begin{pmatrix}
					1 & 0\\
					0 & 1\\
				\end{pmatrix}
			\right]_E
			&
			\left[
				\begin{pmatrix}
					0 & 1\\
					0 & 0\\
				\end{pmatrix}
			\right]_E
			&
			\left[
				\begin{pmatrix}
					1 & 0\\
					0 & -1\\
				\end{pmatrix}
			\right]_E
			&
			\left[
				\begin{pmatrix}
					0 & 0\\
					1 & 0\\
				\end{pmatrix}
			\right]_E
			\\
		\end{pmatrix}^{-1}\\
		&=
		\begin{pmatrix}
			1 & 0 & 1 & 0\\
			0 & 1 & 0 & 0\\
			0 & 0 & 0 & 1\\
			1 & 0 & -1 & 0\\
		\end{pmatrix}^{-1}\\
		&= 
		\begin{pmatrix}
			\sfrac{1}{2} & 0 & 0 & \sfrac{1}{2}\\
			0 & 1 & 0 & 0\\
			\sfrac{1}{2} & 0 & 0 & - \sfrac{1}{2}\\
			0 & 0 & 1 & 0\\
		\end{pmatrix}
	\end{align*}
\end{solution}

\section{Completing to a Basis}

\begin{definition}
	Let $V$ be a $n$ dimensional vector space. Let $W \subset V$ be a subset of $V$ and let its basis $B$ have $m < n$ vectors.\\
	We can find a set $D$ having $n - m$ vectors, s.t. $B \cup D$ will be a basis of $V$.
\end{definition}
The vectors in $D$ will be linearly independent, and $\vspan(B) \cap \vspan(D) = \{0\}$.

\subsection{Method}

In order to find the set $D$, we will write a matrix that has the vectors of $B$ as its rows, and will add row vectors having all zeroes except in the positions corresponding to the free variables, which have 1.

\end{document}
