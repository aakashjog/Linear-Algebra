\date{}
\documentclass[fleqn, a4paper]{amsart}
\usepackage[top=1in, left=1in, right=1in, bottom=1in]{geometry}
\usepackage{amsmath, amssymb, amsthm, thmtools, amsfonts, commath}
\usepackage{datetime}
\usepackage{hyperref}
\usepackage{ulem}
\usepackage{tikz}
\usepackage{enumerate, enumitem}
\usepackage{cancel}
\usepackage{multicol}
\usepackage{xfrac}
%\usepackage{background}
\setcounter{secnumdepth}{4}

%section headings on left
\makeatletter
\def\specialsection{\@startsection{section}{1}%
	\z@{\linespacing\@plus\linespacing}{.5\linespacing}%
	%  {\normalfont\centering}}% DELETED
	{\normalfont}}% NEW
\def\section{\@startsection{section}{1}%
	\z@{.7\linespacing\@plus\linespacing}{.5\linespacing}%
	%  {\normalfont\scshape\centering}}% DELETED
	{\normalfont\scshape}}% NEW
\makeatother

\newcommand\numberthis{\addtocounter{equation}{1}\tag{\theequation}}

\DeclareMathOperator{\vspan}{\mathrm{span}} %declares span operator for matrices

\newenvironment{amatrix}[1]{%declares augmented matrix environment
	\left(\begin{array}{@{}*{#1}{c}|c@{}}
	}{%
\end{array}\right)
} 

\theoremstyle{definition}
\newtheorem{example}{Example} %defines example environment
\newtheorem{definition}{Definition} %defines definition environment

\theoremstyle{theorem}
\newtheorem{theorem}{Theorem} %defines theorem environment
\newtheorem{corollary}{Corollary}

\theoremstyle{remark}
\newtheorem{remark}{Remark}
\newtheorem{case}{Case}

\newcommand{\suchthat}{\mathrm{\,s.t.\,}}

\newcommand{\R}{\mathrm{R}}

\newcommand{\C}{\mathrm{C}}

\newcommand{\rr}{\mathrm{rr}}

\newcommand{\im}{\mathrm{im}\,}

\newenvironment{solution} %declares solution environment and removes qed at end
	{\begin{proof}[Solution]\let\qed\relax}
	{\end{proof}}

\makeatletter
\@addtoreset{section}{part} %resets section numbers in new part
\makeatother

\makeatletter
\@addtoreset{theorem}{part} %resets theorem numbers in new part
\makeatother

\makeatletter
\@addtoreset{corollary}{theorem} %resets corollary numbers after a theorem
\makeatother

\numberwithin{corollary}{theorem}

\numberwithin{equation}{theorem}

\makeatletter %changes spacing between rows of matrices to fit fractions
\newif\ifcenter@asb@\center@asb@false
\def\center@arstrutbox{%
	\setbox\@arstrutbox\hbox{$\vcenter{\box\@arstrutbox}$}%
}
\newcommand*{\CenteredArraystretch}[1]{%
	\ifcenter@asb@\else
	\pretocmd{\@mkpream}{\center@arstrutbox}{}{}%
	\center@asb@true
	\fi
	\renewcommand{\arraystretch}{#1}%
}
\makeatother

\renewcommand{\thesubsubsection}{\roman{subsubsection}}
\renewcommand{\thesubsection}{\alph{subsection}}
\renewcommand{\thesection}{\arabic{section}}

\title{Linear Algebra : Homework 9}
\author{Aakash Jog\\
	ID : 989323563}

\begin{document}

\maketitle

\begin{multicols}{1}

\section{}

\subsection{}

\begin{align*}
	T(f_1(x) + f_2(x)) &= \dod{}{x}(f_1(x) + f_2(x))\\
	&= \dod{}{x} (f_1(x)) + \dod{}{x} (f_2(x))\\
	&= T(f_1(x)) + T(f_2(x))
\end{align*}

\begin{align*}
	T(\alpha f(x)) &= \dod{}{x} (\alpha f(x))\\
	&= \alpha \dod{}{x} f(x)\\
	&= \alpha T(f(x))
\end{align*}
Therefore, $T$ is linear.

\subsection{}

$\{1, x, x^2, x^3\}$ is a basis for $V$.

\subsection{}

$\{1, x, x^2\}$ is a basis for the range of $T$ with input $V$.

\subsection{}

\begin{align*}
	T(1, x, x^2, x^3) &= A (1, x, x^2, x^3)\\
	&= (0, 1, 2x, 3x^2)\\
	\therefore A 
		\begin{pmatrix}
		1 & 0 & 0 & 0\\
		0 & 1 & 0 & 1\\
		0 & 0 & 1 & 0\\
		0 & 0 & 0 & 1\\
		\end{pmatrix}
		&=
		\begin{pmatrix}
			0 & 1 & 0 & 0\\
			0 & 0 & 2 & 0\\
			0 & 0 & 0 & 3\\
			0 & 0 & 0 & 0\\
		\end{pmatrix}\\
	\therefore A &=
		\begin{pmatrix}
			0 & 1 & 0 & 0\\
			0 & 0 & 2 & 0\\
			0 & 0 & 0 & 3\\
			0 & 0 & 0 & 0\\
		\end{pmatrix}
\end{align*}

\subsection{}

\begin{align*}
	T(k) &= 0\\
	\therefore \ker T &= \mathbb{R}
\end{align*}

\subsection{}

\begin{equation*}
	\dim(\text{range}) + \dim(\text{kernel}) = \dim V
\end{equation*}
is valid for all vector spaces $V$ and all linear transformations.

\section{}

\subsection{}

\begin{align*}
	[T]_E &= P [T]_B P^{-1}
\end{align*}

\begin{align*}
	b_1 &= 1 e_1 + 2 e_2\\
	b_2 &= 2 e_1 - 3 e_2
\end{align*}

Therefore,
\begin{align*}
	P &= 
		\begin{pmatrix}
			1 & 2\\
			2 & -3\\
		\end{pmatrix}\\
	\therefore P^{-1} &= 
		\begin{pmatrix}
			\sfrac{3}{7} & \sfrac{2}{7}\\
			\sfrac{2}{7} & -\sfrac{1}{7}\\
		\end{pmatrix}\\
	\therefore [T]_E &= 
		\begin{pmatrix}
			1 & 2\\
			2 & -3\\
		\end{pmatrix}
		\begin{pmatrix}
			2 & 1\\
			-1 & 3\\
		\end{pmatrix}
		\begin{pmatrix}
			\sfrac{3}{7} & \sfrac{2}{7}\\
			\sfrac{2}{7} & -\sfrac{1}{7}\\
		\end{pmatrix}\\
	&= 
		\begin{pmatrix}
			2 & -1\\
			1 & 3\\
		\end{pmatrix}
\end{align*}

\subsection{}

\begin{align*}
	T(x, y) &= 
		\begin{pmatrix}
			2 & -1\\
			1 & 3\\
		\end{pmatrix}
		\begin{pmatrix}
			x\\
			y\\
		\end{pmatrix}\\
	&=
	\begin{pmatrix}
		2x - y\\
		x + 3y\\
	\end{pmatrix}
\end{align*}

\subsection{}

\begin{align*}
	\im (T) &= \vspan 
	\left\lbrace
		\begin{pmatrix}
			2\\
			7\\
		\end{pmatrix}
		,
		\begin{pmatrix}
			-1\\
			3\\
		\end{pmatrix}
	\right\rbrace
\end{align*}

\begin{align*}
	2x - y &= 0\\
	x + 3y &= 0\\
\end{align*}

The matrix is
\begin{align*}
	\begin{pmatrix}
		2 & -1\\
		1 & 3\\
	\end{pmatrix}
	\to
	\begin{pmatrix}
		 1 & 0\\
		 0 & 1\\
	\end{pmatrix}
\end{align*}
Therefore, $\ker T = \{\mathbb{O}\}$.

\section{}

\subsection{}

\begin{gather*}
	T 
		\begin{pmatrix}
			x\\
			y\\
		\end{pmatrix}
	=
		\begin{pmatrix}
			x + y\\
			y\\
		\end{pmatrix}\\
	T : \mathbb{R}^2 \to \mathbb{R}^2\\
	B = B' = 
		\left\lbrace
			\begin{pmatrix}
				1\\
				1\\
			\end{pmatrix}
			,
			\begin{pmatrix}
				0\\
				1\\
			\end{pmatrix}
		\right\rbrace
\end{gather*}

\subsubsection{}

\begin{align*}
	T(e_1) &= 
		\begin{pmatrix}
			1\\
			0\\
		\end{pmatrix}\\
	T(e_2) &= 
		\begin{pmatrix}
			1\\
			1\\
		\end{pmatrix}\\
	\therefore [T]_{B_0, B_0} &= 
		\begin{pmatrix}
			1 & 1\\
			0 & 1\\
		\end{pmatrix}
\end{align*}

\begin{align*}
	b_1 &= 1 e_1 + 1 e_2\\
	b_2 &= 0 e_1 + 1 e_2
\end{align*}

\begin{align*}
	[T]_{B, B'} &= P^{-1} [T]_{B_0, B_0} P
\end{align*}

Therefore,
\begin{align*}
	P &=
		\begin{pmatrix}
			1 & 0\\
			1 & 1\\
		\end{pmatrix}\\
	\therefore P^{-1} &= 
		\begin{pmatrix}
			1 & 0\\
			-1 & 1\\
		\end{pmatrix}\\
	\therefore [T]_{B, B'} &= 
		\begin{pmatrix}
		1 & 0\\
		-1 & 1\\
		\end{pmatrix}
		\begin{pmatrix}
			1 & 1\\
			0 & 1\\
		\end{pmatrix}
		\begin{pmatrix}
			1 & 0\\
			1 & 1\\
		\end{pmatrix}\\
	&= 
		\begin{pmatrix}
			2 & 1\\
			-1 & 0\\
		\end{pmatrix}
\end{align*}

\subsubsection{}

\begin{align*}
		\begin{pmatrix}
			2 & -1\\
			-1 & 0\\
		\end{pmatrix}
		\begin{pmatrix}
			x\\
			y\\
		\end{pmatrix}
	&=
		\begin{pmatrix}
			2x + y\\
			-x\\
		\end{pmatrix}\\
	\therefore \im T & = 
		\left\lbrace
			\begin{pmatrix}
				2x + y\\
				-x\\
			\end{pmatrix}
		\right\rbrace
\end{align*}

\subsubsection{}

\begin{align*}
	\ker T &= \{0\}
\end{align*}

\subsubsection{}

\begin{align*}
	\dim (\im T) &= 2
\end{align*}

\subsubsection{}

\begin{align*}
	\dim (\ker T) &= 0
\end{align*}

\subsubsection{}

$T$ is one-to-one and onto.

\subsection{}

\begin{gather*}
	T 
		\begin{pmatrix}
			x\\
			y\\
			z\\
		\end{pmatrix}
	=
		\begin{pmatrix}
			2x - 4y + 9z\\
			5x + 3y + 2z\\
		\end{pmatrix}\\
	T : \mathbb{R}^3 \to \mathbb{R}^2\\
	B = B' = 
		\left\lbrace
			\begin{pmatrix}
				1\\
				0\\
				0\\
			\end{pmatrix}
			,
			\begin{pmatrix}
				0\\
				1\\
				0\\
			\end{pmatrix}
			,
			\begin{pmatrix}
				0\\
				0\\
				1\\
			\end{pmatrix}
		\right\rbrace
\end{gather*}

\subsubsection{}

\begin{align*}
	T(e_1) &=
		\begin{pmatrix}
			2\\
			5\\
		\end{pmatrix}\\
	T(e_2) &=
		\begin{pmatrix}
			-4\\
			3\\
		\end{pmatrix}\\
	T(e_3) &=
		\begin{pmatrix}
			9\\
			2\\
		\end{pmatrix}\\
	\therefore [T]_B &= 
		\begin{pmatrix}
			2 & -4 & 9\\
			5 & 3 & 2\\
		\end{pmatrix}
\end{align*}

\subsubsection{}

\begin{align*}
	\im T &= \vspan
		\left\lbrace
			\begin{pmatrix}
				2\\
				5\\
			\end{pmatrix}
			,
			\begin{pmatrix}
				-4\\
				3\\
			\end{pmatrix}
			,
			\begin{pmatrix}
			9\\
			2\\
			\end{pmatrix}
		\right\rbrace
\end{align*}

\subsubsection{}

\begin{align*}
	\ker T &= \{2x - 4y + 9z = 5x + 3y + 2z = 0\}
\end{align*}

\subsubsection{}

\begin{align*}
	\dim (\im T) &= 2
\end{align*}

\subsubsection{}

$T$ is onto but not one-to-one as the system of equations for the kernel has more than one solutions.

\subsection{}

\begin{gather*}
	T 
		\begin{pmatrix}
			x\\
			y\\
			z\\
		\end{pmatrix}
	=
		\begin{pmatrix}
			3x + 4y\\
			5x - 2y\\
			x + 7z\\
			4x\\
		\end{pmatrix}\\
	T : \mathbb{R}^3 \to \mathbb{R}^4\\
	B = B' = 
		\left\lbrace
			\begin{pmatrix}
				1\\
				0\\
				0\\
			\end{pmatrix}
		,
			\begin{pmatrix}
				0\\
				1\\
				0\\
			\end{pmatrix}
		,
			\begin{pmatrix}
				0\\
				0\\
				1\\
			\end{pmatrix}
		\right\rbrace
\end{gather*}

\subsubsection{}

\begin{align*}
	T(e_1) &= 
		\begin{pmatrix}
			3\\
			5\\
			1\\
			4\\
		\end{pmatrix}\\
	T(e_2) &= 
		\begin{pmatrix}
			4\\
			-2\\
			0\\
			0\\
		\end{pmatrix}\\
	T(e_3) &= 
		\begin{pmatrix}
			0\\
			0\\
			7\\
			0\\
		\end{pmatrix}\\
	\therefore [T]_{B_0, B_0} &= 
		\begin{pmatrix}
			3 & 4 & 0\\
			5 & -2 & 0\\
			1 & 0 & 7\\
			4 & 0 & 0\\
		\end{pmatrix}
\end{align*}

\subsubsection{}

\begin{align*}
	\im T &= \vspan
		\left\lbrace
			\begin{pmatrix}
				3\\
				5\\
				1\\
				4\\
			\end{pmatrix}
			,
			\begin{pmatrix}
				4\\
				-2\\
				0\\
				0\\
			\end{pmatrix}
			,
			\begin{pmatrix}
				0\\
				0\\
				7\\
				0\\
			\end{pmatrix}
		\right\rbrace
\end{align*}

\subsubsection{}

\begin{align*}
	\ker T &= \{0\}
\end{align*}

\subsubsection{}

\begin{align*}
	\dim (\im T) &= 4
\end{align*}

\subsubsection{}

\begin{align*}
	\dim (\im T) &= 0
\end{align*}

\subsection{}

\begin{gather*}
	T 
		\begin{pmatrix}
			x\\
			y\\
			z\\
			w\\
		\end{pmatrix}
	= 2x + 3y - 7z + w\\
	T : \mathbb{R}^4 \to \mathbb{R}\\
	B = B' = 
		\left\lbrace
			\begin{pmatrix}
				1\\
				0\\
				0\\
			\end{pmatrix}
			,
			\begin{pmatrix}
				0\\
				1\\
				0\\
			\end{pmatrix}
			,
			\begin{pmatrix}
				0\\
				0\\
				1\\
			\end{pmatrix}
		\right\rbrace
\end{gather*}

\subsubsection{}

\begin{align*}
	T(e_1) &= 2\\
	T(e_2) &= 3\\
	T(e_3) &= -7\\
	T(e_4) &= 1\\
	\therefore [T]_B &= 
		\begin{pmatrix}
			2 & 3 & -7 & 1\\
		\end{pmatrix}
\end{align*}

\subsubsection{}

\begin{align*}
	\im T &= \vspan\{2, 3, -7, 1\}
\end{align*}

\subsubsection{}

\begin{align*}
	\ker T &= \{2x + 3y - 7z + w = 0\}
\end{align*}

\subsubsection{}

\begin{align*}
	\dim (\im T) &= 1
\end{align*}

\subsubsection{}

\begin{align*}
	\dim (\ker T) &= 1
\end{align*}

\end{multicols}

\end{document}