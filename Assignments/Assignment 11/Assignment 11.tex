\date{}
\documentclass[fleqn, a4paper, draft]{amsart}
\usepackage[top=1in, left=1in, right=1in, bottom=1in]{geometry}
\usepackage{amsmath, amssymb, amsthm, thmtools, amsfonts, commath}
\usepackage{datetime}
\usepackage{hyperref}
\usepackage{ulem}
\usepackage{tikz}
\usepackage{enumerate, enumitem}
\usepackage{cancel}
\usepackage{multicol}
\usepackage{xfrac}
%\usepackage{background}
\setcounter{secnumdepth}{4}

%section headings on left
\makeatletter
\def\specialsection{\@startsection{section}{1}%
	\z@{\linespacing\@plus\linespacing}{.5\linespacing}%
	%  {\normalfont\centering}}% DELETED
	{\normalfont}}% NEW
\def\section{\@startsection{section}{1}%
	\z@{.7\linespacing\@plus\linespacing}{.5\linespacing}%
	%  {\normalfont\scshape\centering}}% DELETED
	{\normalfont\scshape}}% NEW
\makeatother

\newcommand\numberthis{\addtocounter{equation}{1}\tag{\theequation}}

\DeclareMathOperator{\vspan}{\mathrm{span}} %declares span operator for matrices

\newenvironment{amatrix}[1]{%declares augmented matrix environment
	\left(\begin{array}{@{}*{#1}{c}|c@{}}
	}{%
\end{array}\right)
} 

\theoremstyle{definition}
\newtheorem{example}{Example} %defines example environment
\newtheorem{definition}{Definition} %defines definition environment

\theoremstyle{theorem}
\newtheorem{theorem}{Theorem} %defines theorem environment
\newtheorem{corollary}{Corollary}

\theoremstyle{remark}
\newtheorem{remark}{Remark}
\newtheorem{case}{Case}

\newcommand{\suchthat}{\mathrm{\,s.t.\,}}

\newcommand{\R}{\mathrm{R}}

\newcommand{\C}{\mathrm{C}}

\newcommand{\N}{\mathrm{N}}

\newcommand{\rr}{\mathrm{rr}}

\newcommand{\im}{\mathrm{im}\,}

\newcommand{\trace}{\mathrm{trace\,}}

\newenvironment{solution} %declares solution environment and removes qed at end
	{\begin{proof}[Solution]\let\qed\relax}
	{\end{proof}}

\makeatletter
\@addtoreset{section}{part} %resets section numbers in new part
\makeatother

\makeatletter
\@addtoreset{section}{part} %resets section numbers in new part
\makeatother

\makeatletter
\@addtoreset{theorem}{part} %resets theorem numbers in new part
\makeatother

\makeatletter
\@addtoreset{corollary}{theorem} %resets corollary numbers after a theorem
\makeatother

\numberwithin{corollary}{theorem}

\numberwithin{equation}{theorem}

\makeatletter %changes spacing between rows of matrices to fit fractions
\newif\ifcenter@asb@\center@asb@false
\def\center@arstrutbox{%
	\setbox\@arstrutbox\hbox{$\vcenter{\box\@arstrutbox}$}%
}
\newcommand*{\CenteredArraystretch}[1]{%
	\ifcenter@asb@\else
	\pretocmd{\@mkpream}{\center@arstrutbox}{}{}%
	\center@asb@true
	\fi
	\renewcommand{\arraystretch}{#1}%
}
\makeatother

\renewcommand{\thesubsubsection}{\roman{subsubsection}}
\renewcommand{\thesubsection}{\alph{subsection}}
\renewcommand{\thesection}{\arabic{section}}

\title{Linear Algebra : Homework 11}
\author{Aakash Jog\\
	ID : 989323563}

\begin{document}

\maketitle

%\begin{multicols}{2}

\part{Eigenvalues and Eigenvectors Continued, Diagonalization}

\section{}

\subsection{}

\begin{align*}
	A &=
		\begin{pmatrix}
			7 & -1 & -4\\
			14 & 1 & -12\\
			8 & -1 & -5\\
		\end{pmatrix}\\
	\therefore p_A(\lambda) &= 
		\begin{vmatrix}
			7 - \lambda & -1 & -4\\
			14 & 1 - \lambda & -12\\
			8 & -1 & -5 - \lambda\\
		\end{vmatrix}\\
	&= (7 - \lambda)((1 - \lambda)(-5 - \lambda) - 12) \\
	&\quad+ (14(-5 - \lambda) + 96) \\
	&\quad- 4(-14 - 8(1 - \lambda))\\
	&= -(\lambda + 1)(\lambda^2 - 4 \lambda + 5)\\
	\therefore \lambda &= -1
\end{align*}
Therefore, as $p_A(\lambda)$ does not split completely, $A$ is not diagonalizable.

\begin{align*}
	V_{-1} &= \N
		\begin{pmatrix}
			8 & -1 & -4\\
			14 & 2 & -12\\
			8 & -1 & -4\\
		\end{pmatrix}\\
	&= \N
		\begin{pmatrix}
			8 & -1 & -4\\
			14 & 2 & 12\\
			0 & 0 & 0\\
		\end{pmatrix}\\
	&= \vspan
		\left\lbrace
			\begin{pmatrix}
				2\\
				4\\
				3\\
			\end{pmatrix}
		\right\rbrace
\end{align*}

\subsection{}

\begin{align*}
	p_A(\lambda) &= -(\lambda + 1)(\lambda^2 - 4 \lambda + 5)\\
	&= -(\lambda + 1)(\lambda - 2 + i)(\lambda - 2 - i)\\
	\therefore \lambda &= -1, 2 + i, 2 - i
\end{align*}

\begin{align*}
	V_{-1} &= \vspan
		\left\lbrace 
			\begin{pmatrix}
				2\\
				4\\
				3\\
			\end{pmatrix}
		\right\rbrace\\
	V_{2 + i} &= \N
		\begin{pmatrix}
			5 - i & -1 & -4\\
			14 & -1 - i & -12\\
			8 & -1 & -7 - i\\
		\end{pmatrix}\\
	&= 
		\left\lbrace
			\begin{pmatrix}
				1\\
				1 - i\\
				1\\
			\end{pmatrix}
		\right\rbrace\\
	V_{2 - i} &= \N
		\begin{pmatrix}
			5 + i & -1 & -4\\
			14 & -1 + i & -12\\
			8 & -1 & -7 + i\\
		\end{pmatrix}\\
	&= 
		\left\lbrace
			\begin{pmatrix}
				1\\
				1 + i\\
				1\\
			\end{pmatrix}
		\right\rbrace
\end{align*}
Therefore, for all three eigenvalues, the algebraic and geometric multiplicities are 1. Therefore, $A$ is diagonalizable.

\begin{align*}
	D &= 
		\begin{pmatrix}
			-1 & 0 & 0\\
			0 & 2 + i & 0\\
			0 & 2 - i & 0\\
		\end{pmatrix}\\
	P &= 
		\begin{pmatrix}
			2 & 1 & 1\\
			4 & 1 - i & 1 + i\\
			3 & 1 & 1\\
		\end{pmatrix}
\end{align*}

\section{For the following linear transformation find the eigenvalues and a basis for the eigenspace of each eigenvalue. Determine whether the transformation is diagonalizable.}

\begin{gather*}
	T(x,y,z) = (2x + y, y - z, 2y + 4z)\\
	T : \mathbb{R}^3 \to \mathbb{R}^3
\end{gather*}

\begin{align*}
	P_T (\lambda) &= \left\lvert [T]_E - \lambda I \right\rvert\\
	&= 
		\left\lvert
			\begin{pmatrix}
				[T(1,0,0)]_E & [T(0,1,0)]_E & [T(0,0,1)]_E\\
			\end{pmatrix}
			- \lambda I
		\right\rvert\\
	&= 
		\left\lvert
			\begin{pmatrix}
				[(2,0,0)]_E & [(1,1,2)]_E & [(0,-1,4)]_E\\
			\end{pmatrix}
			- \lambda I
		\right\rvert\\
	&= 
		\left\vert
			\begin{pmatrix}
				2 & 1 & 0\\
				0 & 1 & -1\\
				0 & 2 & 4\\
			\end{pmatrix}
			-
			\begin{pmatrix}
				\lambda & 0 & 0\\
				0 & \lambda & 0\\
				0 & 0 & \lambda\\
			\end{pmatrix}
		\right\rvert\\
	&= 
		\begin{vmatrix}
			2 - \lambda & 1 & 0\\
			0 & 1 - \lambda & -1\\
			0 & 2 & 4 - \lambda\\
		\end{vmatrix}\\
	&= (2 - \lambda) ((1 - \lambda) (4 - \lambda) + 2)\\
	&= (2 - \lambda) (\lambda^2 - 5 \lambda + 6)\\
	&= -(\lambda - 2)^2 (\lambda - 3)\\
	\therefore \lambda &= 2, 3
\end{align*}

\begin{align*}
	V_1 &= \N
		\begin{pmatrix}
			0 & 1 & 0\\
			0 & -1 & -1\\
			0 & 2 & 2\\
		\end{pmatrix}\\
	&= \vspan
		\left\lbrace
			\begin{pmatrix}
				1\\
				0\\
				0\\
			\end{pmatrix}
		\right\rbrace
\end{align*}

\begin{align*}
	V_2 &= \N
		\begin{pmatrix}
			-1 & 1 & 0\\
			0 & -2 & -1\\
			0 & 2 & 1\\
		\end{pmatrix}\\
	&= \N
		\begin{pmatrix}
			-1 & 1 & 0\\
			0 & 2 & 1\\
			0 & 0 & 0\\
		\end{pmatrix}\\
	&= \vspan
		\left\lbrace
			\begin{pmatrix}
				1\\
				1\\
				-2\\
			\end{pmatrix}
		\right\rbrace
\end{align*}
For eigenvalue 2, the geometric multiplicity is not equal to the algebraic multiplicity. Therefore, the transformation is not diagonalizable.

\section{}

\begin{align*}
	\det(A - I) &= 0\\
	A x &= 3 x
\end{align*}
Therefore, 1 and 3 are eigenvalues of $A$.\\
As $\mathrm{rank\,}(A + I) = 2$, there is a zero row in $(A - I)$. Therefore,
\begin{align*}
	\det(A + I) &= 0\\
\end{align*}
Therefore, -1 is an eigenvalue of $A$.

\subsection{}

As
\begin{align*}
	\lambda &= -1, 1, 3
\end{align*}

\begin{align*}
	P_A(\lambda) &= (\lambda + 1) (\lambda - 1) (\lambda - 3)
\end{align*}

\subsection{}

\begin{align*}
	D &= 
		\begin{pmatrix}
			-1 & 0 & 0\\
			0 & 1 & 0\\
			0 & 0 & 3\\
		\end{pmatrix}
\end{align*}
As $A \sim D$, the trace of $A$ is equal to the trace of $D$.\\
Therefore, the trace of $A$ is 3.

\subsection{}

As $A \sim D$, $\det A = \det D$.\\
Therefore,
\begin{align*}
	\det A &= -3
\end{align*}

\subsection{}

As -3 is not an eigenvalue of $A$,
\begin{align*}
	\det(A + 3I) &\neq 0
\end{align*}
Therefore, $(A + 3I)$ is invertible.

\part{Orthogonality}

\section{Which of the following vector spaces is an inner product space?}

\subsection{}

\begin{align*}
	\left\langle
		\begin{pmatrix}
			x_1\\
			x_2\\
		\end{pmatrix}
		,
		\begin{pmatrix}
			x_1\\
			x_2\\
		\end{pmatrix}
	\right\rangle
	&= 8 x_1 x_2
\end{align*}

As $8 x_1 x_2$ may be negative, it is not an inner product space.

\subsection{}

\begin{align*}
	\left\langle
		\begin{pmatrix}
			x_1\\
			x_2\\
		\end{pmatrix}
		,
		\begin{pmatrix}
			x_1\\
			x_2\\
		\end{pmatrix}
	\right\rangle
	&= {x_1}^2 + 7 {x_2}^2\\
	\therefore \langle v_1, v_1 \rangle &\geq 0
\end{align*}
The equality holds if and only if $v_1 = \mathbb{O}$.

\begin{align*}
	\langle v_1 + v_2, v_3 \rangle &= (x_1 + y_1) z_1 + 7 (x_2 + y_2) z_2\\
	&= x_1 z_1 + y_1 z_1 + 7 x_2 z_2 + 7 y_2 z_2\\
	&= \langle v_1, v_3 \rangle + \langle v_2, v_3 \rangle\\
\end{align*}

\begin{align*}
	\langle \alpha v_1, v_2 \rangle &= \alpha x_1 y_1 + 7 \alpha x_2 y_2\\
	&= 7 \langle v_1, v_2 \rangle
\end{align*}

\begin{align*}
	\overline{\langle v_2, v_1 \rangle} &= x_1 y_1 + 7 x_2 y_2\\
	&= \langle v_1, v_2 \rangle
\end{align*}
Therefore, it is an inner product space.

\subsection{}

Let
\begin{align*}
	A &=
		\begin{pmatrix}
			x_{11} & x_{12}\\
			x_{21} & x_{22}\\
		\end{pmatrix}\\
	B &=
		\begin{pmatrix}
			y_{11} & y_{12}\\
			y_{21} & y_{22}\\
		\end{pmatrix}\\
	\therefore AB &= 
		\begin{pmatrix}
			x_{11} y_{11} + x_{12} y_{21} & x_{11} y_{12} + x_{12} y_{22}\\
			x_{21} y_{11} + x_{22} y_{21} & x_{21} y_{12} + x_{22} y_{22}\\
		\end{pmatrix}\\
	\therefore \trace(AB) &= x_{11} y_{11} + x_{12} y_{21} + x_{21} y_{12} + x_{22} y_{22}
\end{align*}

\begin{align*}
	\langle A, B \rangle &= \trace(AB)\\
	&= x_{11} y_{11} + x_{12} y_{21} + x_{21} y_{12} + x_{22} y_{22}
\end{align*}
Therefore,
\begin{align*}
	\langle A, A \rangle &= {x_{11}}^2 + 2 x_{12} x_{21} + {x_{22}}^2
\end{align*}
$\trace(A^2)$ may be negative. Therefore, it is not an inner product space.

%\end{multicols}

\end{document}