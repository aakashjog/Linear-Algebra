\documentclass[fleqn, a4paper, 12pt]{article}
\setcounter{secnumdepth}{4}
\usepackage{amsmath, amssymb, amsthm}
\usepackage{mathtools, xfrac}
\usepackage{datetime}
\usepackage{ulem}
\usepackage{enumerate}

\newcommand\numberthis{\addtocounter{equation}{1}\tag{\theequation}}

\newcommand{\rank}{\mathrm{rk\,}}

\newcommand{\R}{\mathrm{R}}

\newcommand{\C}{\mathrm{C}}

\newenvironment{amatrix}[1]{%
	\left(\begin{array}{@{}*{#1}{c}|c@{}}
}{%
	\end{array}\right)
}

%\setlength{\mathindent}{0pt}

\theoremstyle{definition}
\newtheorem{example}{Example}
\newtheorem{definition}{Definition}

\theoremstyle{theorem}
\newtheorem{theorem}{Theorem}

\newenvironment{solution}
{\begin{proof}[Solution]\let\qed\relax}
	{\end{proof}}

\DeclareMathOperator{\vspan}{\mathrm{span}} %declares span operator for matrices

%opening
\title{Recitation 8}
\author{}
\date{\formatdate{17}{12}{2014}}

\begin{document}

\maketitle
%\setlength{\mathindent}{0pt}

\tableofcontents

\newpage
\section{Linear Maps}

\begin{example}
	Is the following map linear?
	\begin{align*}
		T &: \mathbb{R}^3 \to \mathbb{R}^2 &;\quad T((x, y, z)) = (x+2y, 2x+z)
	\end{align*}
\end{example}

\begin{solution}
	\begin{align*}
		T((x_1, y_1, z_1) + (x_2, y_2, z_2)) &= T((x_1 + x_2, y_1 + y_2, z_1 + z_2))\\
		&= (x_1 + x_2 + 2y_1 + 2y_2, 2x_1 + 2x_2, z_1, z_2)\\
		&= (x_1 + 2y_1, 2x_1 + z_1) + (x_2 + 2y_2, 2x_2 + z_2)\\
		&= T((x_1, y_1, z_1)) + T((x_2, y_2, z_2))
	\end{align*}
	\begin{align*}
		T(\alpha (x, y, z)) &= T((\alpha x, \alpha y, \alpha z))\\
		&= (\alpha x + 2 \alpha y, 2 \alpha x + \alpha z)\\
		&= \alpha T((x, y, z))
	\end{align*}
\end{solution}

\begin{example}
	Is the following map linear?
	\begin{align*}
		T : \mathbb{R}^3 \to \mathbb{R}^2 \quad; T((x, y, z)) = (xy, x^2)
	\end{align*}
\end{example}

\begin{solution}
	\begin{align*}
		T(-1(0, 0, 1)) &= (0,1)\\
		-1 \cdot T((0, 0, 1)) &= (0, -1)\\
		\therefore T(-1(0, 0, 1)) &\neq -1 \cdot T((0, 0, 1))
	\end{align*}
	Hence the map is not linear.
\end{solution}

\begin{theorem}
	If $\{u_1, \dots, u_n\}$ is a basis of $U$, we can define a linear map $T : U \to V$ using the images of the elements of the basis, i.e. $\{T(u_1), \dots, T(u_n)\}$.
\end{theorem}

\begin{example}
	Find, a linear map $T$, if it exists, such that is satisfies the given conditions.
	\begin{align*}
		T &: \mathbb{R}^3 \to \mathbb{R}^3\\
		T
		\left(
			\begin{pmatrix}
				1\\
				1\\
				1\\
			\end{pmatrix}
		\right)
		&=
		\begin{pmatrix}
			3\\
			0\\
			0\\
		\end{pmatrix}\\
		T
		\left(
			\begin{pmatrix}
			2\\
			-1\\
			2\\
			\end{pmatrix}
		\right)
		&=
		\begin{pmatrix}
			0\\
			0\\
			0\\
		\end{pmatrix}\\
		T
		\left(
			\begin{pmatrix}
				1\\
				0\\
				0\\
			\end{pmatrix}
		\right)
		&=
		\begin{pmatrix}
			1\\
			0\\
			0\\
		\end{pmatrix}\\
	\end{align*}
\end{example}

\begin{solution}
	\begin{align*}
		\left\lbrace
			u_1 =
			\begin{pmatrix}
				1\\
				1\\
				1\\
			\end{pmatrix}
			,
			u_2 =
			\begin{pmatrix}
				2\\
				-1\\
				2\\
			\end{pmatrix}
			,
			u_3 =
			\begin{pmatrix}
				1\\
				0\\
				0\\
			\end{pmatrix}
		\right\rbrace
	\end{align*}
	is linearly independent, and $\dim \mathbb{R}^3 = 3$. Therefore, the above set is a basis of $\mathbb{R}^3$.\\
	Therefore, $T$ exists.
	\begin{align*}
		T
		\left(
			\begin{pmatrix}
				1\\
				0\\
				0\\
			\end{pmatrix}
		\right)
		&= 
		\begin{pmatrix}
			1\\
			0\\
			0\\
		\end{pmatrix}\\
		T
		\left(
			\begin{pmatrix}
				0\\
				1\\
				0\\
			\end{pmatrix}
		\right)
		&= 
		\begin{pmatrix}
			2\\
			0\\
			0\\
		\end{pmatrix}\\
		T
		\left(
			\begin{pmatrix}
				0\\
				1\\
				0\\
			\end{pmatrix}
		\right)
		&= 
		\begin{pmatrix}
			0\\
			0\\
			0\\
		\end{pmatrix}\\
		\therefore T
		\left(
			\begin{pmatrix}
				x\\
				y\\
				z\\
			\end{pmatrix}
		\right)
		&= T(x e_1 + y e_2 + z e_3)\\
		&= 
		\begin{pmatrix}
			x + 2y\\
			0\\
			0\\
		\end{pmatrix}
	\end{align*}
\end{solution}

\end{document}
