\documentclass[fleqn, a4paper, 12pt]{article}
\setcounter{secnumdepth}{4}
\usepackage{amsmath, amssymb, amsthm}
\usepackage{mathtools, xfrac}
\usepackage{datetime}
\usepackage{ulem}
\usepackage{enumerate}
\usepackage[table]{xcolor}

\newcommand\numberthis{\addtocounter{equation}{1}\tag{\theequation}}

\newcommand{\rank}{\mathrm{rk\,}}

\newcommand{\R}{\mathrm{R}}

\newcommand{\C}{\mathrm{C}}

\newcommand{\N}{\mathrm{N}}

\newcommand{\im}{\mathrm{im}\,}

\newenvironment{amatrix}[1]{%
	\left(\begin{array}{@{}*{#1}{c}|c@{}}
}{%
	\end{array}\right)
}

%\setlength{\mathindent}{0pt}

\theoremstyle{definition}
\newtheorem{example}{Example}
\newtheorem{definition}{Definition}

\theoremstyle{theorem}
\newtheorem{theorem}{Theorem}

\newenvironment{solution}
{\begin{proof}[Solution]\let\qed\relax}
	{\end{proof}}

\DeclareMathOperator{\vspan}{\mathrm{span}} %declares span operator for matrices

%opening
\title{Recitation 12}
\author{}
\date{\formatdate{7}{1}{2015}}

\begin{document}

\maketitle
%\setlength{\mathindent}{0pt}

\tableofcontents

\newpage
\section{Linear Transformation Diagonalization}

\begin{example}
	Let $V = M_{2 \times 2} (\mathbb{R})$ and $T : V \to V$.
	\begin{equation*}
	T 
		\begin{pmatrix}
			a & b\\
			c & d\\
		\end{pmatrix}
	=
		\begin{pmatrix}
			d & a - 2b\\
			c & a\\
		\end{pmatrix}
	\end{equation*}
	Find $[T]_E$, all eigenvalues, the eigenspace of the maximal eigenvalue. Determine whether $T$ is diagonalizable. 
\end{example}

\begin{solution}
	\begin{align*}
		[T]_E &= 
			\begin{pmatrix}
				[T(e_1)]_E & [T(e_2)]_E & [T(e_3)]_E & [T(e_4)]_E\\
			\end{pmatrix}\\
		&= 
			\begin{pmatrix}
				\left[
					\begin{pmatrix}
						0 & 1\\
						0 & 1\\
					\end{pmatrix}
				\right]_E
			&
				\left[
					\begin{pmatrix}
						0 & -2\\
						0 & 0\\
					\end{pmatrix}
				\right]_E
			&
				\left[
					\begin{pmatrix}
						0 & 0\\
						1 & 0\\
					\end{pmatrix}
				\right]_E
			&
				\left[
					\begin{pmatrix}
						1 & 1\\
						0 & 0\\
					\end{pmatrix}
				\right]_E
			\end{pmatrix}\\
		&=
			\begin{pmatrix}
				0 & 0 & 0 & 1\\
				1 & -2 & 0 & 0\\
				0 & 0 & 1 & 0\\
				1 & 0 & 0 & 0\\
			\end{pmatrix}
	\end{align*}
	\begin{align*}
		P_T (\lambda) = \left\lvert [T]_E - \lambda I \right\rvert\\
		&= 
			\begin{vmatrix}
				-\lambda & 0 & 0 & 1\\
				1 & -2 - \lambda & 0 & 0\\
				0 & 0 & 1 - \lambda & 0\\
				1 & 0 & 0 & -\lambda\\
			\end{vmatrix}\\
		&= (1 - \lambda) 
			\begin{pmatrix}
				-\lambda & 0 & 1\\
				1 & -2 -\lambda & 0\\
				1 & 0 & \lambda\\
			\end{pmatrix}\\
		&= - (1 - \lambda) (2 + \lambda) (\lambda^2 - 1)\\
		\therefore \lambda &= 1, -1, -2
	\end{align*}
	\begin{align*}
		V_1 &= \N 
			\begin{pmatrix}
				-1 & 0 & 0 & 1\\
				1 & -3 & 0 & 0\\
				0 & 0 & 0 & 0\\
				1 & 0 & 0 & -1\\
			\end{pmatrix}\\
		&= \N
			\begin{pmatrix}
				-1 & 0 & 0 & 1\\
				0 & -3 & 0 & 1\\
				0 & 0 & 0 & 0\\
				0 & 0 & 0 & 0\\
			\end{pmatrix}\\
		&= 
			\begin{pmatrix}
				w\\
				\sfrac{w}{3}\\
				z\\
				w\\
			\end{pmatrix}\\
		&= \vspan
			\left\lbrace
				\begin{pmatrix}
					0\\
					0\\
					1\\
					0\\
				\end{pmatrix}
				,
				\begin{pmatrix}
					3\\
					1\\
					0\\
					3\\
				\end{pmatrix}
			\right\rbrace
	\end{align*}
	For each eigenvalue, the algebraic multiplicity is equal to the geometeric multiplicity. Therefore $T$ is diagonalizable.
\end{solution}

\begin{example}
	Let $A \in M_{2 \times 2} (\mathbb{R})$, s.t.
	\begin{align*}
		A
			\begin{pmatrix}
				1\\
				3\\
			\end{pmatrix}
		&= 
			\begin{pmatrix}
				-2\\
				-6\\
			\end{pmatrix}\\
		A
			\begin{pmatrix}
				2\\
				5\\
			\end{pmatrix}
		&= 
			\begin{pmatrix}
				2\\
				5\\
			\end{pmatrix}
	\end{align*}
	Find $A$ and the eigenvalues and eigenspaces of $A^{-1}$.
\end{example}

\begin{solution}
	\begin{align*}
		A v_1 &= -2 v_1
	\end{align*}
	Therefore, -2 is an eigenvalue of $A$, and $v_1$ is an eigenvector of $A$, corresponding to -2.
	\begin{align*}
		A v_2 &= 1 v_2
	\end{align*}
	Therefore, 1 is an eigenvalue of $A$, and $v_2$ is an eigenvector of $A$, corresponding to 1.\\
	Therefore, the characteristic polynomial is
	\begin{align*}
		p_A (\lambda) &= (\lambda - 1) (\lambda + 2)
	\end{align*}
	Therefore,
	\begin{align*}
		V_{-2} &= \vspan
			\left\lbrace
				\begin{pmatrix}
					1\\
					3\\
				\end{pmatrix}
			\right\rbrace\\
		V_{1} &= \vspan
			\left\lbrace
				\begin{pmatrix}
					2\\
					5\\
				\end{pmatrix}
			\right\rbrace
	\end{align*}
	Therefore, $A$ is diagonalizable, and its diagonal form is
	\begin{align*}
		D &= 
			\begin{pmatrix}
				-2 & 0\\
				0 & 1\\
			\end{pmatrix}
	\end{align*}
	and the corresponding $P$ is
	\begin{align*}
		\begin{pmatrix}
			1 & 2\\
			3 & 5\\
		\end{pmatrix}
	\end{align*}
	\begin{align*}
		A &= P D P^{-1}\\
		\therefore A &= 
			\begin{pmatrix}
				1 & 2\\
				3 & 5\\
			\end{pmatrix}
			\begin{pmatrix}
				-2 & 0\\
				0 & 1\\
			\end{pmatrix}
			\begin{pmatrix}
				-5 & 2\\
				3 & -1\\
			\end{pmatrix}\\
		&=
			\begin{pmatrix}
				16 & -6\\
				45 & -17\\
			\end{pmatrix}
	\end{align*}
	If $\lambda$ is an eigenvalue of an invertible matrix $A$, then $\lambda \neq 0$ and eigenvector $v$ of $A$, corresponding to $\lambda$ is also an eigenvector of $A^{-1}$ corresponding to $\lambda_{-1}$.\\
	Therefore,
	\begin{tabular}{|c|c|}
		\hline
		Eigenvalue & Eigenvector\\
		\hline
		$-\dfrac{1}{2}$ & 
			$
				\begin{pmatrix}
					1\\
					3\\
				\end{pmatrix}
			$\\
		\hline
		$1$ & 
			$
				\begin{pmatrix}
					2\\
					5\\
				\end{pmatrix}
			$\\
		\hline
	\end{tabular}
\end{solution}

\section{Inner Product Spaces}

\begin{example}
	Determine whether the following is an inner product.
	\begin{align*}
		V &= \mathbb{R}^2\\
		\langle x, y \rangle &= x_1 y_1 - x_1 y_2 - x_2 y_1 + 3 x_2 y_2
	\end{align*}
\end{example}

\begin{solution}
	Let
	\begin{align*}
		x &= 
			\begin{pmatrix}
				x_1\\
				x_2\\
			\end{pmatrix}\\
		y &= 
			\begin{pmatrix}
				y_1\\
				y_2\\
			\end{pmatrix}\\
		z &= 
			\begin{pmatrix}
				z_1\\
				z_2\\
			\end{pmatrix}
	\end{align*}
	Therefore,
	\begin{align*}
		\langle x, x \rangle &= 
			\left\langle
				\begin{pmatrix}
					x_1\\
					x_2\\
				\end{pmatrix}
				,
				\begin{pmatrix}
					x_1\\
					x_2\\
				\end{pmatrix}
			\right\rangle
		&= {x_1}^2 - x_1 x_2 - x_2 x_1 + 3 {x_2}^2\\
		&= {x_1}^2 - 2 x_1 x_2 + {x_2}^2 + 2 {x_2}^2\\
		&= (x_1 - x_2)^2 + 2 {x_2}^2\\
		&\geq 0
	\end{align*}
	\begin{align*}
		\langle x, x \rangle = 0 \iff x_1 = x_2 = 0
	\end{align*}
	\begin{align*}
		\langle x + y, z \rangle &= (x_1 + y_1) z_1 - (x_1 + y_1) z_2 - (x_2 + y_2) z_2 + 3 (x_2 + y_2) z_2\\
		&= \langle x, z \rangle + \langle y, z \rangle
	\end{align*}
	\begin{align*}
		\langle \alpha x, y \rangle &= \alpha \langle x, y \rangle
	\end{align*}
	\begin{align*}
		\langle x, y \rangle &= \langle y, x \rangle
	\end{align*}
	Therefore, it is an inner product.
\end{solution}

\end{document}