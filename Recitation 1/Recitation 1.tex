\documentclass[fleqn]{article}
\setcounter{secnumdepth}{4}
\usepackage{amsmath, amssymb}
\usepackage{datetime}
\usepackage{ulem}
\usepackage{enumerate}
\newcommand\numberthis{\addtocounter{equation}{1}\tag{\theequation}}


\setlength{\mathindent}{0pt}

%opening
\title{Recitation 1}
\author{}
\date{\formatdate{29}{10}{2014}}

\begin{document}

\maketitle
\setlength{\mathindent}{0pt}

\tableofcontents

\newpage
\section{General Information}

\textbf{Zahi(Tzahi) Hazan}\\
zahihaza@post.tau.ac.il\\
Reception Hour: Sunday, \formattime{11}{00}{10} - \formattime{11}{00}{00}\\

\newpage
\section{Complex Numbers}

Let us set $i = \sqrt{-1}$. \\
A complex number is a number of the form $z = a + ib$, where $a, b \in \mathbb{R}$.\\
$\Re(z) = Re(z) = a$\\
$\Im(z) = Im(z) = b$\\

\subsection{Addition}

\[(a + ib) + (c + id) = (a + c) + i(b + d)\]

\subsection{Multiplication}

\[(a + ib)(c + id) = ac + ibc + iad - bd = (ac - bd) + i(bc + ad)\]

\subsection{Complex Conjugate}

For $z = a + ib$, its conjugate is $\overline{z} = a - ib$\\
$z + \overline{z} = (a + ib) + (a - ib) = 2a = 2\Re(z)$\\
$z - \overline{z} = (a + ib) - (a - ib) = 2ib = 2\Im(z)$\\
$z \cdot \overline{z} = (a + ib) (a - ib) = a^2 + b^2$\\

\subsection{Absolute Value}

$\lvert z \rvert = \sqrt{z \overline{z}} = \sqrt{\Re^2(z) + \Im^2(z)}$

\subsection{Complex Division}

$\dfrac{z}{w} = \dfrac{z}{w} \cdot \dfrac{\overline{z}}{\overline{w}} = \dfrac{z \cdot \overline{w}}{\lvert w \rvert ^2}$

\subsection{Examples}

\subsubsection{Example 1}

\paragraph{Express $\dfrac{(2+i)(4+i)}{(1+i)}$ as a complex number.\\}
$(2 + i)(4 + i) = 8 + 6i -1 = 7 + 6i$\\
$\therefore \dfrac{(2+i)(4+i)}{(1+i)} = \dfrac{7 + 6i}{1 + i} = \dfrac{7 + 6i}{1 + i} \cdot \dfrac{1 - i}{1 + i} = \dfrac{7 + 6 - i}{2} = \dfrac{13}{2} - \dfrac{i}{2}$

\paragraph{Show that $i^{77} = i$\\}
$i^1 = i, i^2 = -1, i^3 = -i, i^4 = 1$\\
$n = 4m + k$\\
$\therefore i^n = i^{4m+k} = (i^{4m}) i^k = (i^4)^m i^k = i^k$\\
$\therefore i^{77} = i^{4 \cdot 19 + 1} = i$

\subsection{Complex Plane}

For every complex number, there is a corresponding point in the complex plane. This plane is called \emph{Complex plane} or \emph{Gauss plane}.\\
The $x$-axis represents the real part and the $y$-axis represents the imaginary part.\\

\begin{align*}
\lvert z \rvert = \sqrt{a^2 + b^2} =  \text{the distance from} O(0,0) \nonumber \\
z = r \cos \theta + r \sin \theta = r(\cos \theta + i \sin \theta)\\
\\r = \lvert z \rvert = \sqrt{a^2 + b^2}\\
\theta = \arctan \dfrac{b}{a}\\
z = r(\cos \theta + i \sin \theta) = r \text{cis} \theta = r e^{i \theta}\\
z_1 z_2 = (r_1 e^{i \theta_1})(r_2 e^{i \theta_2}) = r_1 r_2 e^{i(\theta_1 + \theta_2)}
\end{align*}

\subsection{De Moivre's Formula}

\begin{equation*}
	(r e^{i \theta})^n = r^n e^{i n \theta}
\end{equation*}

\newpage
\section{Matrices}

A matrix of order $m \times n$, over a field $\mathbb{F}$ is a table with $n$ rows and $m$ columns.

\begin{equation*}
A =
\begin{matrix}

a_{11} & a_{12} & \dots & a_{1n}\\	
a_{21} & a_{22} & \dots & a_{2n}\\
\vdots & \vdots & & \vdots\\
a_{m1} & a_{m2} & \dots & a_{mn}\\
\end{matrix}
= (a_{ij})
\end{equation*}

\subsection{Row Vector}

\emph{Row vector} is a matrix of order $1 \times m$\\

\subsection{Column Vector}

\emph{Column vector} is a matrix of order $n \times 1$\\

\subsection{Multiplying a Matrix by a Scalar}

\begin{align*}
	\alpha A = (\alpha a_{ij})_{ij}
\end{align*}

\subsection{Addition of Matrices}

Addition of matrices is defined only if the matrices are of same order.
\begin{align*}
	A + B = (a_{ij} + b_{ij})
\end{align*}

\subsection{Multiplication of Matrices}

Multiplication of matrices is defined for only matrices of the type $A_{n \times m}$ and $B_{m \times p}$.\\
\begin{align*}
	C_{n \times p} = A_{n \times m} \cdot B_{m \times p}\\
	c_{ik} = \sum_{j = 1}^{n} a_{ij} b_{jk}
\end{align*}

\end{document}
