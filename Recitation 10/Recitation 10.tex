\documentclass[fleqn, a4paper, 12pt]{article}
\setcounter{secnumdepth}{4}
\usepackage{amsmath, amssymb, amsthm}
\usepackage{mathtools, xfrac}
\usepackage{datetime}
\usepackage{ulem}
\usepackage{enumerate}

\newcommand\numberthis{\addtocounter{equation}{1}\tag{\theequation}}

\newcommand{\rank}{\mathrm{rk\,}}

\newcommand{\R}{\mathrm{R}}

\newcommand{\C}{\mathrm{C}}

\newcommand{\N}{\mathrm{N}}

\newcommand{\im}{\mathrm{im}\,}

\newenvironment{amatrix}[1]{%
	\left(\begin{array}{@{}*{#1}{c}|c@{}}
}{%
	\end{array}\right)
}

%\setlength{\mathindent}{0pt}

\theoremstyle{definition}
\newtheorem{example}{Example}
\newtheorem{definition}{Definition}

\theoremstyle{theorem}
\newtheorem{theorem}{Theorem}

\newenvironment{solution}
{\begin{proof}[Solution]\let\qed\relax}
	{\end{proof}}

\DeclareMathOperator{\vspan}{\mathrm{span}} %declares span operator for matrices

%opening
\title{Recitation 10}
\author{}
\date{\formatdate{31}{12}{2014}}

\begin{document}

\maketitle
%\setlength{\mathindent}{0pt}

\tableofcontents

\newpage
\section{Eigenvalues, Eigenvectors and Eigenspaces}

\begin{enumerate}
	\item Eigenvectors corresponding to different eigenvlues are linearly independent.
	\item If $v$ is an eigenvector of $A$ corresponding to $\lambda$, then $v$ is also an eigenvector of $A^k$ corresponding to $\lambda^k$.
	\item If $v$ is an eigenvector of $A$ corresponding to eigenvalue $\lambda$ and $p(x) = \sum_{j = 0}^{k} a_j x^j$, then $v$ is also an eigenvector of $p(A)$ corresponding to the eigenvalue $p(\lambda)$.
	\item If $A$ is invertible, then 0 is not an eigenvalue of $A$.
	\item For all eigenvectors $v$ corresponding to the eigenvalue $\lambda$ of $A$, $v$ is also an eigenvector of $A$, corresponding to the eigenvalue $\lambda^{-1}$.
\end{enumerate}

\begin{example}
	Find eigenvalues and eigenspaces of 
	\begin{align*}
		A &=
			\begin{pmatrix}
				2 & -1\\
				-1 & 2\\
			\end{pmatrix}
	\end{align*}
\end{example}

\begin{solution}
	\begin{align*}
		P_A (\lambda) &= \det 
			\begin{pmatrix}
				2 - \lambda & -1\\
				-1 & 2 - \lambda\\
			\end{pmatrix}\\
		&= (2 - \lambda)^2 - 1\\
		&= \lambda^2 - 4 \lambda + 3\\
		&= (\lambda - 1) (\lambda - 3)
	\end{align*}
	Therefore, 1 and 4 are the eigenvalues of $A$.
	\begin{align*}
		V_1 &= \N (A - 1 \cdot I)\\
		&= \N 
			\begin{pmatrix}
				1 & -1\\
				-1 & 1\\
			\end{pmatrix}\\
		\therefore V_1 &=
			\left\lbrace
				\begin{pmatrix}
					x\\
					x\\
				\end{pmatrix}
			\right\rbrace\\
		&= \vspan
			\left\lbrace
				\begin{pmatrix}
					1\\
					1\\
				\end{pmatrix}
			\right\rbrace
	\end{align*}
	\begin{align*}
		V_3 &= \N (A - 3 \cdot I)\\
		&= \N
			\begin{pmatrix}
				-1 & -1\\
				-1 & -1\\
			\end{pmatrix}\\
		\therefore V_3 &= 
			\left\lbrace
				\begin{pmatrix}
					x\\
					-x\\
				\end{pmatrix}
			\right\rbrace\\
		&= \vspan
			\left\lbrace
				\begin{pmatrix}
					1\\
					-1\\
				\end{pmatrix}
			\right\rbrace
	\end{align*}
\end{solution}

\begin{example}
	Find eigenvalues and eigenspaces of 
	\begin{align*}
		B &=
			\begin{pmatrix}
				0 & 0 & 0\\
				0 & 0 & -2\\
				0 & 1 & 3\\
			\end{pmatrix}
	\end{align*}
\end{example}

\begin{solution}
	\begin{align*}
		P_B (\lambda) &= \det (A - \lambda I)\\
		&= 
			\begin{vmatrix}
				\lambda & 0 & 0\\
				0 & -\lambda & -2\\
				0 & 1 & 3 - \lambda\\
			\end{vmatrix}\\
		&= \lambda(\lambda^2 - 3 \lambda + 2)\\
		&= \lambda^3 - 3 \lambda^2 + 2\lambda\\
		&= \lambda (\lambda - 1) (\lambda - 2)
	\end{align*}
	Therefore, 0, 1, 2 are eigenvalues of $B$.
	\begin{align*}
		V_0 &= \N (B - 0 \cdot I)\\
		&= \N 
			\begin{pmatrix}
				0 & 0 & 0\\
				0 & 0 & -2\\
				0 & 1 & 3\\
			\end{pmatrix}\\
		&= 
			\left\lbrace
				\begin{pmatrix}
					x\\
					0\\
					0\\
				\end{pmatrix}
			\right\rbrace\\
		&= 
			\left\lbrace
				\begin{pmatrix}
					1\\
					0\\
					0\\
				\end{pmatrix}
			\right\rbrace
	\end{align*}
	\begin{align*}
		V_1 &= \N (B - 1 \cdot I)\\
		&= \N 
			\begin{pmatrix}
				-1 & 0 & 0\\
				0 & -1 & -2\\
				0 & 1 & 2\\
			\end{pmatrix}\\
		&= \N
			\begin{pmatrix}
				-1 & 0 & 0\\
				0 & 1 & 2\\
				0 & 0 & 0\\
			\end{pmatrix}\\
		&= 
			\left\lbrace
				\begin{pmatrix}
					0\\
					-2z\\
					z\\
				\end{pmatrix}
			\right\rbrace\\
		&= \vspan
			\left\lbrace
				\begin{pmatrix}
					0\\
					-2\\
					1\\
				\end{pmatrix}
			\right\rbrace
	\end{align*}
	\begin{align*}
		V_2 &= \N (B - 2 \cdot I)\\
		&= \N
			\begin{pmatrix}
				-2 & 0 & 0\\
				0 & -2 & -2\\
				0 & 1 & 1\\
			\end{pmatrix}\\
		&= 
			\left\lbrace
				\begin{pmatrix}
					0\\
					-z\\
					z\\
				\end{pmatrix}
			\right\rbrace\\
		&= \vspan
			\left\lbrace
				\begin{pmatrix}
					0\\
					-1\\
					1\\
				\end{pmatrix}
			\right\rbrace
	\end{align*}
\end{solution}

\begin{example}
	Let $A$ be a square matrix of order 3 with the eigenvalues 
	\begin{align*}
		\lambda_1 &= 0\\
		\lambda_2 &= 1\\
		\lambda_3 &= 2
	\end{align*}
	Let $v_1$, $v_2$, $v_3$ be the corresponding eigenvectors.
	\begin{enumerate}
		\item Find a basis for $N(A)$.
		\item Find a solution for $A v = v_2 + v_3$.
	\end{enumerate}
\end{example}

\begin{solution}
	As $v_1$, $v_2$, $v_3$ are eigenvectors corresponding to different eigenvalues, they are linearly independent. Hence they form a basis of $\mathbb{R}^3.$\\
	Let $v \in \N (A)$.\\
	Therefore,
	\begin{align*}
		v &= \alpha v_1 + \beta v_2 + \gamma v_3\\
		\therefore A v &= A(\alpha v_1 + \beta v_2 + \gamma v_3)\\
		\intertext{As $v \in \N (A)$, $ A v = 0$}
		\therefore 0 &= A(\alpha v_1 + \beta v_2 + \gamma v_3)\\
		&= \alpha A v_1 + \beta A v_2 + \gamma A v_3\\
		\intertext{As $A v_1 = \lambda_1 v_1$,}
		0 &= \alpha \lambda_1 v_1 + \beta \lambda_2 v_2 + \gamma \lambda_3 v_3\\
		&= \beta v_2 + 2 \gamma v_3\\
		\intertext{As $v_2$ and $v_3$ are linearly independent, $\beta = \gamma = 0$}
		\therefore v &= \alpha v_1\\
		\therefore \N (A) &= \vspan \{v_1\}
	\end{align*}
	
	\begin{align*}
		v &= \alpha v_1 + \beta v_2 + \gamma v_3\\
		A v &= v_2 + v_3\\
		\therefore \alpha A v_1 + \beta A v_2 + \gamma A v_3 &= v_2 + v_3\\
		\therefore (\beta - 1) v_2 + (2 \gamma - 1) v_3 &= 0
	\end{align*}
	Therefore,
	\begin{align*}
		\beta &= 1\\
		\gamma &= \dfrac{1}{2}
	\end{align*}
	Therefore,
	\begin{align*}
		v &= v_2 + \dfrac{1}{2} v_3
	\end{align*}
\end{solution}

\end{document}
