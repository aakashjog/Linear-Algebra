\documentclass[fleqn, a4paper, 12pt]{article}
\setcounter{secnumdepth}{4}
\usepackage{amsmath, amssymb, amsthm, commath}
\usepackage{mathtools, xfrac}
\usepackage{datetime}
\usepackage{ulem}
\usepackage{enumerate}
\usepackage[table]{xcolor}

\newcommand\numberthis{\addtocounter{equation}{1}\tag{\theequation}}

\newcommand{\rank}{\mathrm{rk\,}}

\newcommand{\R}{\mathrm{R}}

\newcommand{\C}{\mathrm{C}}

\newcommand{\N}{\mathrm{N}}

\newcommand{\im}{\mathrm{im}\,}

\newcommand{\trace}{\mathrm{trace\,}}

\newenvironment{amatrix}[1]{%
	\left(\begin{array}{@{}*{#1}{c}|c@{}}
}{%
	\end{array}\right)
}

%\setlength{\mathindent}{0pt}

\theoremstyle{definition}
\newtheorem{example}{Example}
\newtheorem{definition}{Definition}

\theoremstyle{theorem}
\newtheorem{theorem}{Theorem}

\newenvironment{solution}
{\begin{proof}[Solution]\let\qed\relax}
	{\end{proof}}

\DeclareMathOperator{\vspan}{\mathrm{span}} %declares span operator for matrices

%opening
\title{Recitation 14}
\author{}
\date{\formatdate{21}{1}{2015}}

\begin{document}

\maketitle
%\setlength{\mathindent}{0pt}

\tableofcontents

\newpage
\section{Orthogonal Complement}

\begin{example}
	Find the orthogonal complement of 
	\begin{equation*}
		U = \vspan
			\left\lbrace
				\begin{pmatrix}
					1\\
					1\\
					0\\
				\end{pmatrix}
				,
				\begin{pmatrix}
					0\\
					1\\
					1\\
				\end{pmatrix}
			\right\rbrace
	\end{equation*}
	with inner product
	\begin{equation*}
		\langle u, v, \rangle = u^T v
	\end{equation*}
	over $\mathbb{R}^3$.
\end{example}

\begin{solution}
	\begin{align*}
		U^{\perp} &=
			\left\{
				\left.
				\begin{pmatrix}
					x\\
					y\\
					z\\
				\end{pmatrix}
				\right|
				\left\langle
					\begin{pmatrix}
						x\\
						y\\
						z\\
					\end{pmatrix}
					,
					\begin{pmatrix}
						1\\
						1\\
						0\\
					\end{pmatrix}
				\right\rangle
				=
				\left\langle
					\begin{pmatrix}
						x\\
						y\\
						z\\
					\end{pmatrix}
					,
					\begin{pmatrix}
						1\\
						1\\
						0\\
					\end{pmatrix}
				\right\rangle
				=
				0
			\right\}\\
		&= 
			\left\{
				\left.
				\begin{pmatrix}
					x\\
					y\\
					z\\
				\end{pmatrix}
				\right|
				x + y = y + z = 0
			\right\}\\
		&= 
			\left\{
				\begin{pmatrix}
					1\\
					-1\\
					1\\
				\end{pmatrix}
			\right\}
	\end{align*}
\end{solution}

\begin{example}
	Find the orthogonal complement of 
	\begin{equation*}
		U = \{q(x) \in V | q(x) + q(-x) = 0\}
	\end{equation*}
	with inner product
	\begin{equation*}
	\langle p(x), q(x) \rangle = \dfrac{1}{2} \int\limits_{-1}^{1} p(x) q(x) \dif x
	\end{equation*}
	over $\mathbb{R}_2[x]$.
\end{example}

\begin{solution}
	\begin{align*}
		U &= \{a + bx + cx^2 | a + bx + cx^2 + a - bx + cx^2 = 0\}\\
		&= \vspan\{x\}\\
		\therefore U^{\perp} &= \{a + bx + cx^2 | \langle a + bx + cx^2, x \rangle = 0\}\\
		&= \vspan\{1, x^2\}
	\end{align*}
\end{solution}

\begin{example}
	Find the orthogonal projection of
	$
		B = 
		\begin{pmatrix}
			1 & 1\\
			0 & 1\\
		\end{pmatrix}
	$
	on
	\begin{equation*}
		U = \{s \in V | s = s^T\} \subset M_{2 \times 2}(\mathbb{R})
	\end{equation*}
	with respect to
	\begin{equation*}
		\langle A, B \rangle = \trace \left( A^T \cdot B \right)
	\end{equation*}
\end{example}

\begin{solution}
	\begin{align*}
		U &= 
			\left\{
				\begin{pmatrix}
					a & b\\
					b & d\\
				\end{pmatrix}
			\right\}\\
		&= \vspan
			\left\{
				\begin{pmatrix}
					1 & 0\\
					0 & 0\\
				\end{pmatrix}
				,
				\begin{pmatrix}
					0 & 1\\
					1 & 0\\
				\end{pmatrix}
				,
				\begin{pmatrix}
					0 & 0\\
					0 & 1\\
				\end{pmatrix}
			\right\}\\
	\end{align*}
	\begin{align*}
		B &= B_U + B_{U^{\perp}}\\
		\therefore B - B_U &= B_{U^{\perp}}
	\end{align*}
	Therefore,
	\begin{align*}
		\langle B - B_U, s_1 \rangle &= 0\\
		\langle B - B_U, s_2 \rangle &= 0\\
		\langle B - B_U, s_3 \rangle &= 0
	\end{align*}
	Solving,
	\begin{align*}
		B_U &= 
			\begin{pmatrix}
				1 & \sfrac{1}{2}\\
				\sfrac{1}{2} & 1\\
			\end{pmatrix}
	\end{align*}
\end{solution}

\begin{example}
	Prove that for the Fibonacci series staring with 0 and 1, the $n^{\textnormal{th}}$ element is
	\begin{equation*}
		F_n = \dfrac{1}{\sqrt{5}} \left( \left( \dfrac{1 + \sqrt{5}}{2} \right)^n - \left( \dfrac{1 - \sqrt{5}}{2} \right)^n \right)
	\end{equation*}
\end{example}

\begin{solution}
	\begin{align*}
		\begin{pmatrix}
			F_1\\
			F_0\\
		\end{pmatrix}
		&=
		\begin{pmatrix}
			1\\
			0\\
		\end{pmatrix}\\
		\begin{pmatrix}
			F_2\\
			F_1\\
			\end{pmatrix}
		&=
			\begin{pmatrix}
			1\\
			1\\
		\end{pmatrix}\\
		&\vdots\\
			\begin{pmatrix}
				F_{n + 1}\\
				F_n\\
			\end{pmatrix}
		&=
			\begin{pmatrix}
				1 & 1\\
				1 & 0\\
			\end{pmatrix}
			\begin{pmatrix}
				F_n\\
				F_{n - 1}\\
			\end{pmatrix}\\
		&= 			
			\begin{pmatrix}
				1 & 1\\
				1 & 0\\
			\end{pmatrix}^n
			\begin{pmatrix}
				F_1\\
				F_0\\
			\end{pmatrix}
	\end{align*}
	If $A = P^{-1} D P$,
	\begin{align*}
		A^n &= P D^n P^{-1}
	\end{align*}
	Let
	\begin{align*}
		A &= 
			\begin{pmatrix}
				1 & 1\\
				1 & 0\\
			\end{pmatrix}\\
		\therefore p_A(\lambda) &= \lambda^2 - \lambda - 1\\
		\therefore \lambda &= \dfrac{1 \pm \sqrt{5}}{2}\\
		&= \phi_{\pm}
	\end{align*}
	Therefore,
	\begin{align*}
		V_{\phi_{+}} &= \N
			\begin{pmatrix}
				1 - \phi_{+} & 1\\
				1 & - \phi_{+}
			\end{pmatrix}\\
		&= \vspan
			\left\{
				\begin{pmatrix}
					\phi_{+}\\
					1\\
				\end{pmatrix}
			\right\}\\
		V_{\phi_{+}} &= \N
			\begin{pmatrix}
				1 - \phi_{-} & 1\\
				1 & - \phi_{-}
			\end{pmatrix}\\
		&= \vspan
			\left\{
				\begin{pmatrix}
					\phi_{-}\\
					1\\
				\end{pmatrix}
			\right\}
	\end{align*}
	Therefore,
	\begin{align*}
		D &= 
			\begin{pmatrix}
				\phi_{+} & 0\\
				0 & \phi_{-}\\
			\end{pmatrix}\\
		P &= 
			\begin{pmatrix}
				\phi_{+} & \phi_{-}\\
				1 & 1\\
			\end{pmatrix}\\
		\therefore P^{-1} &= \dfrac{1}{\phi_{+} - \phi_{-}}
			\begin{pmatrix}
				1 & -\phi_{-}\\
				-1 & \phi_{+}\\
			\end{pmatrix}\\
		&= \dfrac{1}{\sqrt{5}}
			\begin{pmatrix}
				1 & -\phi_{-}\\
				-1 & \phi_{+}\\
			\end{pmatrix}
	\end{align*}
	
	\begin{align*}
		A^n &= P D^n P^{-1}\\
		&= \dfrac{1}{\sqrt{5}}
			\begin{pmatrix}
				{\phi_{+}}^{n+1} - {\phi_{-}}^{n+1} & {\phi_{+}}^n \cdot \phi_{-} - \phi_{+} \cdot {\phi_{+}}^{n+1}\\
				{\phi_{+}}^n - {\phi_{-}}^n & {\phi_{+}}^n \cdot \phi_{-} - \phi_{+} \cdot {\phi_{-}}^n\\
			\end{pmatrix}
	\end{align*}
	Therefore,
	\begin{align*}
			\begin{pmatrix}
				F_{n+1}\\
				F_n\\
			\end{pmatrix}
		&= \dfrac{1}{\sqrt{5}} A^n 
			\begin{pmatrix}
				1\\
				0\\
			\end{pmatrix}\\
		&= \dfrac{1}{\sqrt{5}}
			\begin{pmatrix}
				{\phi_{+}}^{n+1} - {\phi_{-}}^{n+1}\\
				{\phi_{+}}^n - {\phi_{-}}^n\\
			\end{pmatrix}\\
		\therefore F_n &= \dfrac{1}{\sqrt{5}} \left( {\phi_{+}}^n - {\phi_{-}}^n \right)\\
		&= \dfrac{1}{\sqrt{5}} \left( \left( \dfrac{1 + \sqrt{5}}{2} \right)^n - \left( \dfrac{1 - \sqrt{5}}{2} \right)^n \right)
	\end{align*}
\end{solution}

\begin{example}
	Find the projection of $a = (1,2,3)$ on the subspace 
	\begin{equation*}
		U = \left\{ (a_1, a_2, a_3) | a_1 + a_2 + a_3 = 0 \right\}
	\end{equation*}
\end{example}

\begin{solution}
	\begin{align*}
		U &= \left\{ (- a_2 - a_3, a_2, a_3) \right\}\\
		&= \left\{ -a_2 (1, -1, 0) - a_3 (1, 0, -1) \right\}\\
		&= \vspan \left\{ (1, -1, 0), (1, 0, -1) \right\}
	\end{align*}
	
	\begin{align*}
		a &= a_U + a_{u^{\perp}}\\
		a_{U^{\perp}} &= a - a_U
	\end{align*}
	Therefore,
	\begin{align*}
		\langle a - a_U, u_1 \rangle &= 0\\
		\langle a - a_U, u_2 \rangle &= 0
	\end{align*}
	Solving,
	\begin{align*}
		a_U &= (-1, 0, 1)
	\end{align*}
\end{solution}

\end{document}          