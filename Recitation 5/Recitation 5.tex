\documentclass[fleqn, a4paper, 12pt]{article}
\setcounter{secnumdepth}{4}
\usepackage{amsmath, amssymb, amsthm}
\usepackage{mathtools}
\usepackage{datetime}
\usepackage{ulem}
\usepackage{enumerate}

\newcommand\numberthis{\addtocounter{equation}{1}\tag{\theequation}}

\newcommand{\rank}{\mathrm{rk\,}}

\newenvironment{amatrix}[1]{%
	\left(\begin{array}{@{}*{#1}{c}|c@{}}
}{%
	\end{array}\right)
}

%\setlength{\mathindent}{0pt}

\theoremstyle{definition}
\newtheorem{example}{Example}
\newtheorem{definition}{Definition}

\theoremstyle{theorem}
\newtheorem{theorem}{Theorem}

\newenvironment{solution}
{\begin{proof}[Solution]\let\qed\relax}
	{\end{proof}}

%opening
\title{Recitation 5}
\author{}
\date{\formatdate{26}{11}{2014}}

\begin{document}

\maketitle
%\setlength{\mathindent}{0pt}

\tableofcontents

\newpage
\section{}

\begin{example}
	Let $B = \{v_1, v_2, v_3, v_4\}$ be a base of a vector space $V$. If $C = \{v_1 + v_2, v_2 + v_3, v_3 + v_4, v_4 + v_1\}$ a base of $V$?
	\begin{solution}
		Let
		\begin{align*}
			\alpha (v_1 + v_2) + \beta (v_2 + v_3) + \gamma (v_3 + v_4) + \delta (v_4 + v_1) &= 0\\
			\therefore v_1 (\delta + \alpha) + v_2 (\alpha + \beta) + v_3 (\beta + \gamma) + v_4 (\gamma + \delta) &= 0
		\end{align*}
		As $B = \{v_1, v_2, v_3, v_4\}$ is a basis of $V$,
		\begin{align*}
			\delta + \alpha &= 0\\
			\alpha + \beta &= 0\\
			\beta + \gamma &= 0\\
			\gamma + \delta &= 0
		\end{align*}
		If 
		\begin{align*}
			\alpha &= -1 & \beta &= 1 & \gamma &= -1 & \delta &= 1
		\end{align*}
		the above system of equations hold.\\
		Therefore, $C$ is linearly dependent. Therefore, it is not a base of $V$.	\end{solution}
\end{example}


\begin{definition}[Row space]
	The row space of $A \in M_{n \times m}(\mathbb{F})$ is a subspace of $\mathbb{F}^m$ spanned by the rows of $A$. The row space of $A$ is denoted by $R(A)$.
\end{definition}

\begin{definition}[Column space]
	The column space of $A \in M_{n \times m}(\mathbb{F})$ is a subspace of $\mathbb{F}^n$ spanned by the columns of $A$. The column space of $A$ is denoted by $C(A)$.
\end{definition}

\begin{definition}
	The rank of $A \in M_{n \times m}(\mathbb{F})$ is defined as
	\begin{enumerate}
		\item the number of dependent variables in the REF of $A$
		\item the number of non-zero rows on the REF of $A$
	\end{enumerate}
	\begin{equation*}
		\rank(A) = \dim (C(A)) = \dim (R(A))
	\end{equation*}
\end{definition}
\end{document}
