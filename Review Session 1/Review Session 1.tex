\documentclass[fleqn, a4paper, 12pt]{article}
\usepackage{amsmath, amssymb, amsthm, thmtools, amsfonts, commath}
\usepackage{datetime}
\usepackage{hyperref}
\usepackage{ulem}
\usepackage{tikz}
\usepackage{enumerate, enumitem}
\usepackage{cancel}
\usepackage{xfrac}
%\usepackage{background}
\setcounter{secnumdepth}{4}

\newcommand\numberthis{\addtocounter{equation}{1}\tag{\theequation}}

\DeclareMathOperator{\vspan}{\mathrm{span}} %declares span operator for matrices

\newenvironment{amatrix}[1]{%declares augmented matrix environment
	\left(\begin{array}{@{}*{#1}{c}|c@{}}
	}{%
\end{array}\right)
} 

\theoremstyle{definition}
\newtheorem{example}{Example} %defines example environment
\newtheorem{definition}{Definition} %defines definition environment

\theoremstyle{theorem}
\newtheorem{theorem}{Theorem} %defines theorem environment
\newtheorem{corollary}{Corollary}

\theoremstyle{remark}
\newtheorem{remark}{Remark}
\newtheorem{case}{Case}

\newcommand{\suchthat}{\mathrm{\,s.t.\,}}

\newcommand{\R}{\mathrm{R}}

\newcommand{\C}{\mathrm{C}}

\newcommand{\N}{\mathrm{N}}

\newcommand{\rr}{\mathrm{rr}}

\newcommand{\im}{\mathrm{im}\,}

\newcommand{\distance}{\mathrm{d}}

\newenvironment{solution} %declares solution environment and removes qed at end
	{\begin{proof}[Solution]\let\qed\relax}
	{\end{proof}}

\makeatletter
\@addtoreset{section}{part} %resets section numbers in new part
\makeatother

\makeatletter
\@addtoreset{theorem}{part} %resets theorem numbers in new part
\makeatother

\makeatletter
\@addtoreset{corollary}{theorem} %resets corollary numbers after a theorem
\makeatother

\numberwithin{corollary}{theorem}

\numberwithin{equation}{theorem}

\makeatletter %changes spacing between rows of matrices to fit fractions
\newif\ifcenter@asb@\center@asb@false
\def\center@arstrutbox{%
	\setbox\@arstrutbox\hbox{$\vcenter{\box\@arstrutbox}$}%
}
\newcommand*{\CenteredArraystretch}[1]{%
	\ifcenter@asb@\else
	\pretocmd{\@mkpream}{\center@arstrutbox}{}{}%
	\center@asb@true
	\fi
	\renewcommand{\arraystretch}{#1}%
}
\makeatother

\title{Review Session 1}
\author{Aakash Jog}
\date{\formatdate{21}{1}{2015}}

\begin{document}

\maketitle

\begin{example}
	Let $T : \mathbb{R}^3 \to \mathbb{R}^3$ be a linear operator given by
	\begin{equation*}
		T
			\begin{pmatrix}
				x\\
				y\\
				z\\
			\end{pmatrix}
		=
			\begin{pmatrix}
				2x + y\\
				y - z\\
				2y + 4z\\
			\end{pmatrix}
	\end{equation*}
	Is $T$ diagonalizable?
\end{example}

\begin{solution}
	Let $B = \{e_1, e_2, e_3\}$ be the standard basis of $\mathbb{R}$.\\
	Therefore,
	\begin{align*}
		T(e_1) &=
			\begin{pmatrix}
				2\\
				0\\
				0\\
			\end{pmatrix}\\
		T(e_2) &= 
			\begin{pmatrix}
				1\\
				1\\
				2\\
			\end{pmatrix}\\
		T(e_3) &=
			\begin{pmatrix}
				0\\
				-1\\
				4\\
			\end{pmatrix}
	\end{align*}
	Therefore,
	\begin{align*}
		T(e_1) &= 2 e_1\\
		T(e_2) &= e_1 + e_2 + 2 e_3\\
		T(e_3) &= -e_2 + 4 e_3\\
		\therefore [T]_B &= 
			\begin{pmatrix}
				2 & 1 & 0\\
				0 & 1 & -1\\
				0 & 2 & 4\\
			\end{pmatrix}
	\end{align*}
	Therefore,
	\begin{align*}
		p_T(\lambda) &= 
			\begin{vmatrix}
				2 - \lambda & 1 & 0\\
				0 & 1 - \lambda & -1\\
				0 & 2 & 4 - \lambda\\
			\end{vmatrix}\\
		&= (2 - \lambda) \left( (1 - \lambda) (4 - \lambda) + 2 \right)\\
		&= (2 - \lambda) (6 - 5 \lambda + \lambda^2 )\\
		&= - (\lambda - 2) (\lambda - 2) (\lambda - 3)\\
		\therefore \lambda &= 2, 3
	\end{align*}
	\begin{align*}
		V_2 &= \N
			\begin{pmatrix}
				0 & 1 & 0\\
				0 & -1 & -1\\
				0 & 2 & 2\\
			\end{pmatrix}\\
		&= \N
			\begin{pmatrix}
				0 & 1 & 0\\
				0 & 1 & 1\\
				0 & 0 & 0\\
			\end{pmatrix}\\
		&= \vspan
			\left\lbrace
				\begin{pmatrix}
					1\\
					0\\
					0\\
				\end{pmatrix}
			\right\rbrace
	\end{align*}
	Therefore, $\dim V_2 = 1$. Hence the transformation is not diagonalizable.
\end{solution}

\begin{example}
	Let 
	\begin{equation*}
		A =
			\begin{pmatrix}
				1 & b & b^2\\
				0 & a & 2ab\\
				0 & 0 & a^2\\
			\end{pmatrix}
	\end{equation*}
	Find all $(a,b)$ such that $A$ is diagonalizable. Find all $(a,b)$ such that $A$ is invertible, and for these $(a,b)$, find $A^{-1}$.
\end{example}

\begin{solution}
	\begin{align*}
		p_A(x) &= 
			\begin{vmatrix}
				x - 1 & b & b^2\\
				0 & x - a & 2ab\\
				0 & 0 & x - a^2\\
			\end{vmatrix}\\
		&= (x - 1) (x - a) (x - a^2)\\
		\therefore \lambda &= 1, a, a^2
	\end{align*}\\
	If $a \neq 0$, $a \neq 1$, $a \neq -1$, the algebraic and geometric multiplicities are 1. Therefore, $A$ is diagonalizable.\\
	If $a = 0$,
	\begin{align*}
		\lambda &= 0, 1
	\end{align*}
	Therefore,
	\begin{align*}
		V_0 &= \N
			\begin{pmatrix}
				-1 & -b & -b^2\\
				0 & 0 & 0\\
				0 & 0 & 0\\
			\end{pmatrix}\\
		&= \N
			\begin{pmatrix}
				1 & b & b^2\\
				0 & 0 & 0\\
				0 & 0 & 0\\
			\end{pmatrix}\\
		&= \vspan
			\left\lbrace
				\begin{pmatrix}
					-b\\
					1\\
					0\\
				\end{pmatrix}
				,
				\begin{pmatrix}
					-b^2\\
					0\\
					1\\
				\end{pmatrix}
			\right\rbrace
	\end{align*}
	Therefore, for $a = 0$, $A$ is diagonalizable.\\
	~\\
	If $a = 1$,
	\begin{align*}
		\lambda &= 1
	\end{align*}
	If $b = 0$,
	\begin{align*}
		V_1 &= \N
			\begin{pmatrix}
				0 & 0 & 0\\
				0 & 0 & 0\\
				0 & 0 & 0\\
			\end{pmatrix}\\
		&=
			\left\lbrace
				\begin{pmatrix}
					1\\
					0\\
					0\\
				\end{pmatrix}
				,
				\begin{pmatrix}
					0\\
					1\\
					0\\
				\end{pmatrix}
				,
				\begin{pmatrix}
					0\\
					0\\
					1\\
				\end{pmatrix}
			\right\rbrace
	\end{align*}
	If $b \neq 0$,
	\begin{align*}
		V_1 &= \N
			\begin{pmatrix}
				0 & b & b^2\\
				0 & 0 & 2b\\
				0 & 0 & 0\\
			\end{pmatrix}\\
		&=
			\left\lbrace
				\begin{pmatrix}
					1\\
					0\\
					0\\
				\end{pmatrix}
			\right\rbrace
	\end{align*}
	Therefore, for $a = 1$, $A$ is diagonalizable if $b = 0$.\\
	~\\
	If $a = -1$,
	\begin{align*}
		p_A(x) &= (x - 1)^2 (x + 1)\\
		\therefore \lambda &= -1, 1
	\end{align*}
	Therefore,
	\begin{align*}
		V_1 &= \N
			\begin{pmatrix}
				0 & b & b^2\\
				0 & -2 & -2b\\
				0 & 0 & 0\\
			\end{pmatrix}\\
		&= \N
			\left\lbrace
				\begin{pmatrix}
					1\\
					0\\
					0\\
				\end{pmatrix}
				,
				\begin{pmatrix}
					0\\
					-b\\
					1\\
				\end{pmatrix}
			\right\rbrace
	\end{align*}
	Therefore, for $a = -1$, $A$ is diagonalizable.\\
	~\\
	Therefore, $A$ is diagonalizable for all $(a,b) \in \mathbb{R}^2 \setminus (1, b \neq 0)$.\\
	~\\
	~\\
	For $A$ to be invertible, $\det(A) \neq 0$. Therefore, $A$ is invertible if and only if $a \neq 0$.\\
	\begin{align*}
		\therefore A^{-1} &=
			\begin{pmatrix}
				1 & -\sfrac{b}{a} & \sfrac{b^2}{a^2}\\
				0 & \sfrac{1}{a} & -\sfrac{2b}{a^2}\\
				0 & 0 & \sfrac{1}{a^2}\\
			\end{pmatrix}
	\end{align*}
\end{solution}

\end{document}
